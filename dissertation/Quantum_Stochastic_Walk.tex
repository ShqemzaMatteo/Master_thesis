\section{Quantum Stochastic Random Walk}\label{C_Quantum Stochastic Walk}

One of the early approaches to an Open Quantum Walk on networks was proposed by Whitfield, Rodr\'iguez-Rosario and Aspuru-Guzik. \cite{QSW}. They defined a quantum walk on a network in contact with a thermal bath with the dynamics described by a Lindblad master equation. In this framework, the jump operators are proportional to the adjacency matrix $A_{ij}$ of the network. The thermal bath introduces noise into the dynamics, causing a different motion respect to the von Neumann equation \eqref{Von Neumann equation}. The system dissipation is reminiscent of the classic random walk.

Let us consider a quantum walk on a network $G(N,M)$, the system in contact with a thermal bath.
Let us introduce a Hilbert space $\mathcal{H}$ with an orthonormal basis $\{\ket{i}\}_{i<N}$, where each element $\ket{i}$ corresponds to the node $i$, satisfying $\braket{i}{j}=\delta_{ij}$. The system is described by a density matrix $\hat \rho$ whose evolution follows the Lindblad master equation \eqref{Lindblad_master_equation}.
The Laplacian operator $\hat L$, defined as in equation \eqref{Laplacian_operator}, serves as the Hamiltonian $\hat H$, while the jump operators $\{\hat J_k\}_{k<M}$ represent the thermal jumps between two node linked together. For convenience, we will denote the jump operator with two indices referring to the starting node $j$ and the ending node $i$ of the jump. Therefore, the jump operators are $\hat J_{ij} = \ket{i}\bra{j}$. The damping rates are given by $\gamma_{ij} =A_{ij}/d_i$, like the transition rates in the classical random walk \eqref{transition_rates}.
The master equation can be expressed as follows:
\begin{equation}\label{stochastic_lindblad_master}
    \frac{d}{dt}\hat \rho = -\frac{i}{2}\left[\hat L,\hat\rho\right] + \sum_{ij}\gamma_{ij}\left[\hat J_{ij} \hat\rho\hat J_{ij}^\dagger -\frac{1}{2} \left\{ \hat J_{ij}^\dagger \hat J_{ij}, \hat\rho\right\}\right],
\end{equation}
where $[\cdot,\cdot]$ and $\{\cdot,\cdot\}$ denote the commutator and the anticommutator respectively.
The equation \eqref{stochastic_lindblad_master} is composed of two distinct terms. 
The first term $\mathcal{L}^{qm}\left[\hat\rho(t)\right] = -i\left[H,\hat\rho\right]$, called coherent dynamics, corresponds to the quantum walk dynamics. In contrast, the second term  $\mathcal{L}^{cl}\left[\hat\rho(t)\right] = \sum_i \gamma_i \left(\hat J_i \hat\rho \hat J^\dagger_i - \frac{1}{2}\left\{ \hat J^\dagger_i\hat J_i, \hat\rho\right\} \right)$, denoted as decoherent dynamics, encodes the dissipation. 
When $\gamma_{ij} = 0$ we recover the von Neumann equation for the quantum walk \eqref{Von Neumann equation}. 
$\mathcal{L}\left[\hat\rho(t)\right]$ act as a superoperator in the space of the density matrix.

In the Fock-Liouville space, the quantum system evolves according to the equation
\begin{equation}
    \sket{\rho(t)}  = U(t,0) \sket{\rho(0)}
\end{equation} 
where the evolution operator is defined as \cite{Domino}
\begin{equation}
    \begin{split}
        \hat U(t,0) = \exp&\left\{-it\left(\hat H\otimes\mathbb{I}-\mathbb{I}\otimes\hat H\right)\right.\\
        &+\left. t\sum_{ij}\gamma_{ij}\left[ \hat J_{ij}\otimes\hat J^\dagger_{ij}-\frac{1}{2}\hat J_{ij}\hat J^\dagger_{ij}\otimes\mathbb{I}-\frac{1}{2}\mathbb{I}\otimes\hat J_{ij}\hat J^\dagger_{ij}\right]\right\}.
    \end{split}
\end{equation}

The master equation \eqref{stochastic_lindblad_master} contains both the quantum and classical aspects of a random walk over a network. Thus, the particle can go through both quantum and classical transitions. As a matter of fact, the classical random walk behavior emerges when we consider the evolution of the diagonal elements of the density matrix under the dissipative part alone. Let $\rho = \ket{k}\bra{k}$ represent the density matrix of a system localized at node $k$. Its evolution is given by
\begin{equation}
    \begin{split}
        \mathcal{L}^{cl}\ket{k}\bra{k} &= \sum_{ij}\gamma_{ij}\left[\hat J_{ij} \ket{k}\bra{k}\hat J_{ij}^\dagger -\frac{1}{2} \left\{ \hat J_{ij}^\dagger \hat J_{ij}, \ket{k}\bra{k}\right\}\right]\\
        %&= \sum_{ij}\left[\sqrt{A_{ij}}\ket{i}\braket{j}{a}\braket{a}{j}\bra{i}\sqrt{A_{ij}} -\frac{1}{2}\ket{i}\sqrt{A_{ia}}\braket{j}{j}\sqrt{A_{ia}}\braket{i}{a}\bra{a}\right.\\ 
        %&\quad \left. - \frac{1}{2} \ket{a}\braket{a}{i}\sqrt{A_{ia}}\braket{j}{j}\sqrt{A_{ia}}\bra{i}\right]\\
        &= \sum_{i}\left[\gamma_{ik} \ket{i}\bra{i} -\gamma_{ik}\ket{k}\bra{k}\right]\\
        &=\sum_{i} \left(\gamma_{ki} - \gamma_{ik}\delta_{ki}\right) \ket{i}\bra{i} = -\sum_i L_{ki} \ket{i}\bra{i}.
    \end{split} 
\end{equation} 
This expression recovers the dynamics of the classical random walk over the network.
Next, considering the off-diagonal terms, their evolution is described by
\begin{equation}
    \begin{split}
        \mathcal{L}^{cl}\ket{k}\bra{l} &= \sum_{ij}\gamma_{ij}\left[\hat J_{ij} \ket{k}\bra{l}\hat J_{ij}^\dagger -\frac{1}{2} \left\{ \hat J_{ij}^\dagger \hat J_{ij}, \ket{k}\bra{l}\right\}\right]\\
        %&= \sum_{ij}\left[\sqrt{A_{ij}}\ket{i}\braket{j}{k}\braket{l}{j}\bra{i}\sqrt{A_{ij}} -\frac{1}{2}\ket{i}\sqrt{A_{ik}}\braket{j}{j}\sqrt{A_{ik}}\braket{i}{k}\bra{l}\right.\\ 
        %&\quad\left. - \frac{1}{2} \ket{k}\braket{l}{i}\sqrt{A_{ik}}\braket{j}{j}\sqrt{A_{ik}}\bra{i}\right]\\
        &= \sum_{j}\left[-\frac{1}{2} \gamma_{jk}\ket{k}\bra{l} - \frac{1}{2} \gamma_{jl}\ket{k}\bra{l}\right]\\
        &= -\ket{k}\bra{l}.
    \end{split} 
\end{equation}

The operator $\mathcal{L}^{cl}$ does not mix the diagonal terms with the off-diagonal ones, allowing us to separate the superoperator into two blocks: one for the diagonal elements and the another for the off-diagonal ones.
Thus, the superoperator $\mathcal{L}^{cl}$ has a diagonal form with spectrum given by $\sigma^{cl} = -(\lambda_1,...,\lambda_N,1,...,1))$, where $\lambda_i$ are the eigenvalue of the Laplacian matrix \cite{Bruderer_Plenio}.
If the network satisfies the detailed balance condition \eqref{detail_condition}, the Laplacian matrix has a zero eigenvalue. Therefore, the superoperator $\mathcal{L}^{cl}$ will also have a zero eigenvalue indicating the presence of a stationary distribution.

\begin{comment}
    \begin{equation}
    \begin{split}
    \mathcal{L}^{qm}\ket{k}\bra{l} =& -i\hat L\ket{k}\bra{l} + i \ket{k}\bra{l}\hat L\\
    &=\frac{i}{2}\sum_{ij} - L_{ij}\ket{i}\braket{j}{k}\bra{l} + \ket{k}\braket{l}{i}\bra{j}\\
    &=\frac{1}{2}\sum_i -L_{ik} \ket{i}\bra{l} + \sum_i L_{li}\ket{k}\bra{i}\\
\end{split}
\end{equation}

Instead the diagonal terms
\begin{equation}
\begin{split}
\mathcal{L}^{qm}\ket{l}\bra{l} =& -i\hat L\ket{l}\bra{l} + i \ket{l}\bra{l}\hat L\\
&=\frac{i}{2}\sum_{ij} - L_{ij}\ket{i}\braket{j}{l}\bra{l} + \ket{l}\braket{l}{i}\bra{j}\\
&=\frac{1}{2}\sum_i -L_{il} \ket{i}\bra{l} + \sum_i L_{li}\ket{l}\bra{i} = 0\\
\end{split}
\end{equation}
\end{comment}

\subsection{Stationary Distribution}

As mention before, the master equation \eqref{stochastic_lindblad_master} has a stationary matrix $\hat\rho^*$ for the quantum stochastic walk.
In order to find it, we first consider only the dissipative dynamics, which decouples the diagonal and off-diagonal terms. The evolution of the diagonal elements is described by:
\begin{equation}
    \frac{d}{dt}\rho_{ii} = \sum_j\left[\gamma_{ij}\rho_{jj}(t) - \gamma_{ji}\rho_{ii}(t)\right],
\end{equation}

The stationary distribution must satisfy the detailed balance condition, namely
\begin{equation}
    \gamma_{ij}\rho_{jj}(t) = \gamma_{ji}\rho_{ii}.
\end{equation}
Because the damping rates for this system are symmetric, the diagonal entries $\rho_{ii}$ must be  equal. 
In contrast, considering the vector $\sket{\rho}$, the block corresponding to the off-diagonal part of $\mathcal{L}$ is already an eigenstate with eigenvalue $1$. Thus, the off-diagonal terms must be equal to zero.
The stationary density matrix can then be expressed as
\begin{equation}\label{QSW_stationary_distribution}
    \hat\rho^* = \frac{1}{N}\begin{pmatrix}
        1&&0\\
        &\ddots&\\
        0&&1\\
    \end{pmatrix}.
\end{equation}
The stationary density matrix has maximal von Neumann entropy 
\begin{equation}
    S\left(\hat\rho^*\right) = \ln N.
\end{equation}

Because the density matrix \eqref{QSW_stationary_distribution} commutes with the Laplacian matrix, it is indeed the stationary density matrix for the dynamics described by \eqref{stochastic_lindblad_master}.

However, this framework do not introduce an temperature. Therefore, in order to explain the network's entropy we need another framework.


\begin{comment}
    Instead, adding also the coherent part, we need to go in the basis $\{\ket{\lambda}\}$ eigenvector of $\hat L$. 
    The decoherent part becomes
    \begin{equation}
    \begin{split}
    \mathcal{L}^{cl}\ket{\lambda}\bra{\lambda} &= \sum_{ij}\gamma_{ij}\left[\hat J_{ij} \ket{\lambda}\bra{\lambda}\hat J_{ij}^\dagger -\frac{1}{2} \left\{ \hat J_{ij}^\dagger \hat J_{ij}, \ket{\lambda}\bra{\lambda}\right\}\right]\\
        &= \sum_{ij}\gamma_{ij}\left[\ket{i}\braket{j}{\lambda}\braket{\lambda}{j}\bra{i} -\frac{1}{2}\ket{j}\braket{i}{i}\braket{j}{\lambda}\bra{\lambda} - \frac{1}{2}\ket{\lambda}\braket{\lambda}{j}\braket{i}{i}\bra{j} \right]\\
        & = \sum_{ij}\gamma_{ij}\left[|\braket{j}{\lambda}|^2\ket{i}\bra{i} -\frac{1}{2}\ket{j}\braket{j}{\lambda}\bra{\lambda} - \frac{1}{2}\ket{\lambda}\braket{\lambda}{j}\bra{j} \right]\\
        &= \sum_{ij}\left[\pi_{ij}\rho_j\ket{i}\bra{i} -\frac{\pi_{ij}}{2}\braket{j}{\lambda}\ket{j}\bra{\lambda} - \frac{\pi_{ij}}{2}\braket{\lambda}{j}\ket{\lambda}\bra{j} \right]
    \end{split}
\end{equation}
\end{comment}
