\chapter{Quantum Stochastic Walk}\label{C_Quantum Stochastic Walk}

In chapter \ref{Network_Theory} we have introduced two possibility for a random walk: one using classic particles, the other using quantum ones. But they have different properties that distinguish them self
In this chapter, we propose a way to unify the two approach using the Lindblad master equation. This new approach generates a more general random walk on network that include the properties of both the system \cite{QSW}

\section{Quantum master equation}
Let we consider a quantum system in contact with a thermal bath that randomly interact with; the density matrix of the system $\hat\rho$ should follow the Lindblad master equation \eqref{Lindblad_master_equation}
\begin{equation}\label{Lindblad_equation_(stoc_chapter)}
    \frac{d}{dt}\hat\rho =\mathcal{L}\left[\hat\rho\right] = -i\left[\hat H,\hat\rho\right] + \sum_i \gamma_i \left(\hat J_i \hat\rho \hat J^\dagger_i - \frac{1}{2}\left\{ \hat J^\dagger_i\hat J_i, \hat\rho\right\} \right),
\end{equation}

where $[\cdot,\cdot]$ and $\{\cdot,\cdot\}$ are respectively the commutator and anticommutator; $\hat H$ is the Hamiltonian; $\{\hat J_i\}_i$ are the jump operators which describe the interaction with the thermal bath and they are dissipative part of the dynamics; $\gamma_i$ are the dumping rates, they must be non negative and real, when $\gamma_i = 0$ we recover the Von Neumann equation \eqref{Von Neumann equation}. $\mathcal{L}\left[\hat\rho\right]$ is a superoperator that act in the space of the density matrix.


The equation \eqref{Lindblad_equation_(stoc_chapter)} is composed by two distinct terms: the first term, also called coherent dynamics,
$\mathcal{L}^{qm}\left[\hat\rho(t)\right] = -i\left[H,\hat\rho\right]$ 
is equal to the quantum walk dynamics; instead, the second 
$\mathcal{L}^{cl}\left[\hat\rho(t)\right] = \sum_i \gamma_i \left(\hat J_i \hat\rho \hat J^\dagger_i - \frac{1}{2}\left\{ \hat J^\dagger_i\hat J_i, \hat\rho\right\} \right)$, 
called decoherent dynamics, encodes the dissipation. 
As in the quantum random walk, the Hamiltonian $\hat H$ is the Laplacian operator $\hat L$.
We can define a set of jump operator $\{\hat J_{ij} \}_{i,j=0}^{N}$ as general measurement and a set of coefficients $\{\gamma_{ij}\}_{i,j=0}^{N}$ that behave as the Adjacency matrix, namely $\gamma_{ij} =A_{ij}$ and $\hat J_{ij} = \ket{i}\bra{j}$.
From that, we obtain a master equation that contains both the quantum and classical aspect of a diffusion over a network as
\begin{equation}\label{stochastic_lindblad_master}
    \frac{d}{dt}\hat \rho = -i(1-\omega)\left[\hat H,\hat\rho\right] + \omega\sum_{ij}\gamma_{ij}\left[\hat J_{ij} \hat\rho\hat J_{ij}^\dagger -\frac{1}{2} \left\{ \hat J_{ij}^\dagger \hat J_{ij}, \hat\rho\right\}\right].
\end{equation}
The parameter $\omega \in [0,1]$ allow us to switch between the coherent and dissipative phenomena.

The classical random walk behavior emerges considering the evolution of the diagonal element of the density matrix for just $\mathcal{L}^{cl}$, that is in the limit $\omega = 1$
\begin{equation}
    \begin{split}
        \mathcal{L}^{cl}\ket{a}\bra{a} &=  \sum_{ij}\gamma_{ij}\left[\hat J_{ij} \ket{a}\bra{a}\hat J_{ij}^\dagger -\frac{1}{2} \left\{ \hat J_{ij}^\dagger \hat J_{ij}, \ket{a}\bra{a}\right\}\right]\\
        %&= \sum_{ij}\left[\sqrt{A_{ij}}\ket{i}\braket{j}{a}\braket{a}{j}\bra{i}\sqrt{A_{ij}} -\frac{1}{2}\ket{i}\sqrt{A_{ia}}\braket{j}{j}\sqrt{A_{ia}}\braket{i}{a}\bra{a}\right.\\ 
        %&\quad \left. - \frac{1}{2} \ket{a}\braket{a}{i}\sqrt{A_{ia}}\braket{j}{j}\sqrt{A_{ia}}\bra{i}\right]\\
        &= \sum_{i}\left[A_{ia} \ket{i}\bra{i} -A_{ia}\ket{a}\bra{a}\right]\\
        &=\sum_{i} A_{ai} \ket{i}\bra{i} - d_a \ket{a}\bra{a} = -\sum_i L_{ai} \ket{i}\bra{i}.
    \end{split}
\end{equation} 
where $d_i$ is the degree of the node $i$.
We have recovered the dynamics of the classical random walk over a network.

Considering now the off-diagonal terms, they  evolve as 
\begin{equation}
    \begin{split}
        \mathcal{L}^{cl}\ket{a}\bra{b} &= \sum_{ij}\gamma_{ij}\left[\hat J_{ij} \ket{a}\bra{b}\hat J_{ij}^\dagger -\frac{1}{2} \left\{ \hat J_{ij}^\dagger \hat J_{ij}, \ket{a}\bra{b}\right\}\right]\\
        &= \sum_{ij}\left[\sqrt{A_{ij}}\ket{i}\braket{j}{a}\braket{b}{j}\bra{i}\sqrt{A_{ij}} -\frac{1}{2}\ket{i}\sqrt{A_{ia}}\braket{j}{j}\sqrt{A_{ia}}\braket{i}{a}\bra{b}\right.\\ 
        &\quad\left. - \frac{1}{2} \ket{a}\braket{b}{i}\sqrt{A_{ia}}\braket{j}{j}\sqrt{A_{ia}}\bra{i}\right]\\
        &= \sum_{j}\left[-\frac{1}{2} A_{ja}\ket{a}\bra{b} - \frac{1}{2} A_{ja}\ket{a}\bra{b}\right]\\
        &= - \frac{1}{2}(d_a + d_b)\ket{a}\bra{b}.
    \end{split} 
\end{equation}

$\mathcal{L}^{cl}$ do not mixed the diagonal term with the off-diagonal ones. Thus, the superoperator is divide in two block: one for the diagonal element, the other for the off-diagonal ones.
The superoperator $\mathcal{L}^{cl}$ has a diagonal form with spectrum $\sigma^{cl} = -(\lambda_1,...,\lambda_N,\frac{1}{2}(d_1 + d_2),...,\frac{1}{2}(d_N + d_{n-1}))$, where $\lambda_i$ are the eigenvalue of the Laplacian matrix \cite{Bruderer_Plenio}.


\begin{comment}

Instead if we consider the jump operator as $\hat J_{ij} =\sqrt{L_{ij}}\ket{i\bra{j}}$, the evolution of the diagonal and off-diagonals term become 
\begin{equation}
    \begin{split}
        \mathcal{L}^{cl}\ket{a}\bra{a} &=  \sum_{ij}\left[\hat J_{ij} \ket{a}\bra{a}\hat J_{ij}^\dagger -\frac{1}{2} \left\{ \hat J_{ij}^\dagger \hat J_{ij}, \ket{a}\bra{a}\right\}\right]\\
        &= \sum_{ij}\left[\sqrt{L_{ij}}\ket{i}\braket{j}{a}\braket{a}{j}\bra{i}\sqrt{L_{ij}} -\frac{1}{2}\ket{i}\sqrt{L_{ia}}\braket{j}{j}\sqrt{L_{ia}}\braket{i}{a}\bra{a}\right.\\
        &\quad \left. - \frac{1}{2} \ket{a}\braket{a}{i}\sqrt{L_{ia} v}\braket{j}{j}\sqrt{L_{ia}}\bra{i}\right]\\
        &= \sum_{i}\left[L_{ia} \ket{i}\bra{i} -L_{ia}\ket{a}\bra{a}\right]\\
        &=\sum_{i} L_{ai} \ket{i}\bra{i}.
    \end{split}
\end{equation}
Since $\sum_i L_{ia} = 0$. For the same motivation, the off-diagonal terms are killed by $\mathcal{L}^{cl}$.
Both the two possible jump operator recover the classical random walk over a network behavior, but they create  different evolution for the density matrix.
\end{comment}




\section{Time evolution}
Following the resolution proposed by Fujii \cite{fujii2012} for a quantum harmonic oscillator, we can solve the Lindblad master equation. 
First of all, we vectorize the density matrix: let introduce an Hilbert space with dimension $N^2$ such that a vector is $\sket{\rho} = (\rho_{00},\rho_{01},...,\rho_{NN-1},\rho_{NN})^T$ and the scalar product is $\sbraket{\phi}{\rho}= \Tr[\hat\phi^\dagger\hat\rho]$. This is called Fock-Liouville space \cite{Manzano}.

The follow operation can be vectorize as
\begin{equation}
    \hat A \hat\rho \hat B \rightarrow (\hat A\otimes\hat B)\sket{\rho} \qquad \hat A \hat\rho +\hat\rho\hat B \rightarrow \left(\hat A\otimes\mathbb{I} +\mathbb{I}\otimes\hat B\right)\sket{\rho},
\end{equation}
where $\mathbb{I}$ is the identity matrix, $\hat A$ and $\hat B$ are two generic operator. The glyph $\otimes$ represents the tensorial product that generates a $N^2\times N^2$ matrix defined as
\begin{equation}
    \hat A\otimes\hat B = \begin{pmatrix}
        A_{11} \hat B & \cdots & A_{1N}\hat B\\
        \vdots & \ddots &\vdots\\
        A_{n1}\hat B& \cdots & A_{NN}\hat B\\
    \end{pmatrix}.
\end{equation}
More detail are shown in the Appendix \ref{A_vectorial_density_matrix}.

In this space the Lindblad equation \eqref{stochastic_lindblad_master} becomes
\begin{equation}\label{vectorize_master_equation}
        \frac{d}{dt}\sket{\rho(t)} = \tilde{\mathcal{L}}\sket{\rho(t)},
    \end{equation}
    where $\tilde{\mathcal{L}}$ is the operator 
    \begin{equation}
        \tilde{\mathcal{L}}=-i(1-\omega)\left[\hat H\otimes\mathbb{I}-\mathbb{I}\otimes\hat H\right]+\omega\sum_{ij}\gamma_{ij}\left[\hat J_{ij}\otimes\hat J^\dagger_{ij} + \hat J_{ij}^\dagger \hat J_{ij}\otimes\mathbb{I} +\mathbb{I}\otimes\hat J_{ij}^\dagger \hat J_{ij}\right].
\end{equation}
The solution equation \eqref{vectorize_master_equation} can be written as
\begin{equation}
    \sket{\rho(t)} = \hat U(t,\omega)\sket{\rho(0)},
\end{equation}
where $U(t,\omega)$ is the evolution operator \cite{Domino}
\begin{equation}
    \begin{split}
            \hat U(t,\omega) = \exp&\left\{-i\left(1-\omega\right)t\left(\hat H\otimes\mathbb{I}-\mathbb{I}\otimes\hat H\right)\right.\\
            &+\left.\omega t\sum_{ij}\gamma_{ij}\left[ \hat J_{ij}\otimes\hat J^\dagger_{ij}-\frac{1}{2}\hat J_{ij}\hat J^\dagger_{ij}\otimes\mathbb{I}-\frac{1}{2}\mathbb{I}\otimes\hat J_{ij}\hat J^\dagger_{ij}\right]\right\}
        \end{split}
\end{equation}
The evolution operator is not unitary.

\section{Entropy production}
We can compute also the Von Neumann entropy for the network. 
Looking at its derivative $\dot S(\hat\rho(t))$, we can consider just the dissipative part, since the unitary one does not change entropy. Knowing that the dynamics is trace preserving, namely $\Tr\left[\frac{d\hat\rho}{dt}\right] = 0$, we obtain
\begin{equation}
        \dot S(\hat\rho) = -\Tr\left[\frac{d\hat\rho}{dt}\ln\hat\rho\right] = -\Tr\left[\sum_{ij}\gamma_{ij}\left[\hat J_{ij} \hat\rho\hat J_{ij}^\dagger -\frac{1}{2} \left\{ \hat J_{ij}^\dagger \hat J_{ij}, \hat\rho\right\}\right]\ln\hat\rho\right].\\
\end{equation}
We expand the commutator
\begin{equation}
    \dot S(\hat\rho) =-\sum_{ij}\gamma_{ij}\Tr\left[\hat J_{ij} \hat\rho\hat J_{ij}^\dagger\ln\hat\rho -\frac{1}{2}\hat J_{ij}^\dagger \hat J_{ij}\hat\rho \ln\hat\rho-\frac{1}{2} \hat\rho\hat J_{ij}^\dagger \hat J_{ij}\ln\hat\rho\right]\\
\end{equation}
        
Since $\ln\hat\rho$ and $\hat\rho$ commute, the second and third terms can be summed. Thus, substituting $\gamma_{ij} = A_{ij}$ we arrive to
\begin{equation}
        \dot S(\hat\rho) = -\sum_{ij}A_{ij}\Tr\left[\hat J_{ij} \hat\rho\hat J_{ij}^\dagger\ln\hat\rho -\hat J_{ij}^\dagger \hat J_{ij}\hat\rho\ln\hat\rho\right].
\end{equation}

Reordering the terms we obtain
\begin{equation}\label{QSW_entropy_production}
    \dot S(\hat\rho) =  -\sum_{ij}A_{ij}\left(\Tr\left[\hat J_{ij} \hat\rho\hat J_{ij}^\dagger\ln\hat\rho\right]-\Tr\left[\hat J_{ij}^\dagger \hat J_{ij}\hat\rho\ln\hat\rho\right]\right)
\end{equation}

The equation \eqref{QSW_entropy_production} is always positive, the proof is in the appendix \ref{A_entropy_increasing}. 

The last result tell us that the dynamics increases the entropy and, thus, change a pure density matrix in a mixed one.
As a consequent, the the stationary distribution must have maximum entropy.

\section{Stationary distribution}

Following the prove of the stationary distribution in chapter \ref{C_Lindblad}, we can find the stationary density matrix $\hat\rho^*$ for the quantum stochastic walk.
Let consider the case $\omega = 1$, thus, only the dissipative term survive.
The dynamics decouples the diagonal and off-diagonal terms. The first follow the evolution 
\begin{equation}
    \frac{d}{dt}\rho_{ii} = \sum_j\left[W(i|j)\rho_{jj}(t) - W(j|i)\rho_{ii}(t)\right],
\end{equation}
with transition rate
\begin{equation}
    W(i|j) = \sum_{lm} \gamma_{lm}\bra{i}J_{lm}\ket{j}\bra{i}J_{lm}^\dagger\ket{j} = W(j|i).
\end{equation}

Since the stationary distribution must satisfy the detail balance, namely
\begin{equation}
    W(i|j)\rho_{jj}(t) = W(j|i)\rho_{ii}.
\end{equation}
Thus, the diagonal entries must be equal. 
Instead, considering the vector $\sket{\rho}$, the block of corresponding to the off diagonal part of $\mathcal{L}$ is already diagonal with eigenvalue $\frac{1}{2}(d_i + d_j)>0$. Thus, off diagonal terms must be equal to zero.
The stationary density matrix will be
\begin{equation}
    \hat\rho^* = \frac{1}{N}\begin{pmatrix}
        1&&\\
        &\ddots&\\
        &&1\\
    \end{pmatrix}.
\end{equation}

\begin{comment}
In chapter  we prove that the master equation \eqref{stochastic_lindblad_master} allows a stationary distribution in the form
\begin{equation}
    \rho^* = \frac{1}{Z} e^{-\beta H}
\end{equation}  
with $\beta$ the inverse of the temperature of the thermal bath.
    
Let consider the temperature of bath  decreasing very slowly such that the system is always in thermal equilibrium with the bath. When the temperature tends to zero ($\beta \rightarrow \infty$), the system will reach the final distribution $\hat \rho^* = \frac{1}{N}\;diag(1,...,1)$, where the system has collapsed into the state with zero eigenvalue of the superoperator $\mathcal{L}^{cl}$. The diagonal element give us the probability to found the particle in the respective node and it is the same result we have obtained for the classical random walk.
\end{comment}

\section{Kirchhoff}
Let take for this chapter the standard Lindblad equation \eqref{Lindblad_equation_(stoc_chapter)} with Hamiltonian $\hat H = \hat L$
\begin{equation}
    \frac{d}{dt}\hat\rho =\mathcal{L}\left[\hat\rho\right] = -i\left[\hat H,\hat\rho\right] + \sum_{ij} \gamma_{ij} \left(\hat J_{ij} \hat\rho \hat J^\dagger_{ij} - \frac{1}{2}\left\{ \hat J^\dagger_{ij}\hat J_{ij}, \hat\rho\right\} \right),
\end{equation}
the jump operator $\hat J_{ij} = \ket{i}\bra{k}$ indicates the jumps between two node and the coefficient $\gamma_{ij}$ is not already defined.
We can change the basis to $\{\ket{\lambda}\}$ such that $\hat H = \sum_\lambda \epsilon_\lambda\ket{\lambda}\bra{\lambda}$ is diagonal. The previous equation becomes
\begin{equation}
    \frac{d}{dt}\hat\rho = -i\left[\hat H,\hat\rho\right] + \sum_{\lambda\mu} \gamma_{\lambda\mu} \left(\hat J_{\lambda\mu} \hat\rho \hat J^\dagger_{\lambda\mu} - \frac{1}{2}\left\{ \hat J^\dagger_{\lambda\mu}\hat J_{\lambda\mu}, \hat\rho\right\} \right),
\end{equation}
with $\hat J_{\lambda\mu} = \ket{\lambda}\bra{\mu}$ and $\gamma_{\lambda\mu}$ is still not defined.
We assume that the dynamics will tend to a stationary distribution in the form of
\begin{equation}\label{pretended_stationary_distribution}
    \hat \rho^* = \frac{e^{-\beta\hat H}}{Z}.
\end{equation}
Thus, the master equation for the stationary distribution is
\begin{equation}\label{cancel_master_equation}
    0 = -i\left[\hat H, \frac{e^{-\beta\hat H}}{Z}\right] + \sum_{\lambda\mu} \gamma_{\lambda\mu} \left(\hat J_{\lambda\mu}  \frac{e^{-\beta\hat H}}{Z} \hat J^\dagger_{\lambda\mu} - \frac{1}{2}\left\{ \hat J^\dagger_{\lambda\mu}\hat J_{\lambda\mu},  \frac{e^{-\beta\hat H}}{Z}\right\} \right).
\end{equation}
The first term in the r.h.s. vanish because the commutator is zero. 
The first term of the sum can be written as 
\begin{equation}
        \sum_{\lambda\mu} \gamma_{\lambda\mu} \ket{\lambda}\bra{\mu}\frac{e^{-\beta\hat H}}{Z} \ket{\mu}\bra{\lambda} = 
        \sum_{\lambda\mu} \gamma_{\lambda\mu} \frac{e^{-\beta \epsilon_\mu}}{Z} \ket{\lambda}\bra{\lambda}.
\end{equation}

While the second becomes
\begin{equation}
    \sum_{\lambda\mu} \gamma_{\lambda\mu}\left[\frac{1}{2}\ket{\mu}\braket{\lambda}{\lambda}\bra{\mu}\frac{e^{-\beta\hat H}}{Z} +\frac{1}{2}\frac{e^{-\beta\hat H}}{Z}\ket{\mu}\braket{\lambda}{\lambda}\bra{\mu}\right]= \sum_{\lambda\mu} \gamma_{\lambda\mu}\left[\frac{e^{-\beta \epsilon_\mu}}{Z}\ket{\mu}\bra{\mu}\right].
\end{equation}

The master equation \eqref{cancel_master_equation} reduces to 
\begin{equation}
    \sum_{\lambda\mu}\left[\gamma_{\lambda\mu}\frac{e^{-\beta \epsilon_\mu}}{Z} - \gamma_{\mu\lambda}\frac{e^{-\beta \epsilon_\lambda}}{Z}\right]\ket{\lambda}\bra{\lambda}.
\end{equation}
It is the Kirchhoff's current law that says that the sum of all the currents must vanish. The system should satisfy this request in order to have the canonical distribution.
The same distribution can be obtain in position space. As a matter of fact, the master equation \eqref{stochastic_lindblad_master} for the distribution \eqref{pretended_stationary_distribution}. Let assume that it is the stationary distribution, thus
\begin{equation}
    0 = \sum_{\lambda\mu} \gamma_{ij} \left(\hat J_{ij}  \frac{e^{-\beta\hat H}}{Z} \hat J^\dagger_{ij} - \frac{1}{2}\left\{ \hat J^\dagger_{ij}\hat J_{ij},  \frac{e^{-\beta\hat H}}{Z}\right\} \right).
\end{equation}

the first term can be written as
\begin{equation}
    \sum_{ij} \gamma_{ij} \ket{i}\bra{j}  \frac{e^{-\beta\hat H}}{Z} \ket{j}\bra{i} = 
    \sum_{ji} \gamma_{ij} \frac{e^{-\beta H_{jj}}}{Z} \ket{i}\bra{i}
\end{equation}

Whereas, the second term becomes
\begin{equation}
    \sum_{ij}\frac{\gamma_{ij}}{2}\ket{j}\bra{j} \frac{e^{-\beta\hat H}}{Z} + \frac{\gamma_{ij}}{2}\frac{e^{-\beta\hat H}}{Z}\ket{j}\bra{j}=
    \sum_{ijk} \frac{1}{2}\left(\gamma_{ij}\frac{e^{-\beta H_{jk}}}{Z} + \gamma_{ik}\frac{e^{-\beta H_{jk}}}{Z}\right)\ket{j}\bra{k}
\end{equation}

We can combine together all the terms obtaining
\begin{equation}
    \begin{split}
        \sum_{ji} \left(\gamma_{ji}\frac{e^{-\beta H_{ii}}}{Z} - \gamma_{ij}\frac{e^{-\beta H_{jj}}}{Z}\right)\ket{j}\bra{j} = 0\\
        \sum_{jk} \frac{1}{2}\left(\sum_i\gamma_{ij} + \sum_i\gamma_{ik}   \right)\frac{e^{-\beta H_{jk}}}{Z}\ket{j}\bra{k} = 0\\
    \end{split}
\end{equation}
The first equation is again the Kirchhoff's current law, while the second is a condition over the parameter $\gamma$.

If the previous condition are satisfy the stationary distribution of the system is the canonical one. In our case it becomes
\begin{equation}
    \hat\rho^* = \frac{e^{-\beta\hat L}}{Z} \qquad Z = \Tr[e^{-\beta\hat L}]  
\end{equation}
that is the density matrix \eqref{density_matrix} introduced by De Domenico to identify networks.

\newpage
\section{Symmetry breaking}

Until now, we have considered the network holding the detail balance condition and, therefore, be mapped in a symmetric matrix; 
but the majority of the networks do not satisfy this condition. To deal with them, we modify slightly the Lindblad master equation \eqref{stochastic_lindblad_master}. 
As a matter of fact, in the chapter \ref{C_Lindblad} we have analyzed also the case where the interaction with the environment is not symmetric \eqref{environment_coefficients}. Thus, taking the dissipative part of the equation \eqref{C_rotating_wave} in the Schrödinger picture with the coefficients $\Gamma_{ij} = L_{ij}$ and the jump operators $J_{ij} = \ket{i}\bra{j}$ we obtain 
\begin{equation}
    \frac{d\hat\rho(t)}{dt} = \sum_{ij}\Gamma_{ij}\left[\hat J_j\hat\rho(t),\hat J_i^\dagger\right]+\Gamma_{ji}^\dagger\left[\hat J_j,\hat\rho(t)\hat J_i^\dagger\right].
\end{equation}
Isolating the symmetric and antisymmetric part of the Laplacian, respectively $\gamma_{ij} = \left(L_{ij} + L_{ji}\right)$ and $\pi_{ij} =  \frac{-i}{2}\left(L_{ij}-L_{ji}\right)$ such that $\Gamma_{ij}(\omega) =\frac{1}{2}\gamma_{ij}(\omega)+i\pi_{ij}(\omega)$, we arrive to the equation
\begin{equation}
    \frac{d\hat\rho(t)}{dt} = \sum_{ij}\gamma_{ij}\hat J_j\hat\rho(t)\hat J_i^\dagger -\frac{\gamma_{ij}}{2}\left\{\hat J_i^\dagger\hat J_j,\hat\rho(t)\right\} + i\pi_{ij}\left[\hat J_i^\dagger\hat J_j,\hat\rho(t)\right],
\end{equation}
where $[\cdot,\cdot]$ and $\{\cdot,\cdot\}$ are respectively the commutator and anticommutator.

Let define a new Hamiltonian $\hat H_{A} = \sum_{ij}\pi_{ij}\hat J_i^\dagger\hat J_j$ that encodes the dynamics of the not symmetric part. 
It give origin to a coherent dynamics that follow the Von Neumann equation. As a matter of fact the total dynamics can be written as
\begin{equation}\label{antisymmetric_master_equation}
    \frac{d\hat\rho(t)}{dt} = i\left[\hat H_{A},\hat\rho(t)\right] + \sum_{ij}\gamma_{ij}\hat J_j\hat\rho(t)\hat J_i^\dagger -\frac{\gamma_{ij}}{2}\left\{\hat J_i^\dagger\hat J_j,\hat\rho(t)\right\}.
\end{equation}

The dynamics \eqref{antisymmetric_master_equation} does not converge no more to a stationary state due to the Von Neumann part.
We can generalize as in the \eqref{stochastic_lindblad_master} 

\begin{equation}
    \frac{d}{dt}\hat \rho = -i\left[(1-\omega)\hat H + \omega\hat H_{A},\hat\rho\right] + \omega\sum_{ij}\gamma_{ij}\left[\hat J_{ij} \hat\rho\hat J_{ij}^\dagger -\frac{1}{2} \left\{ \hat J_{ij}^\dagger \hat J_{ij}, \hat\rho\right\}\right].
\end{equation}
where $\hat H$ is the hermitian part of the Laplacian operator.


