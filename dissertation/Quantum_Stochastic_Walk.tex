\section{Quantum Stochastic Random Walk}\label{C_Quantum Stochastic Walk}

One of the firsts approaches to a Open Quantum Walk on network was proposed by Whitfield, Rodr\'iguez-Rosario and Aspuru-Guzik. \cite{QSW}. They defined a quantum walk on network in contact with a thermal bath, the dynamics is describe by a Lindblad master equation with the jump operators proportional to the Adjacency matrix $A_{ij}$ of the network. The thermal bath act as noise in the dynamics which does not follow strictly the Von Neumann equation \eqref{Von Neumann equation}. The system dissipation reminds the classic random walk.

Let we consider a quantum walk on a network $G(N,M)$, the system in contact with a thermal bath that randomly interact with modifying the dynamic of the quantum particle. we can describe the its evolution with the Lindblad master equation \eqref{Lindblad_master_equation}: the Laplacian operator $\hat L$ is Hamiltonian $\hat H$ and the jump operator $\{\hat J_k\}_{k<M}$ are the jumps from two node connected by a link. For convenience, we will call the jump operator with two indeces referring to the starting node and the ending node of the jump;therefore, the jumps operator are $\hat J_{ij} = \ket{i}\bra{j}$. The dumping rates $\gamma_i$ are the $\gamma_{ij} =A_{ij}$.
We obtain the master equation
\begin{equation}\label{stochastic_lindblad_master}
    \frac{d}{dt}\hat \rho = -i\left[\hat H,\hat\rho\right] + \sum_{ij}\gamma_{ij}\left[\hat J_{ij} \hat\rho\hat J_{ij}^\dagger -\frac{1}{2} \left\{ \hat J_{ij}^\dagger \hat J_{ij}, \hat\rho\right\}\right],
\end{equation}
where $[\cdot,\cdot]$ and $\{\cdot,\cdot\}$ are respectively the commutator and the anticommutator.
The equation \eqref{stochastic_lindblad_master} is composed by two distinct terms: the first term, also called coherent dynamics,
$\mathcal{L}^{qm}\left[\hat\rho(t)\right] = -i\left[H,\hat\rho\right]$ is equal to the quantum walk dynamics; instead, the second $\mathcal{L}^{cl}\left[\hat\rho(t)\right] = \sum_i \gamma_i \left(\hat J_i \hat\rho \hat J^\dagger_i - \frac{1}{2}\left\{ \hat J^\dagger_i\hat J_i, \hat\rho\right\} \right)$, called decoherent dynamics, encodes the dissipation. 
When $\gamma_{ij} = 0$ we recover the Von Neumann equation for the quantum walk \eqref{Von Neumann equation}. 
$\mathcal{L}\left[\hat\rho\right]$ is a superoperator that act in the space of the density matrix.

In the Fock-Liouville space, the quantum system evolves following the equation
\begin{equation}
    \sket{\rho(t)}  = U(t) \sket{\rho(0)}
\end{equation} 
where the evolution operator is defined as \cite{Domino}
\begin{equation}
    \begin{split}
        \hat U(t) = \exp&\left\{-it\left(\hat H\otimes\mathbb{I}-\mathbb{I}\otimes\hat H\right)\right.\\
        &+\left. t\sum_{ij}\gamma_{ij}\left[ \hat J_{ij}\otimes\hat J^\dagger_{ij}-\frac{1}{2}\hat J_{ij}\hat J^\dagger_{ij}\otimes\mathbb{I}-\frac{1}{2}\mathbb{I}\otimes\hat J_{ij}\hat J^\dagger_{ij}\right]\right\}.
    \end{split}
\end{equation}



The master equation \eqref{stochastic_lindblad_master} contains both the quantum and classical aspect of a diffusion over a network. As a matter of fact,the classical random walk behavior emerges considering the evolution of the diagonal element of the density matrix under just the dissipative part. Let $\rho = \ket{k}\bra{k}$ be the density matrix of a system in node $k$, it evolves following
\begin{equation}
    \begin{split}
        \mathcal{L}^{cl}\ket{k}\bra{k} &= \sum_{ij}\gamma_{ij}\left[\hat J_{ij} \ket{k}\bra{k}\hat J_{ij}^\dagger -\frac{1}{2} \left\{ \hat J_{ij}^\dagger \hat J_{ij}, \ket{k}\bra{k}\right\}\right]\\
        %&= \sum_{ij}\left[\sqrt{A_{ij}}\ket{i}\braket{j}{a}\braket{a}{j}\bra{i}\sqrt{A_{ij}} -\frac{1}{2}\ket{i}\sqrt{A_{ia}}\braket{j}{j}\sqrt{A_{ia}}\braket{i}{a}\bra{a}\right.\\ 
        %&\quad \left. - \frac{1}{2} \ket{a}\braket{a}{i}\sqrt{A_{ia}}\braket{j}{j}\sqrt{A_{ia}}\bra{i}\right]\\
        &= \sum_{i}\left[A_{ik} \ket{i}\bra{i} -A_{ik}\ket{k}\bra{k}\right]\\
        &=\sum_{i} A_{ki} \ket{i}\bra{i} - d_a \ket{k}\bra{k} = -\sum_i L_{ki} \ket{i}\bra{i}.
    \end{split}
\end{equation} 
where $d_i$ is the degree of the node $i$.
We have recovered the dynamics of the classical random walk over a network.

Considering the off-diagonal terms, they evolve as 
\begin{equation}
    \begin{split}
        \mathcal{L}^{cl}\ket{k}\bra{kl} &= \sum_{ij}\gamma_{ij}\left[\hat J_{ij} \ket{k}\bra{l}\hat J_{ij}^\dagger -\frac{1}{2} \left\{ \hat J_{ij}^\dagger \hat J_{ij}, \ket{k}\bra{l}\right\}\right]\\
        %&= \sum_{ij}\left[\sqrt{A_{ij}}\ket{i}\braket{j}{k}\braket{l}{j}\bra{i}\sqrt{A_{ij}} -\frac{1}{2}\ket{i}\sqrt{A_{ik}}\braket{j}{j}\sqrt{A_{ik}}\braket{i}{k}\bra{l}\right.\\ 
        %&\quad\left. - \frac{1}{2} \ket{k}\braket{l}{i}\sqrt{A_{ik}}\braket{j}{j}\sqrt{A_{ik}}\bra{i}\right]\\
        &= \sum_{j}\left[-\frac{1}{2} A_{jk}\ket{k}\bra{l} - \frac{1}{2} A_{jl}\ket{k}\bra{l}\right]\\
        &= - \frac{1}{2}(d_k + d_l)\ket{a}\bra{b}.
    \end{split} 
\end{equation}

$\mathcal{L}^{cl}$ do not mixed the diagonal term with the off-diagonal ones. Thus, the superoperator is divide in two block: one for the diagonal element, the other for the off-diagonal ones.
The superoperator $\mathcal{L}^{cl}$ has a diagonal form with spectrum $\sigma^{cl} = -(\lambda_1,...,\lambda_N,\frac{1}{2}(d_1 + d_2),...,\frac{1}{2}(d_N + d_{N-1}))$, where $\lambda_i$ are the eigenvalue of the Laplacian matrix \cite{Bruderer_Plenio}.
If the network satisfies the detailed balance condition \eqref{detail_condition}, the Laplacian matrix has a zero eigenvalue and, therefore, also the superoperator $\mathcal{L}^{cl}$ has a zero eigenvalue and a stationary distribution.


\subsection{Stationary distribution}

Following the prove of the stationary distribution in chapter \ref{C_Lindblad}, we can find the stationary density matrix $\hat\rho^*$ for the quantum stochastic walk.
Let consider the only dissipation dynamics.
The dynamics decouples the diagonal and off-diagonal terms. The first follow the evolution 
\begin{equation}
    \frac{d}{dt}\rho_{ii} = \sum_j\left[W(i|j)\rho_{jj}(t) - W(j|i)\rho_{ii}(t)\right],
\end{equation}
with transition rate
\begin{equation}
    W(i|j) = W(j|i) = A_{ij}.
\end{equation}

Since the stationary distribution must satisfy the detail balance, namely
\begin{equation}
    W(i|j)\rho_{jj}(t) = W(j|i)\rho_{ii}.
\end{equation}
Thus, the diagonal entries must be equal. 
Instead, considering the vector $\sket{\rho}$, the block of corresponding to the off diagonal part of $\mathcal{L}$ is already diagonal with eigenvalue $\frac{1}{2}(d_i + d_j)>0$. Thus, off diagonal terms must be equal to zero.

The stationary density matrix will be
\begin{equation}
    \hat\rho^* = \frac{1}{N}\begin{pmatrix}
        1&&0\\
        &\ddots&\\
        0&&1\\
    \end{pmatrix}.
\end{equation}

As anticipated, the stationary distribution has maximal Von Neumann entropy 
\begin{equation}
    S\left(\hat\rho^*\right) = \ln N.
\end{equation}
