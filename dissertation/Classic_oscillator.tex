\subsection{Network of Classical Harmonic Oscillators}

Let consider a hamiltonian system with Hamiltonian \eqref{H_L} in contact with a thermal bath using the Langevin equation
\begin{equation}
    \begin{aligned}
        &\dot q_i = p_i; \\
        %&\dot p_i  = -H_{ij}q_j - \gamma \sum_j \left(\delta_{ij} - 1_{ij}\right)p_j + \sqrt{2T\gamma}\xi_i(t),
        &\dot p_i  = -H'_{ij}q_j - \gamma p_i + \sqrt{2T\gamma}\xi_i(t),
    \end{aligned}
\end{equation}
where $\gamma$ is the friction coefficient, $T$ is the temperature (with Boltzmann constant $K_B =1$). The term $\xi_i(t)$ represents white noise defined as
\begin{equation}\label{white_noise}
    \langle\xi_i(t)\rangle = 0 \qquad \langle\xi_i^2(t)\rangle = 1 
\end{equation}
The white noises must hold the condition $\sum_i \xi_i = 0$, that leaves invariant the motion of  system's center of mass but $\xi_i(t)$ are no more independent.
As a matter of fact, the total momentum $P = \sum_i p_i$ is an integral of motion
\begin{equation}
    %\frac{d}{dt} \sum_i \dot p_i = - \gamma \sum_{ij}\left(\delta_{ij} - 1_{ij}\right)p_j+ \cancel{\sqrt{2T\gamma}\sum_i\xi_i(t)} = 0.
    \frac{d}{dt} \sum_i \dot p_i = - \gamma \sum_{i}p_i + \cancel{\sqrt{2T\gamma}\sum_i\xi_i(t)} = 0.
\end{equation}

The condition over the white noises $\sum_i \xi_i = 0$ breaks the independence between them and it adds correlation.
We can rewriting the noise using i.i.d. white noise $w_i(t)$ as
\begin{equation}
    \xi_i(t) = w_i (t) + \frac{1}{N} \sum_k w_k(t).
\end{equation}
The covariance matrix of $\xi_i(t)$ yields
\begin{equation}
    \left<\xi_i(t)\xi_j(s)\right> = \left[\delta_{ij} - 1_{ij}\right]\delta(t-s)
\end{equation}

The distribution $\rho(q,p,t)$ is a Gaussian and satisfies the Fokker-Planck equation \cite{Fokker}
\begin{equation}
    %\frac{\partial\rho}{\partial t} = -\sum_i p_i\frac{\partial \rho}{\partial q_i} + \sum_{ij} H_{ij}q_j\frac{\partial \rho}{\partial p_i} + \gamma\sum_{ij}\left(\delta_{ij}-1_{ij}\right)\left[\frac{\partial}{\partial p_i}p_j\rho + T \frac{\partial^2\rho}{\partial p_i \partial p_j}\right].
    \frac{\partial\rho}{\partial t} = -\sum_i p_i\frac{\partial \rho}{\partial q_i} + \sum_{ij} H'_{ij}q_j\frac{\partial \rho}{\partial p_i} + \gamma\sum_{i}\left[\frac{\partial}{\partial p_i}p_i\rho + T \frac{\partial^2\rho}{\partial p_i^2}\right].
\end{equation}
The dynamics converges to a stationary distribution, with the time scale depending on the eigenvalues of the Laplacian matrix.  
The solution at equilibrium is give by
\begin{equation}
    \rho(q,p) = Z(\beta)^{-1} \exp\left[ -\beta \left( \sum_j {p_j^2} + \sum_{ij} \frac{1}{2}q_iH'_{ij}q_j\right)\right],
\end{equation}
where $\beta = \frac{1}{T}$ and $Z(\beta)$ is the partition function defined as
\begin{equation}
    Z(\beta) = \int \prod_i dp_i dq_i \; \exp\left[ -\beta \left( \sum_j {p_j^2} + \sum_{ij} q_iH'_{ij}q_j\right)\right].
\end{equation}
The marginal distribution over the coordinates is a Boltzmann distribution 
\begin{equation}
    \rho(q) = Z(\beta)^{-1} e^{-\beta \left(\sum_{ij} q_iH_{ij}q_j\right)}.
\end{equation}
If $H'$ is symmetric, namely the detailed balance condition \eqref{detail_condition} holds, we can diagonalize the marginal distribution obtaining the motion of independent oscillators in the same thermal bath.
Therefore, changing the basis from $q_i$ to $Q_\lambda$ eigenvectors with $\lambda$ eigenvalue of the Hamiltonian $H'$, the marginal distribution becomes
\begin{equation}\label{marginal_probability}
    \rho(q) = Z(\beta)^{-1} e^{-\beta \left(\sum_{\lambda \neq 0} Q_\lambda \lambda Q_\lambda\right)},
\end{equation}
with the partition function 
\begin{equation}
    Z(\beta) = \int \prod_{\lambda\neq 0} dQ_\lambda e^{-\beta \left(\sum_{\lambda \neq 0} \lambda Q_\lambda^2\right)}.
\end{equation}

The thermal distribution does not involve the center of mass motion since the thermal bath does not interact with it. Thus, we can project the system into an invariant subspace orthogonal to the stationary distribution. The oscillator modes $Q_\lambda$ remain the same of the unperturbed case. 
This is a consequence of the condition $\sum_i \xi_i = 0$.
%Moreover, this is also connected to the conservation of the stationary distribution of the master equation \eqref{stationary_distribution}. 
The distribution has mean $\left<Q_\lambda\right>= 0$ and the covariance matrix is diagonal with entries $\left<Q^2_\lambda\right>= \frac{1}{\beta \lambda}$.

The variance can be expressed as
\begin{equation}\label{classic_correlation}
    \mathrm{Cov}(Q) = \frac{1}{\beta}H^{-1},
\end{equation}
where ${H}^{-1} = \sum_{i=2}^N \frac{1}{\lambda_i}v_i^Tv_i$ is the Moore-Penrose generalized inverse of the Hamiltonian. Here, $\lambda$ are the eigenvalues ordered from the smallest to the biggest such that $\lambda_1 < \lambda_2 < ... < \lambda_N$, and $v_i$ are the respective eigenvectors of the Hamiltonian matrix \cite{Generalized_inverse_Laplacian}.

Substituting the equation \eqref{strange_potential} into the correlation \eqref{classic_correlation} we obtain
\begin{equation}
    \mathrm{Cov}(Q) = \frac{1}{\beta}\left(K'\mathbb{I} - L_{ij}\right)^{-1} = \frac{1}{\beta K'}\left(\mathbb{I} - K'^{-1}L_{ij}\right)^{-1}
\end{equation}
where $\mathbb{I}$ is the identity matrix.
This is proportional to the Communicability matrix $G^R(L)$ \eqref{Estrada indeces} when $K' = 1/\alpha$.
%The parameter $K'$ suppresses the motion along the eigenvectors with high eigenvalues. In the limit $K'\rightarrow \infty$, all the eigenvectors are suppressed except for the zero eigenvector.

%When $T\rightarrow 0$ the spread of information drops; and when $T\rightarrow +\infty$ it becomes instantaneous.