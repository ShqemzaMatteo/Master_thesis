\section{Estrada's Communicability matrix}

Most of the study on complex networks focuses on the spread of information following the shorter path, namely the shortest sequence of links that connects two different nodes. 
However, this is not the only way the information can flow, there are plenty of other more long route that are also available, and this vision ignores completely the complexity of the network.
To overcome that we introduce the communicability matrix, defined to consider also these possible path to go beyond the shortest one \cite{Estrada_2012}. It consider the influence over all the path that cross the choose node, weighted by their length.
%This concept is similar to the correlation in physics: it indicates how a node is changes in response to a perturbation in another node. 

Let $G=(V,E)$ be an undirected graph composed of $N$ nodes and $E$ links and let $A$ be the adjacency matrix of the graph.
We can define the communicability matrix as
\begin{equation}
    G(A) = \sum_{k=1}^{\infty}c_k A^k
\end{equation}
and the communicability from node $i$ to node $j$ is given by $G_{ij}$. The power of the adjacency matrix $(A^k)_{ij}$ give us the number of path of length $k$ starting from node $i$ ending in node $j$.
The coefficients $c_k$ indicates the weight of the paths and it is heavier the longer is the path, this is made to give more relevance to the short ones respect to the long ones. It must be chosen such that the series is convergent, they also must penalize long paths to reflect the preference to the shorter one.

An intuitive choice is $c_k = \frac{1}{k!}$, which transforms the communicability into an exponential function \cite{Estrada_2008}
\begin{equation}\label{G_E}
    G^E(A) =\sum_{k=1}^{\infty} \frac{A^k}{k!} = e^{A} .
\end{equation}
We can generalize it adding a constant term $\beta$
\begin{equation}
    G^E(A) =\sum_{k=1}^{\infty} \frac{\beta A^k}{k!} = e^{\beta A} ,
\end{equation}
this formulation is similar to the thermal green function for quantum system with Hamiltonian $A$ and temperature $T = \frac{1}{\beta}$.

Alternatively, we can choose $c_k = \alpha^{k}$ with $\alpha<\frac{1}{\lambda_N}$, where $\lambda_N$ is the largest eigenvalue of the adjacency matrix \cite{Katz}. In this case, it becomes a geometrical series yielding
\begin{equation}\label{G_R}
    G^R(A) =\sum_{k=1}^{\infty} \alpha^k A^k = (I -\alpha A)^{-1}.
\end{equation}
The two formulations for the communicability matrix lead to the same result and conclusion for the network in the limit $\alpha \rightarrow \frac{1}{\lambda_N}$ and $\lambda_N -\lambda_{N-1}$ large \cite{Benzi_Klymko}.


From this, we can introduce an global index for the network that consider all the different possible communication as
\begin{equation}
    EE(A)  = \Tr\left[e^{\beta A}\right].
\end{equation}
In the literature it is called Estrada index \cite{Estrada_2008} and can be interpreted as the sum of all the self-communication, that is the sum of the paths that end in the same node they have started.

However, the communicability matrices \eqref{G_E} and \eqref{G_R} study only the network's topology, namely the paths, and ignore the presence of dynamics over the network that may change the way information spreads.

If we consider the simplest dynamics, the random walk, it is governed by the Laplacian matrix $L$. 
Thus, the communicability matrices for random walk are \cite{Estrada_2012}
\begin{equation}\label{Estrada indeces}
    \begin{split}
        G^E(L) &=\sum_{k=1}^{\infty} \frac{\beta^k L^k}{k!} = e^{\beta L}  \\ 
        G^R(L) &= \sum_{k=1}^{\infty} \alpha^k L^k \rightarrow \alpha^{-1} \tilde{L}^{-1}
    \end{split}
\end{equation}

where $\tilde{L}^{-1} = \sum_{i=2}^N \frac{1}{\mu}v_i^Tv_i$ is the Moore-Penrose generalized inverse of the Laplacian. Here, $\mu$ are the eigenvalue ordered from the smaller to the bigger such that $\mu_1 < \mu_2 < ... < \mu_N$, and $v_i$ the respective eigenvectors of the Laplacian matrix \cite{Generalized_inverse_Laplacian}.
Also, the Laplacian Estrada index is define as
\begin{equation}\label{EE_L}
    EE(L) = \Tr\left[e^{\beta L}\right].
\end{equation}

%While the previous quantities using the adjacency matrix focalized over the topological aspects of the network and information spread, the laplacian communicability matrix embodies also the dynamical ones since the laplacian is involved in the random walk over a network. 

\subsection{Hamiltonian formalism}
The formulae \eqref{Estrada indeces} can be motivated by studying a classic and quantum harmonic oscillator on a network.
Consider a set of $N$ harmonic oscillators with coupling matrix $K = A$, where $A$ is a symmetric adjacency matrix. In this way, the nodes are considered as particle of mass $m = 1$ connected by springs with constant $A_{ij}$. The network should not have self interacting nodes, thus $A_{ii} = 0$. The system is submerged in a thermal bath at the temperature $T$. We assume there are no dumping and no external forces acting in the system besides the thermal fluctuation. 
Let introduce a set of coordinates $q_i$ that indicates the displacement of the $i$ particle respect the equilibrium position, the elastic elastic potential can be define as
\begin{equation}
    V(q) = \frac{1}{4}\sum_{i\neq j} K_{ij}(q_i-q_j)^2 = \frac{1}{2}\sum_{j}K_{jj}q_j^2 - \frac{1}{2} \sum_{i\neq j}K_{ij}q_iq_j,
\end{equation}
where 
\begin{equation}
    K_{jj} = \sum_{j \neq i} K_{ij}.
\end{equation}

We set $H_{ij}= K_{jj}\delta_{ij} - K_{ij}$, therefore the potential can be written as
\begin{equation}
    V(q) = \frac{1}{2}\sum_{i,j} H_{ij} q_i q_j.
\end{equation}
The $H$ matrix is a laplacian matrix and it is equal to the Laplacian of the network $L = D - A$, where $D$ is the degree matrix. It holds the property $\sum_j H_{ij} = 0$, therefore it has not negative eigenvalues and one must be equal to zero.
The zero eigenvalue ensure us that the motion of the center of must is conserved. %% find a way to cite this sentence

We can write the Lagrangian of the system as
\begin{equation}
    \mathcal{L} = \frac{1}{2}\sum_{ij} \dot q_i G_{ij} \dot q_j - \frac{1}{2} \sum_{ij} q_iH_{ij}q_j.
\end{equation}

The equations of motion are
\begin{equation}
    \ddot q_i = -H_{ij} q_j.
\end{equation}

The eigenmodes of the system are defined by the solution of the equation 
\begin{equation}
    \omega^2 \phi_i = H_{ij} \phi_j.
\end{equation}

Rewriting it in matrices form
\begin{equation}
    |\Omega^2 - H| = |\Omega^2 - H|.
\end{equation}

Therefore, the spectral signature of the matrix $H = L$ are the same of the harmonic oscillator. In this way we can connect the harmonic oscillator and the master equation of a network and vice versa. Since $M$ is diagonal, $H$ and $L$ have the same support, eigenvectors and eigenvalues, leading to $E = \omega^2 = \lambda$, which creates a natural ranking between the eigenvectors. 

The Hamiltonian of the system is given by
\begin{equation}\label{H_L}
    H_L = \sum_i \frac{p_i^2}{2} + \sum_{ij} \frac{1}{2}L_{ij}q_iq_j.
\end{equation}

\subsection{Network of classic harmonic oscillators}

To combine this with the thermodynamics, let consider the presence of a thermal bath in the Hamiltonian formalism using the Langevin equation
\begin{equation}
    \begin{aligned}
        &\dot q_i = p_i; \\
        &\dot p_i  = -H_{ij}q_j - \gamma \sum_j \left(\delta_{ij} - 1_{ij}\right)p_j + \sqrt{2T\gamma}\xi_i(t),
    \end{aligned}
\end{equation}
where $\gamma$ is the friction coefficient, $T$ is the temperature (Boltzmann constant $K_B =1$), $\delta_{ij}$ the Kronecker delta and $1_{ij}$ the matrix with all entries equal to 1, $\xi_i(t)$ is white noise, namely
\begin{equation}\label{white_noise}
    \langle\xi_i(t)\rangle = 0 \qquad \langle\xi_i^2(t)\rangle = 1 
\end{equation}

The white noises must hold the condition $\sum_i \xi_i = 0$, that leaves invariant the motion of  system's center of mass but $\xi_i(t)$ are no more independent.
As a matter of fact, the derivative of $\sum_i p_i$ is zero, therefore it is an integral of motion,
\begin{equation}
    \frac{d}{dt} \sum_i \dot p_i = - \gamma \sum_{ij}\left(\delta_{ij} - 1_{ij}\right)p_j+ \cancel{\sqrt{2T\gamma}\sum_i\xi_i(t)} = 0.
\end{equation}

The condition over the white noises $\sum_i \xi_i = 0$ adds breaks the independence between them and it adds correlation.
We can rewriting the noise using i.i.d. white noise $w_i(t)$ as
\begin{equation}
    \xi_i(t) = w_i (t) + \frac{1}{N} \sum_k w_k(t).
\end{equation}
The covariance matrix of $\xi_i(t)$ can be written as
\begin{equation}
    \left<\xi_i(t)\xi_j(s)\right> = \left[\delta_{ij} - 1_{ij}\right]\delta(t-s)
\end{equation}

The distribution $\rho(q,p,t)$ is a Gaussian and satisfies the Fokker-Plank equation \cite{Fokker}
\begin{equation}
    \frac{\partial\rho}{\partial t} = -\sum_i p_i\frac{\partial \rho}{\partial q_i} + \sum_{ij} H_{ij}q_j\frac{\partial \rho}{\partial p_i} + \gamma\sum_{ij}\left(\delta_{ij}-1_{ij}\right)\left[\frac{\partial}{\partial p_i}p_j\rho + T \frac{\partial^2\rho}{\partial p_i \partial p_j}\right].
\end{equation}
The dynamics converges to a stationary distribution with time scale depending on the eigenvalue of the Laplacian matrix.  
The solution at equilibrium is
\begin{equation}
    \rho(q,p) = Z(\beta)^{-1} \exp\left[ -\beta \left( \sum_j {p_j^2} + \sum_{ij} \frac{1}{2}q_iH_{ij}q_j\right)\right],
\end{equation}
where $\beta = \frac{1}{T}$ and $Z(\beta)$ is the partition function defined as
\begin{equation}
    Z(\beta) = \int \prod_i dp_i dq_i \; \exp\left[ -\beta \left( \sum_j {p_j^2} + \sum_{ij} q_iH_{ij}q_j\right)\right].
\end{equation}

The marginal distribution on the coordinates is a Maxwell-Boltzmann distribution with the internal energy 
\begin{equation}
    \rho(q) = Z(\beta)^{-1} e^{-\beta \left(\sum_{ij} q_iH_{ij}q_j\right)}.
\end{equation}

If $H$ is symmetric, namely the detailed balance condition holds, we can diagonalize the equation obtaining the motion of independent oscillators in the same thermal bath.
Therefore, changing the basis from $q_i$ to $Q_\lambda$ eigenvectors of the Hamiltonian, the marginal distribution becomes
\begin{equation}\label{marginal_probability}
    \rho(q) = Z(\beta)^{-1} e^{-\beta \left(\sum_{\lambda \neq 0} Q_\lambda \lambda Q_\lambda\right)},
\end{equation}
with partition function 
\begin{equation}
    Z(\beta) = \int \prod_{\lambda\neq 0} dQ_\lambda e^{-\beta \left(\sum_{\lambda \neq 0} \lambda Q_\lambda^2\right)}.
\end{equation}

The thermal distribution does not involve the zero eigenmode since the thermal bath does not interact with it. Thus, we can project the system into a invariant subspace orthogonal to the stationary distribution. The oscillator modes $Q_\lambda$ remain the same of the unperturbed case. 
This is a consequence of the condition $\sum_i \xi_i = 0$.
%Moreover, this is also connected to the conservation of the stationary distribution of the master equation \eqref{stationary_distribution}. 
The distribution has mean $\left<Q_\lambda\right>= 0$ and the covariance matrix is diagonal with entries $\left<Q^2_\lambda\right>= \frac{1}{\beta \lambda}$.

The variance can be expresses as
\begin{equation}\label{classic_correlation}
    \mathrm{Cov}(Q) = \frac{1}{\beta}L^{-1},
\end{equation}
where ${L}^{-1} = \sum_{i=2}^N \frac{1}{\mu_i}v_i^Tv_i$ is the Moore-Penrose generalized inverse of the Laplacian. Here, $\mu$ are the eigenvalue ordered from the smaller to the bigger such that $\mu_1 < \mu_2 < ... < \mu_N$, and $v_i$ the respective eigenvectors of the Laplacian matrix \cite{Generalized_inverse_Laplacian}.

This is the same result as the Estrada's Communicability matrix $G^R(L)$ \eqref{Estrada indeces} with $\alpha=\beta$.
When $T\rightarrow 0$ the spread of information drops; and when $T\rightarrow +\infty$ it becomes instantaneous.

\subsection{Network of quantum harmonic oscillators}

Instead, for the quantum case ($\hbar = 1$), $H_L$, $q_i$ and $p_j$ are promoted to operators $\hat H_L$, $\hat q_i$ and $\hat p_j$ and  they satisfy the commutator relation $\left[\hat q_i, \hat p_j\right] = i \delta_{ij}$.

We need to add a new term; it should be considered as additional springs with constant $K'$ that connect each node to the ground: it prevent the the center of mass from moving. So the Hamiltonian becomes
\begin{equation}\label{H_L_QM}
    H_L = \sum_i\left(\frac{\hat p_i^2}{2}+\frac{K'}{2}\hat q_i^2\right) + \sum_{ij}\frac{1}{2} L_{ij}\hat q_i\hat q_j.
\end{equation}

We introduce the bosons creation and annihilation operators as
\begin{equation}
     \hat a_i = \frac{1}{\sqrt{2}}\left(\sqrt{\Omega} \hat q_i + \frac{i}{\sqrt{\Omega}}\hat p_i\right) \qquad 
     \hat a_i^\dagger = \frac{1}{\sqrt{2}}\left(\sqrt{\Omega} \hat q_i - \frac{i}{\sqrt{\Omega}}\hat p_i\right), 
\end{equation}
and the inverse as
\begin{equation}
    \hat q_i = \sqrt{\frac{1}{2\Omega}}\left(\hat a_i + \hat a_i^\dagger\right) \qquad
    \hat p_i = i\sqrt{\frac{\Omega}{2}}\left(\hat a_i - \hat a_i^\dagger\right),
\end{equation}
where $\Omega = \sqrt{K'}$.
They satisfy the commutation relation $\left[\hat a_i, \hat a^\dagger_j\right] = \delta_{ij}$. 

The Hamiltonian can be written as 
\begin{equation}
    \hat H_L = \sum_i \Omega \left(\hat a_i\hat a^\dagger_i + \frac{1}{2}\right) + \frac{1}{4\Omega}\sum_{ij}\left(\hat a_i +\hat a_i^\dagger\right)L_{ij}\left(\hat a_i +\hat a_i^\dagger\right).
\end{equation}

Since The network is undirected, $L$ is symmetric and, therefore, we can diagonalize it. The diagonalized laplacian is written in the form $\Lambda = OLO^T$.
This generates a new pair of bosons creation and annihilation operators respect the eigenvalue $\mu$ of the Laplacian
\begin{equation}
    b_\mu = \sum_j a_jO_{\mu j}  \qquad \hat b_\mu^\dagger = \sum_j a_j^\dagger O^T_{\mu j} .
\end{equation}

Thus, the new Hamiltonian becomes a sum of independent Hamiltonians
\begin{equation}
    \hat H_L = \sum_\mu \hat H_\mu,
\end{equation}
with
\begin{equation}
    \begin{split}
        \hat H_\mu &= \Omega \left(\hat b_\mu\hat b^\dagger_\mu + \frac{1}{2}\right) + \frac{1}{4\Omega}\mu\left(\hat b_\mu +\hat b_\mu^\dagger\right)^2\\
        &= \Omega \left(\hat b_\mu\hat b^\dagger_\mu + \frac{1}{2}\right) + \frac{1}{4\Omega}\mu\left[\left(\hat b_\mu\right)^2 +\left(\hat b_\mu^\dagger\right)^2 + 2 \hat b_\mu \hat b_\mu^\dagger + 1 \right]\\
        &= \Omega \left[ 1 + \frac{1}{2\Omega}\mu\right] \left(\hat b_\mu\hat b^\dagger_\mu + \frac{1}{2}\right) + \frac{1}{4\Omega}\mu\left[\left(\hat b_\mu\right)^2 +\left(\hat b_\mu^\dagger\right)^2 \right].
    \end{split}
\end{equation}

We now consider the system as fermionic so the modes do not excite beyond the first excitation state. In this way we can restrict the Hilbert state of a mode to the span of the ground state $\ket{g}$ and the first excited state $\ket{e_\mu}=b_\mu^\dagger\ket{g}$. A consequence of it is that the second term in the Hamiltonian cancel out. 


Now, we can compute the thermal Green function or Matsubara Green function for fermions $G$. This quantity describes the probability amplitude for the particle to travel from one state to the other in a given time $\tau$ (more detail in the Appendix \ref{Appendix_B}). For $\tau > 0$ it is

\begin{equation}
    \begin{split}
        G^L_{ij}(\beta, \tau > 0) &= \frac{\Tr\left[e^{-\beta \hat H_L}\hat a_i (\tau)  \hat a_j^\dagger\right]}{\Tr\left[e^{-\beta \hat H_L}\right]} \\
        &=\sum_{\mu\nu} O_{\mu i}\frac{\Tr\left[ (\tau) e^{-\beta \hat H_L} \hat b_\mu\hat b_\nu^\dagger\right]}{\Tr\left[e^{-\beta \hat H_L}\right]}O_{j\nu}\\
        &= \sum_{\mu} O_{i\mu}\left\{-e^{-\mu \tau}\left[\left(1-f\left( \Omega + \frac{1}{2\Omega^2}\mu\right)\right)\Theta(\tau)\right]\right\}O_{j\mu}\\
        &= \sum_{\mu} O_{i\mu}\left\{\frac{e^{-\mu \tau}}{e^{-\beta\left[ \Omega + \frac{1}{2\Omega^2}\mu\right]} + 1}\right\}O_{j\mu}\\
    \end{split}
\end{equation}
In the limit $\tau \rightarrow 0^+$ and $\beta$ large enough it tend to
\begin{equation}
    G^L(\beta) = \sum_{\mu} O_{i\mu}{e^{\beta\left[ \Omega + \frac{1}{2\Omega^2}\mu\right]}} O_{\mu i},
\end{equation}
that can be written as
\begin{equation}
    G^L_{ij}(\beta)= e^{\beta \Omega}e^{\frac{\beta \omega^2}{2\Omega}L}.
\end{equation}

Comparing it with the eq. \eqref{Estrada indeces}, choosing $2\Omega =\omega^2$ the two equations are related as
\begin{equation}
    G^R(L) = e^{-\beta\Omega}G^L(\beta).
\end{equation}
When the temperature $T \rightarrow 0$ the communicability between the nodes drops to zero and the perturbation does not spread across the network. Instead, when $T \rightarrow \infty$ the communicability tend to infinity and the perturbation spread instantaneously.  
