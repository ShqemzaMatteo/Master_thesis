\section{Quantum Network Master Equation}

The previous description of noise in quantum walk lacks of parameter that in thermodynamics correspond to the temperature. 
Thus, we change picture: instead of considering the noise on the node's basis, we consider the thermal bath interaction in the energy's basis.
Let retake the standard Lindblad equation \eqref{Lindbladian} with Hamiltonian $\hat H = \hat L$.
We consider the basis $\{\ket{\lambda}\}_{\lambda}$ such that $\hat H = \sum_\lambda \lambda\ket{\lambda}\bra{\lambda} = \hat L$ is diagonal (the network must hold the detail balance condition \eqref{detail_condition}).
We define the jump operator $\hat J_{\lambda\mu} = \ket{\lambda}\bra{\mu}$ as the jumps from the energy states $\ket{\mu}$ to the energy state $\ket{\lambda}$ obtaining the master equation
\begin{equation}\label{Lindblad_energy_jump}
    \frac{d}{dt}\hat\rho = -i\left[\hat L,\hat\rho\right] + \sum_{\lambda\mu} \gamma_{\lambda\mu} \left(\hat J_{\lambda\mu} \hat\rho \hat J^\dagger_{\lambda\mu} - \frac{1}{2}\left\{ \hat J^\dagger_{\lambda\mu}\hat J_{\lambda\mu}, \hat\rho\right\} \right),
\end{equation}
the coefficients $\gamma_{\lambda\mu}$ indicate the probability to take their respective jumps.

We assume that the dynamics will tend to a stationary distribution in the form of
\begin{equation}\label{pretended_stationary_distribution}
    \hat \rho^* = \frac{e^{-\beta\hat L}}{Z},
\end{equation}
with $Z = \Tr\left[e^{-\beta\hat L}\right]$ is the partition function.
The master equation for the stationary distribution \eqref{pretended_stationary_distribution} reduces to
\begin{equation}\label{cancel_master_equation}
    0 = -i\left[\hat H, \frac{e^{-\beta\hat L}}{Z}\right] + \sum_{\lambda\mu} \gamma_{\lambda\mu} \left(\hat J_{\lambda\mu}  \frac{e^{-\beta\hat L}}{Z} \hat J^\dagger_{\lambda\mu} - \frac{1}{2}\left\{ \hat J^\dagger_{\lambda\mu}\hat J_{\lambda\mu},  \frac{e^{-\beta\hat H}}{Z}\right\} \right).
\end{equation}
The first term in the r.h.s. vanishes since the commutator is zero.
Now, we analyze the terms in the summation independently.
The first one can be written as 
\begin{equation}\label{BCS_first_term}
        \sum_{\lambda\mu} \gamma_{\lambda\mu} \ket{\lambda}\bra{\mu}\frac{e^{-\beta\hat L}}{Z} \ket{\mu}\bra{\lambda} = 
        \sum_{\lambda\mu} \gamma_{\lambda\mu} \frac{e^{-\beta \epsilon_\mu}}{Z} \ket{\lambda}\bra{\lambda}.
\end{equation}
While the second becomes
\begin{equation}\label{BCS_second_term}
    \sum_{\lambda\mu} \gamma_{\lambda\mu}\left[\frac{1}{2}\ket{\mu}\braket{\lambda}{\lambda}\bra{\mu}\frac{e^{-\beta\hat L}}{Z} +\frac{1}{2}\frac{e^{-\beta\hat L}}{Z}\ket{\mu}\braket{\lambda}{\lambda}\bra{\mu}\right]= \sum_{\lambda\mu} \gamma_{\lambda\mu}\left[\frac{e^{-\beta \epsilon_\mu}}{Z}\ket{\mu}\bra{\mu}\right].
\end{equation}
Therefore, inserting the equations \eqref{BCS_first_term} and \eqref{BCS_second_term} into the master equation \eqref{cancel_master_equation} we obtain 
\begin{equation}\label{Kirchhoff_law_energy}
    \sum_{\lambda\mu}\left[\gamma_{\lambda\mu}\frac{e^{-\beta \mu}}{Z} - \gamma_{\mu\lambda}\frac{e^{-\beta \lambda}}{Z}\right]\ket{\lambda}\bra{\lambda} = 0.
\end{equation}
It is the Kirchhoff's current law that says that the sum of all the currents must vanish. The system should satisfy this request in order to have the Boltzmann distribution.
However, fixed $\beta$, there are several possible choice for the coefficients $\gamma_{\lambda\mu}$ such that equation \eqref{Kirchhoff_law_energy} holds.
Each different choice generate a different path to reach the stationary distribution \eqref{pretended_stationary_distribution}.
We assume a strict condition, the detailed balance condition

We assume that the system must satisfy the maximal entropy production principle: the system should relax to the stationary distribution following the trajectory that has maximal entropy production. The principle is satisfied when the Kirchhoff law \eqref{Kirchhoff_law_energy} reduces to the detail balance condition
\begin{equation}
    \gamma_{\lambda\mu}\frac{e^{-\beta \epsilon_\mu}}{Z} - \gamma_{\mu\lambda}\frac{e^{-\beta \epsilon_\lambda}}{Z} = 0,
\end{equation}
which has solution
\begin{equation}\label{gamma_detailed_balance}
    \gamma_{\lambda\mu} = e^{-\frac{\beta}{2}\left(\epsilon_\lambda - \epsilon_\mu\right)}.
\end{equation}
%with $Z_\gamma = \sum_{\lambda\gamma}e^{-\frac{\beta}{2}\left(\epsilon_\lambda - \epsilon_\mu\right)}$ is the renormalization factor.

Takin the limits $\beta \rightarrow \infty$, that is $T  \rightarrow 0$, the transition rates tends to
\begin{equation}
    \gamma_{\lambda\mu} \rightarrow \left\{\begin{aligned}
        0 \qquad \lambda > \mu\\
        1 \qquad \lambda = \mu\\
        \infty \qquad  \lambda < \mu \\
    \end{aligned}\right. . 
\end{equation}
The transition from lower to higher energy state are suppressed, while the opposite one are extremely favorite. Thus, the system is led to the zero energy state that is the stationary state
\begin{equation}\label{beta_inf_stationary_distribution}
    \hat\rho^* = \ket{\lambda = 0}\bra{\lambda = 0}.
\end{equation} 
It is a pure state therefore the Von Neumann entropy vanishes.

The opposite limit $\beta \rightarrow 0$, that is $T \rightarrow \infty$, the transition rates becomes
\begin{equation}
    \gamma_{\lambda\mu} \rightarrow 1.
\end{equation}
We have the opposite effect, the particle can jump across the different energy state with uniform probability. Thus, the stationary distribution is the maximal entropy state, i.e. the uniform distribution.
\begin{equation}
    \hat\rho^* = \frac{1}{N}\begin{pmatrix}
        1&&0\\
        &\ddots&\\
        0&&1\\
    \end{pmatrix}.
\end{equation}
It is a maximal entropy state $S = \ln N$.

\subsection{Return to node's basis}
We can go back to the position basis $\{\ket{i}\}_{i<N}$, where $\ket{i}$ indicates the particle in the node $i$.  
The jump operators can be expressed in this basis as
\begin{align}
    \hat J_{\lambda\mu} = \sum_{ij} \braket{i}{\lambda}\braket{\mu}{j}\hat J_{ij}\\
    \hat J_{\lambda\mu}^\dagger = \sum_{ij} \braket{j}{\mu}\braket{\lambda}{i}\hat J_{ij}^\dagger
\end{align}
where $\hat J_{ij} = \ket{i}\bra{j}$.
Thus, the equation \eqref{Lindblad_energy_jump} becomes
\begin{equation}\label{quantum_network_position}
    \frac{d}{dt}\hat\rho = -i\left[\hat H,\hat\rho\right] +\sum_{ijkl} \gamma_{ij;kl} \left(\hat J_{ij}\hat\rho \hat J_{kl}^\dagger - \frac{1}{2}\left\{ \hat J_{kl}^\dagger\hat J_{ij}, \hat\rho\right\} \right),
\end{equation}
with the damping coefficient
\begin{equation}\label{gamma_position}
    \gamma_{ij;kl}= \sum_{\lambda\mu}\gamma_{\lambda\mu}\braket{i}{\lambda}\braket{\lambda}{k}\braket{l}{\mu}\braket{\mu}{j}.
\end{equation}
The master equation \eqref{quantum_network_position} is more complicated because its formulate in the general form \eqref{general_Lindblad_equation}. The damping rates are no more diagonal.
Inserting the equation \eqref{gamma_detailed_balance} into \eqref{gamma_position}, the damping rates in this basis are
\begin{equation}
    \gamma_{ij;kl}= \sum_{\lambda\mu}e^{-\frac{\beta}{2}\left(\epsilon_\lambda - \epsilon_\mu\right)}\braket{i}{\lambda}\braket{\lambda}{k}\braket{l}{\mu}\braket{\mu}{j}.
\end{equation}
\begin{comment}
    The master equation for the stationary distribution \eqref{pretended_stationary_distribution} in position space is 
    \begin{equation}
    0 = \sum_{ijkl} \gamma_{ij;kl} \left(\hat J_{ij}  \frac{e^{-\beta\hat H}}{Z} \hat J^\dagger_{kl} - \frac{1}{2}\left\{ \hat J^\dagger_{ij}\hat J_{kl},  \frac{e^{-\beta\hat H}}{Z}\right\} \right).
\end{equation}
The first term reduces to
\begin{equation}
    \sum_{ijkl}\gamma_{ij;kl} \ket{i}\bra{j} \frac{e^{-\beta\hat H}}{Z} \ket{l}\bra{k} = \sum_{ijkl}\gamma_{ij;kl} \frac{e^{-\beta H_{jl}}}{Z}\ket{i}\bra{k}.
\end{equation}
The anticommutator reduces to
\begin{equation}
    \begin{split}
        \sum_{ijkl}  \frac{\gamma_{ij;kl}}{2}\left(\ket{j}\braket{i}{k}\bra{l} \frac{e^{-\beta\hat H}}{Z}+ \frac{e^{-\beta\hat H}}{Z}\ket{j}\braket{i}{k}\bra{l}\right) \\ 
        = \sum_{ijlm} \left(\frac{\gamma_{ij;il}}{2}\frac{e^{-\beta H_{lm}}}{Z} + \frac{\gamma_{il;im}}{2}\frac{e^{-\beta H_{jl}}}{Z}\right)\ket{j}\bra{m}
    \end{split}
\end{equation}
We have applied some algebra and exchange dummy indices.
Thus, combining all together we obtain
\begin{equation}
    \begin{split}
        \sum_{ijkl}\left(\gamma_{ij;kl} \frac{e^{-\beta H_{jl}}}{Z} - \frac{\gamma_{ji;jl}}{2} \frac{e^{-\beta H_{lk}}}{Z} - \frac{\gamma_{jl;jk}}{2}\frac{e^{-\beta H_{il}}}{Z}\right)\ket{i}\bra{k} = 0\\
        \sum_{ijkl}\left(\gamma_{(ik)\leftarrow(jl)} \frac{e^{-\beta H_{jl}}}{Z} - \frac{\gamma_{(jj)\leftarrow(il)}}{2} \frac{e^{-\beta H_{lk}}}{Z} - \frac{\gamma_{(jj)\leftarrow(lk)}}{2}\frac{e^{-\beta H_{il}}}{Z}\right)\ket{i}\bra{k} = 0\\
        %\sum_{ijl}\left(\gamma_{ij;il} \frac{e^{-\beta H_{jl}}}{Z} - \frac{\gamma_{ji;ji}}{2} \frac{e^{-\beta H_{li}}}{Z} - \frac{\gamma_{jl;ji}}{2}\frac{e^{-\beta H_{il}}}{Z}\right)\ket{i}\bra{i} = 0
    \end{split}
\end{equation}
It can be interpreted as a Kirchhoff law in the link. Let suppose that the transition of the particle through the link is not instantaneous, then at each movement the particle chooses one link to travel. The probability to be in a specific link $(i,j)$ is given by $\frac{e^{-\beta H_{ij}}}{Z}$. In order to have a stationary distribution the dumping rate must satisfy the Kirchhoff law over the links distribution.
\end{comment}

Taking again the two limits, in the extremely hot environment, $\beta \rightarrow 0$, the damping rates are
\begin{equation}
    \gamma_{ij;kl}= \sum_{\lambda\mu} 1 \braket{i}{\lambda}\braket{\lambda}{k}\braket{l}{\mu}\braket{\mu}{j}.
\end{equation}
Using the completeness relation we reach
\begin{equation}
    \gamma_{ij;kl} = 1 \delta_{ik}\delta_{jl},
\end{equation}
where $\delta_{ik}$ is the Kronecker delta. Thus, the particle travel always through the same link.
In this case the position quantum network master equation \eqref{quantum_network_position} acquires a “symmetric" form
\begin{equation}
    \frac{d}{dt}\hat\rho = -i\left[\hat H,\hat\rho\right] +\sum_{ij}\left(\hat J_{ij}\hat\rho \hat J_{ij}^\dagger - \frac{1}{2}\left\{ \hat J_{ij}^\dagger\hat J_{ij}, \hat\rho\right\} \right).
\end{equation}
Since the relation \eqref{gamma_detailed_balance} can be modify by a constant, we recover the master equation for the Quantum Stochastic Walk \eqref{stochastic_lindblad_master}.
The stationary distribution is 
\begin{equation}
    \hat\rho^* = \frac{1}{N}\begin{pmatrix}
        1&&0\\
        &\ddots&\\
        0&&1\\
    \end{pmatrix}.
\end{equation}


In contrast, in the extremely cold environment, $\beta \rightarrow \infty$, the damping coefficient are
\begin{equation}
    \begin{split}
        \gamma_{ij;kl}&= \sum_{\lambda\mu}e^{-\frac{\beta}{2}\left(\epsilon_\lambda - \epsilon_\mu\right)}\braket{i}{\lambda}\braket{\lambda}{k}\braket{l}{\mu}\braket{\mu}{j}\Theta(\lambda-\mu)\\
        & + \sum_{\lambda\mu} 1\braket{i}{\lambda}\braket{\lambda}{k}\braket{l}{\mu}\braket{\mu}{j}\delta_{\lambda\mu}\\
        & + \sum_{\lambda\mu} e^{-\frac{\beta}{2}\left(\epsilon_\lambda - \epsilon_\mu\right)}\braket{i}{\lambda}\braket{\lambda}{k}\braket{l}{\mu}\braket{\mu}{j}\Theta(\mu-\lambda)
    \end{split}
\end{equation}
The first term cancel out,  and the second is a sum of Kronecker delta. Thus, it reduces to
\begin{equation}\label{gamma_T=0}
    \begin{split}
        \gamma_{ij;kl}&= \sum_{\lambda}\braket{i}{\lambda}\braket{\lambda}{k}\braket{l}{\lambda}\braket{\lambda}{j}+ \sum_{\mu>\lambda}\infty\braket{i}{\lambda}\braket{\lambda}{k}\braket{l}{\mu}\braket{\mu}{j} \rightarrow \infty
    \end{split}
\end{equation}
The stationary distribution \eqref{beta_inf_stationary_distribution} for the node reduces to
\begin{equation}
    \hat\rho^* = \sum_{ij} \braket{i}{\lambda = 0}\braket{\lambda = 0}{j} \ket{i}\bra{j} = \sqrt{\rho^*_i\rho^*_j} \ket{i}\bra{j}
\end{equation}
The Von Neumann entropy vanishes. 

Collegare a overdamped case 


\begin{comment}
    \newpage
    The same distribution can be obtain in position space. As a matter of fact, the master equation \eqref{stochastic_lindblad_master} for the distribution \eqref{pretended_stationary_distribution}. Let assume that it is the stationary distribution, thus
    \begin{equation}
    0 = \sum_{\lambda\mu} \gamma_{ij} \left(\hat J_{ij}  \frac{e^{-\beta\hat H}}{Z} \hat J^\dagger_{ij} - \frac{1}{2}\left\{ \hat J^\dagger_{ij}\hat J_{ij},  \frac{e^{-\beta\hat H}}{Z}\right\} \right).
\end{equation}

the first term can be written as
\begin{equation}
\sum_{ij} \gamma_{ij} \ket{i}\bra{j}  \frac{e^{-\beta\hat H}}{Z} \ket{j}\bra{i} = 
\sum_{ji} \gamma_{ij} \frac{e^{-\beta H_{jj}}}{Z} \ket{i}\bra{i}
\end{equation}

Whereas, the second term becomes
\begin{equation}
\sum_{ij}\frac{\gamma_{ij}}{2}\ket{j}\bra{j} \frac{e^{-\beta\hat H}}{Z} + \frac{\gamma_{ij}}{2}\frac{e^{-\beta\hat H}}{Z}\ket{j}\bra{j}=
\sum_{ijk} \frac{1}{2}\left(\gamma_{ij}\frac{e^{-\beta H_{jk}}}{Z} + \gamma_{ik}\frac{e^{-\beta H_{jk}}}{Z}\right)\ket{j}\bra{k}
\end{equation}

We can combine together all the terms obtaining
\begin{equation}
\begin{split}
\sum_{ji} \left(\gamma_{ji}\frac{e^{-\beta H_{ii}}}{Z} - \gamma_{ij}\frac{e^{-\beta H_{jj}}}{Z}\right)\ket{j}\bra{j} = 0\\
\sum_{jk} \frac{1}{2}\left(\sum_i\gamma_{ij} + \sum_i\gamma_{ik}   \right)\frac{e^{-\beta H_{jk}}}{Z}\ket{j}\bra{k} = 0\\
\end{split}
\end{equation}
The first equation is again the Kirchhoff's current law, while the second is a condition over the parameter $\gamma$.

If the previous condition are satisfy the stationary distribution of the system is the canonical one. In our case it becomes
\begin{equation}
\hat\rho^* = \frac{e^{-\beta\hat L}}{Z} \qquad Z = \Tr[e^{-\beta\hat L}]  
\end{equation}
that is the density matrix \eqref{density_matrix} introduced by De Domenico to identify networks.
\end{comment}


\subsection{Interpretation}

The equation \eqref{Lindblad_energy_jump} describes the evolution of a quantum walk in presence of a classic noise, the temperature $T$ determines the noise strength.
If we start from the stationary state and we add a perturbation in one node, the spread of the perturbation across the network depends on temperature: higher is the temperature, fewer nodes the perturbation can reach. Thus, the parameter $\beta$ permits us to analyze the system's dynamics over paths of varying length.

In addition, we can apply a Wick rotation to the quantum dynamics \eqref{Lindblad_energy_jump} returning to a special random walk of classic particle. The rotation connect the stationary distribution at inverse temperature $\beta$ \eqref{pretended_stationary_distribution} with the propagator at time $t$ \eqref{random_walk_solution}.
\begin{equation}
e^{-\beta\hat L} \rightarrow e^{-t\hat L}
\end{equation}

Thus, cooling down the quantum system is similar to the temporal evolution of the classical one. In fact, the two limits $\beta \rightarrow \infty$ and $t \rightarrow \infty$ converge to the same distribution: the system will be entirely in the zero eigenstate of the Laplacian.
Moreover, the density distribution \eqref{pretended_stationary_distribution} is always in the maximal entropy state, as a consequence, also the  distribution for the classical random walk should cross state with maximal entropy.

The complexity of the possible paths is encoded into the Von Neumann entropy as explained in chapter \ref{C_Density_Matrix}.
The entropy allows us to classify different network based on the dynamical properties of the network itself. 
We can do it introducing the Kullback-Lieber divergence \eqref{KL_divergence} and the Jensen-Shannon divergence \eqref{JS_metric}.
However, because this quantities employs the trace of a Laplacian's function, the entropy studies only the spectral properties of the system. Therefore, networks with same spectrum but different structure and eigenstate may be indistinguishable with these methods.