\subsection{Network of quantum harmonic oscillators}

Instead, for the quantum case ($\hbar = 1$), $H_L$, $q_i$ and $p_j$ are promoted to operators $\hat H_L$, $\hat q_i$ and $\hat p_j$ and  they satisfy the commutator relation $\left[\hat q_i, \hat p_j\right] = i \delta_{ij}$.

We need to add a new term; it should be considered as additional springs with constant $K'$ that connect each node to the ground: it prevent the the center of mass from moving. So the Hamiltonian becomes
\begin{equation}\label{H_L_QM}
    H_L = \sum_i\left(\frac{\hat p_i^2}{2}+\frac{K'}{2}\hat q_i^2\right) + \sum_{ij}\frac{1}{2} L_{ij}\hat q_i\hat q_j.
\end{equation}

We introduce the bosons creation and annihilation operators as
\begin{equation}
     \hat a_i = \frac{1}{\sqrt{2}}\left(\sqrt{\Omega} \hat q_i + \frac{i}{\sqrt{\Omega}}\hat p_i\right) \qquad 
     \hat a_i^\dagger = \frac{1}{\sqrt{2}}\left(\sqrt{\Omega} \hat q_i - \frac{i}{\sqrt{\Omega}}\hat p_i\right), 
\end{equation}
and the inverse as
\begin{equation}
    \hat q_i = \sqrt{\frac{1}{2\Omega}}\left(\hat a_i + \hat a_i^\dagger\right) \qquad
    \hat p_i = i\sqrt{\frac{\Omega}{2}}\left(\hat a_i - \hat a_i^\dagger\right),
\end{equation}
where $\Omega = \sqrt{K'}$.
They satisfy the commutation relation $\left[\hat a_i, \hat a^\dagger_j\right] = \delta_{ij}$. 

The Hamiltonian can be written as 
\begin{equation}
    \hat H_L = \sum_i \Omega \left(\hat a_i\hat a^\dagger_i + \frac{1}{2}\right) + \frac{1}{4\Omega}\sum_{ij}\left(\hat a_i +\hat a_i^\dagger\right)L_{ij}\left(\hat a_i +\hat a_i^\dagger\right).
\end{equation}

Since The network is undirected, $L$ is symmetric and, therefore, we can diagonalize it. The diagonalized laplacian is written in the form $\Lambda = OLO^T$.
This generates a new pair of bosons creation and annihilation operators respect the eigenvalue $\mu$ of the Laplacian
\begin{equation}
    b_\mu = \sum_j a_jO_{\mu j}  \qquad \hat b_\mu^\dagger = \sum_j a_j^\dagger O^T_{\mu j} .
\end{equation}

Thus, the new Hamiltonian becomes a sum of independent Hamiltonians
\begin{equation}
    \hat H_L = \sum_\mu \hat H_\mu,
\end{equation}
with
\begin{equation}
    \begin{split}
        \hat H_\mu &= \Omega \left(\hat b_\mu\hat b^\dagger_\mu + \frac{1}{2}\right) + \frac{1}{4\Omega}\mu\left(\hat b_\mu +\hat b_\mu^\dagger\right)^2\\
        &= \Omega \left(\hat b_\mu\hat b^\dagger_\mu + \frac{1}{2}\right) + \frac{1}{4\Omega}\mu\left[\left(\hat b_\mu\right)^2 +\left(\hat b_\mu^\dagger\right)^2 + 2 \hat b_\mu \hat b_\mu^\dagger + 1 \right]\\
        &= \Omega \left[ 1 + \frac{1}{2\Omega}\mu\right] \left(\hat b_\mu\hat b^\dagger_\mu + \frac{1}{2}\right) + \frac{1}{4\Omega}\mu\left[\left(\hat b_\mu\right)^2 +\left(\hat b_\mu^\dagger\right)^2 \right].
    \end{split}
\end{equation}

We now consider the system as fermionic so the modes do not excite beyond the first excitation state. In this way we can restrict the Hilbert state of a mode to the span of the ground state $\ket{g}$ and the first excited state $\ket{e_\mu}=b_\mu^\dagger\ket{g}$. A consequence of it is that the second term in the Hamiltonian cancel out. 


Now, we can compute the thermal Green function or Matsubara Green function for fermions $G$. This quantity describes the probability amplitude for the particle to travel from one state to the other in a given time $\tau$ (more detail in the Appendix \ref{Appendix_B}). For $\tau > 0$ it is

\begin{equation}
    \begin{split}
        G^L_{ij}(\beta, \tau > 0) &= \frac{\Tr\left[e^{-\beta \hat H_L}\hat a_i (\tau)  \hat a_j^\dagger\right]}{\Tr\left[e^{-\beta \hat H_L}\right]} \\
        &=\sum_{\mu\nu} O_{\mu i}\frac{\Tr\left[ (\tau) e^{-\beta \hat H_L} \hat b_\mu\hat b_\nu^\dagger\right]}{\Tr\left[e^{-\beta \hat H_L}\right]}O_{j\nu}\\
        &= \sum_{\mu} O_{i\mu}\left\{-e^{-\mu \tau}\left[\left(1-f\left( \Omega + \frac{1}{2\Omega^2}\mu\right)\right)\Theta(\tau)\right]\right\}O_{j\mu}\\
        &= \sum_{\mu} O_{i\mu}\left\{\frac{e^{-\mu \tau}}{e^{-\beta\left[ \Omega + \frac{1}{2\Omega^2}\mu\right]} + 1}\right\}O_{j\mu}\\
    \end{split}
\end{equation}
In the limit $\tau \rightarrow 0^+$ and $\beta$ large enough it tend to
\begin{equation}
    G^L(\beta) = \sum_{\mu} O_{i\mu}{e^{\beta\left[ \Omega + \frac{1}{2\Omega^2}\mu\right]}} O_{\mu i},
\end{equation}
that can be written as
\begin{equation}
    G^L_{ij}(\beta)= e^{\beta \Omega}e^{\frac{\beta \omega^2}{2\Omega}L}.
\end{equation}

Comparing it with the eq. \eqref{Estrada indeces}, choosing $2\Omega =\omega^2$ the two equations are related as
\begin{equation}
    G^R(L) = e^{-\beta\Omega}G^L(\beta).
\end{equation}
When the temperature $T \rightarrow 0$ the communicability between the nodes drops to zero and the perturbation does not spread across the network. Instead, when $T \rightarrow \infty$ the communicability tend to infinity and the perturbation spread instantaneously.  
