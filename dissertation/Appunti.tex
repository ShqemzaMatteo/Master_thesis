\chapter{Appunti}

Let's consider a network, in the node of the network we can put particle and they can move through the links. The $\pi_{ij}$ transition rates, we can write the master equation

\begin{equation}\label{master_equation}
    \dot p_i = \sum_j \pi_{ij}p_j + \pi_{ji} p_i = - \sum_j L_{ij} p_j
    \qquad L_{ij} = \sum_k \pi_{kj}\delta_{ij} - \pi_{ij} .
\end{equation}

The stationary distribution $p^0$ is define using matrix formulation as 
\begin{equation}
    L p^0 = 0 .
\end{equation}

Given any initial point, we can write 
\begin{equation}
    p(0) = \sum_\lambda c_\lambda p^\lambda
\end{equation}
where $p^\lambda$ are the eigenvector with eigenvalue $\lambda$ of the time evolution operator. In this way the evolution of the system can be describe as
\begin{equation}
    p(t) = \sum_\lambda e^{\lambda t} c_\lambda p^\lambda .
\end{equation}

Assuming the detailed balance holds $L_{ij} p^0_j= L_{ji}p^0_i$. We can consider $H_{ij}= L_{ij}p^0_j$ as a symmetric and Laplacian. 
Considering $m_i = p^0$ and Inserting it in the Lagrangian
\begin{equation}
    \mathcal{L} = \frac{1}{2} \sum_i m_i \dot q_i^2 +\frac{1}{2} \sum_i q_i H_{ij} q_j.
\end{equation}

the equation of motion are
\begin{equation} \label{eq_of_motion}
    m_i \ddot q_i = -H_{ij} q_j.
\end{equation}

The eigenmodes of the system are defined by the solution of the equation 
\begin{equation}
    m_i \omega^2 \phi_i = H_{ij} \phi_j.
\end{equation}

Using matrix formalism 

\begin{equation}
    |M\Omega^2 - H| = |\Omega^2 - HM^{-1}|
\end{equation}

Therefore the spectral signature of the matrix $HM^{-1} = L$ and the signature  of the eigenvalue of $L$ are the same of the harmonic oscillator. In this way we can connect the harmonic oscillator and the master equation of a network and viceversa. Because $M$ is diagonal, $H$ and $L$ have the same support and eigenvector and eigenvalue: $E = \omega^2 = \lambda$, so that the energy create a natural ranking between the eigenvectors.

Let now consider the presence of a thermal bath in the Hamiltonian formalism.
\begin{equation}
    \begin{aligned}
        &\dot q_i = \frac{p_i}{m_i}; \\
        &\dot p_i  = -H_{ij}q_j - \gamma \sum_j \left(\delta_{ij} - \frac{1_{ij}}{M}\right)\frac{p_j}{m_j} + \sqrt{2T\gamma}\xi_i(t)
    \end{aligned}
\end{equation}

So the derivative 
\begin{equation}
    \begin{aligned}
        \frac{d}{dt} \sum_i \dot p_i = - \gamma \sum_{ij}\left(\delta_{ij} - \frac{1_{ij}}{M}\right) \frac{p_j}{m_j} + \cancel{\sqrt{2T\gamma}\sum_i\xi_i(t)} = 0
    \end{aligned}
\end{equation}

If the initial condition is 
\begin{equation}
    \sum_i m_i q_i = 1 \qquad \sum_i p_i = 0 S
\end{equation}

rewriting the noise using a i.i.d. variable $w(t)$

\begin{equation}
    \xi_i(t) = w_i (t) + \frac{1}{M} \sum_k w_k(t)
\end{equation}

The variance can be written as

\begin{equation}
    \left<\xi_i(t)\xi_j(s)\right> = \left[\delta_{ij} - \frac{1_{ij}}{M}\right]\delta(t-s)
\end{equation}

The distribution $\rho(q,p,t)$ is a Gaussian and satisfy the Fokker-Plank equation.
\begin{equation}
    \rho(q,p) = A(\beta)^{-1} exp\left[ -\beta \left( \sum_j \frac{p_j^2}{2m_j} + \sum_{ij} q_iH_{ij}q_j\right)\right] .s
\end{equation}

It is possible to introduce a Lie operator 
\begin{equation}
    D_H I(z) =\left\{I,H\right\}_P = \sum_i\frac{\partial I}{\partial q_i}\frac{\partial H}{\partial p_i} - \frac{\partial H}{\partial q_i} \frac{\partial I}{\partial p_i} 
\end{equation}
where $z = (p,q)$ and $H = \sum_j \frac{p_j^2}{2m_j} + \sum_{ij} q_iH_{ij}q_j$ .

The dynamics can be written as 

\begin{equation}
    \dot I (z) = D_H I(z)
\end{equation}

We can now use the Dirac notation: $\ket{z}$ are a N dimensional vector, so
\begin{equation}
    \frac{d}{dt} \ket{z} = D_H \ket{z},
\end{equation}
And the eigenvector of the system respect to the energy are the same of that of the Laplacian operator. 
\begin{equation}
    E = \bra{z}\hat L\ket{z}.
\end{equation}

We can find the density matrix
\begin{equation}
    \hat \rho(t)= \sum_z \rho(z,t) \ket{z}\bra{z}
\end{equation}

and also the Von Neumann entropy 

\begin{equation}
    S = - \Tr \left[\hat\rho(t) \ln \hat\rho(t) \right] =- \sum_z\left[\rho(z,t) \ln\rho(z,t) \right]
\end{equation}

In the case of canonical the eigenstate of the harmonic oscillator are $\ket{\psi^\omega}$. The density matrix for the canonical ensemble in the configuration state reads
\begin{equation}
    \hat\rho = \sum_\omega e^{-\beta \omega} \ket{\psi^\omega}\bra{\psi^\omega}
\end{equation}
and the not normalize operator in the configuration space is
\begin{equation}
    \hat\rho = e^{-\beta \hat L}.
\end{equation}

The following equation holds

\begin{equation}
    \frac{d}{d\beta} \hat\rho = - \hat L \hat \rho
\end{equation}
that is basically the master equation \ref{master_equation} if we consider $\beta$ as the time.

\section{Symmetries}
Let consider that the Lagrangian of the system is invariant for the group $U(1)$, so
\begin{equation}
    e^{\alpha S^T}Me^{-\alpha S}= M \qquad e^{\alpha S^T}He^{-\alpha S}= H
\end{equation}

This means that $S$ anticommute with $M$ and $H$. Also $MS$ and $HS$ must be antisymmetric.
From this we can obtain a first integral of motion $I= \dot qMSq$. Indeed
\begin{equation}
    \frac{dI}{dt} = \ddot qMSq + \dot qMS \dot q = -qHSq  + \dot q MS \dot q = 0.
\end{equation}
We have apply the equation of motion \ref{eq_of_motion} and we know that $HS$ and $MS$ are antisymmetric. We can now prove The Noether theorem
\begin{equation}
    \frac{d}{d\alpha}\mathcal{L}(e^{\alpha S}q,e^{\alpha S}\dot q).
\end{equation}


\section{Di Domenico}

We want to use some formalism derived in quantum information to analyze the networks \cite{De_Domenico_2016}. Stating from the concept of density matrix $\rho$ and Von Neuman Entropy $S$, we want to find a suitable $H$ matrix.
\begin{equation}
    \rho = Z^{-1} e^{-\beta H} \qquad \quad S_V = -\Tr[\rho \ln \rho] 
\end{equation}

In the literature there are different candidate for $H$:
\begin{enumerate}
    \item $H = A$, $A$ the adjacency matrix, is useful to understand the diffusion of the information beyond the shorten path;
    \item $H= L$, the eigenvalue don't sum to unity;
    \item $H = L/\Tr L$, the subaddictive property does not hold. 
\end{enumerate}

We can diagonalize the Laplacian matrix, so 
\begin{equation}
    \begin{aligned}
        \rho =& \sum_i^N \frac{1}{Z} e^{-\beta \lambda_i} \\
        S(G) =& \frac{1}{Z} \sum_i^N e^{-\beta \lambda_i} \left[\ln Z + \beta \lambda_i \right]= \ln Z -\beta \frac{\partial \ln Z}{\partial\beta} \\
        &= \ln Z + \beta\Tr [L\rho]
    \end{aligned}
\end{equation}
We can compute the entropy for simple network:
\begin{enumerate}
    \item network of isolated nodes (no links): $S(G) = \ln Z$;
    \item network of just one link: $S(G) = \ln Z + \frac{2\beta e^{-2\beta}}{Z}$ with $Z= N-1+ e^{-2\beta}$:
    \item square lattice:
    \begin{equation}
        S(G) = \ln Z - \frac{\beta}{Z}\frac{\partial Z}{\partial \beta} \qquad Z = \sum_{j,m}^{\sqrt{N}} e^{-4\beta[\sin^2(\pi l/2\sqrt{N})+\sin^2(\pi m/2\sqrt{N})]};
    \end{equation}
    \item complete network: $N-1$ eigenvalue are equal to $N$, $S(G) = \ln Z + \frac{\beta N(z-1)}{Z}$
\end{enumerate}

In the paper Di Domenico compute numerically the Entropy of some type of graph as E-R, Watts-Strogatz and K-Graph at different parameter and temperature. He found out that varying the temperature the entropy increase monotonically reaching $S = 1$. Moreover the point when the entropy have a knee the spectral gap is max. 

He also found computationally the the quantum entropy holds the subaddictive property for some type of graph like E-R.
\begin{equation}
    S(G+G') \leq S(G) + S(G'). 
\end{equation}

We can generalize defying the Rényi q-entropy 
\begin{equation}
    S_q(G) = \frac{1}{1-q}\ln\Tr\rho^q
\end{equation}
and the q-relative Rényi entropy
\begin{equation}
    D_q(\sigma||\rho)= \frac{1}{1-q}\ln \Tr \left[ \rho^q \sigma^{1-q}\right]
\end{equation}

Di Domenico used the KL divergence for a maximum-likelihood estimation and model selection. It consist in minimize the function
\begin{equation}
    \min_\Theta[D_{KL}(\sigma||\rho)] = \max_\Theta[\Tr (\rho \ln \sigma(\Theta))]
\end{equation}
Another usage is to identify similar layer of a multiplex using the divergence as a distance.
