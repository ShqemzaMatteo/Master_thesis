\chapter{Lindblad Master Equation}\label{C_Lindblad}
Before exploring the following chapter, it is useful to introduce the Lindblad master equation, also called Gorini-Kossakowski-Sudarshan-Lindblad equation \cite{Lindblad,G_K_S}. This equation was introduced to explain the behavior of an open quantum system, namely a quantum system in contact with the environment. This is important because the Schrödinger equation applies only to closed systems which are idealized and not realistic: all the quantum experiment we can build are susceptible to the external environment.
In this way we can recover and justify the basic assumption physicists do in quantum statistical mechanics.

The model investigates the evolution of a quantum system coupled to a Markovian environment, the interaction has no memory of the past.  
The Schrödinger equation requires an unitary time operator that does not allow energy dissipations. In contrast, the time operator of Lindblad master equation permits the system to exchange energy with the surrounding. 
Despite this, the Lindblad dynamics remains trace preserving and completely positive.

\section{Derivation of the formula}
We show the derivation of the Lindblad equation following \cite{Manzano,Breuer-Petruccione}.
First, let $\mathcal{H}_T$ be the Hilbert space of the system and the environment combined, that can be divided between the Hilbert spaces $\mathcal{H}$ of the proper system and $\mathcal{H}_E$ of the environment. The combined system is a quantum closed system and evolves following the Von Neumann equation $\partial_t\hat\rho_T(t) = -i[\hat H_T,\hat\rho_T(t)]$, where $\hat H_T$ is the Hamiltonian of the total universe.
Since we are interesting only in the system's dynamics without the environment, we can trace out the degrees of freedom associated with it, obtaining $\hat\rho(t) = \Tr_E[\hat\rho_T]$. The total Hamiltonian can be separated as $H_T = H \otimes \mathbb{I}_E + \mathbb{I}_S \otimes H_E + \alpha H_I$, where $H$ is the Hamiltonian of the system, $H_E$ the Hamiltonian of the environment and $H_I$ is the interaction Hamiltonian, $\alpha$  measure the strength of the interaction. Tha Hamiltonians $H$ and $H_E$ commutes.
It is useful to work in the interaction pictures where the operators becomes
\begin{equation}
    \tilde O(T) = e^{i(\hat H+\hat H_E)t}\hat O e^{-i(\hat H+\hat H_E)t},
\end{equation}
and the Von Neumann equation reduces to 
\begin{equation}\label{C_interacting_picture}
    \frac{d\tilde\rho_T(t)}{dt}= -i\alpha\left[\tilde H_I(t),\tilde\rho_T(t)\right].
\end{equation}  
The solution to \eqref{C_interacting_picture} is 
\begin{equation}\label{interacting_picture_exact_solution}
    \tilde\rho_T(t) = \tilde\rho_T(0) -i\alpha\int_{0}^{t}ds\left[\tilde H_I(s),\tilde\rho_T(s)\right].
\end{equation}

Even though the equation \eqref{interacting_picture_exact_solution} has an exact solution, it is complicate to compute. To simplify the calculation, a perturbative approach is useful. We apply the \eqref{interacting_picture_exact_solution} into the \eqref{C_interacting_picture} giving
\begin{equation}
    \frac{d\tilde\rho_T(t)}{dt}= -i\alpha\left[\tilde H_I(t),\tilde\rho_T(0)\right] -\alpha^2 \int_{0}^{t}ds\left[\tilde H_I(t),\left[\tilde H_I(s),\tilde\rho_T(s)\right]\right]
\end{equation}
Applying this method again, we obtain
\begin{equation}
    \frac{d\tilde\rho_T(t)}{dt}= -i\alpha\left[\tilde H_I(T),\tilde\rho_T(0)\right] -\alpha^2 \int_{0}^{t}ds\left[\tilde H_I(t),\left[\tilde H_I(s),\tilde\rho_T(t)\right]\right] + O(\alpha^3)
\end{equation}
Now, we make an approximation: we consider the strength of the interaction $\alpha$ weak, in this way we can neglect the last term.
Then, we can trace out the environment obtaining
\begin{equation}\label{trace_C_interacting_picture}
    \frac{d\tilde\rho}{dt} = -i\alpha\Tr_E\left[\tilde H_I(T),\tilde\rho_T(0)\right] -\alpha^2 \int_{0}^{t}ds \Tr_E\left[\tilde H_I(t),\left[\tilde H_I(s),\tilde\rho_T(t)\right]\right].
\end{equation}

However, the equation \eqref{trace_C_interacting_picture} still depends on the total density matrix. To proceed, we make two more assumptions. First, we consider the initial state of the universe as a separable state $\hat\rho_T(0)=\hat\rho(0)\otimes \hat\rho_E(0)$. This holds if the system is just been put in contact with the environment or if the correlation between the system and the environment is short-lived. This is called Born approximation. 
Second, we consider the environment as a thermal reservoir, that is to be in a thermal state 
\begin{equation}
    \hat\rho_E(0) = \frac{e^{-\hat H_E/T}}{\Tr\left[e^{-\hat H_E/T}\right]},
\end{equation}
where $T$ is the temperature (the Boltzmann constant $k_B = 1$).
Moreover, without loss of generality, we can write the interaction Hamiltonian in the form
\begin{equation}\label{interacting_Hamiltonian}
    \hat H_I(t) = \sum_i \hat S_i\otimes \hat E_i,
\end{equation}
where $\hat S_i$ is an operator acting on $\mathcal{H}$ (is not a spin operator) and $\hat E_i$ is an operator acting on $\mathcal{H}_E$. After this assumption, the equation \eqref{trace_C_interacting_picture} becomes
\begin{equation}
    \begin{split}
        \frac{d\tilde\rho}{dt} =&-i\alpha\sum_i\left(\tilde S_i(t)\tilde \rho(0)\Tr_E\left[\tilde E_i(t)\tilde\rho_E(0)\right] - \tilde \rho(0)\tilde S_i(t)\Tr_E\left[\tilde\rho_E(0)\tilde E_i(t)\right]\right) \\ 
        &-\alpha^2 \int_{0}^{t}ds \Tr_E\left[\tilde H_I(t),\left[\tilde H_I(s),\tilde\rho(t)\otimes\tilde\rho_E(t)\right]\right].
    \end{split}
\end{equation}

The first term on the r.h.s. vanishes because $ \Tr_E\left[\tilde E_i(t)\tilde\rho_E(0)\right] = \left<E_i(t)\right>$ can be considered as zero. 
It may seem strange, however, if it does not vanish, we can always redefine the environmental Hamiltonian as $\hat E'_i = \hat E_i - \left<E_i(t)\right>$. The extra term is a constant and does not modify the Von Neumann equation.
The second term requires more stronger assumption: since $\alpha$ is small, the system and the environment should remain uncorrelated throughout the evolution, that is the timescale of the correlation should be much shorter than the timescale of the system. Thus, we can consider that the total density matrix is always separable, with the environment in the thermal state.
Nevertheless, the equation is still not markovian, since it still depend from a specific initial time $t= 0$. To add this property, we can extend the lower limit of the integration to infinity with no real change in the outcome; this is true when integrand disappears sufficiently fast \cite{Breuer-Petruccione}. Then, changing the integral variable to $t - s$, we arrive to 
\begin{eqnarray}\label{C_markovian}
    \frac{d\tilde \rho(t)}{dt} = -\alpha^2 \int_0^\infty ds \Tr_E\left[\tilde H_I(t),\left[\tilde H_I(t-s),\tilde\rho(t)\otimes\tilde\rho_E(t)\right]\right].
\end{eqnarray}
This is called Redfield equation \cite{Redfield}.
This is the Markov approximation, which is justified if the timescale over which the state of the system varies appreciably is large compared to the timescale over which the reservoir correlation functions decay. The sum of approximations made before are called Born-Markov approximation \cite{Breuer-Petruccione}.

%The equation \eqref{C_markovian} can generate a negative density matrix. 
%To exclude this possibility, we consider a superoperator $\mathbb{H} A = \left[H,A\right]$, with $A$ a general operator.

Now, we perform the last approximation known as rapid wave approximation which involves averaging over the rapid oscillating term. To do it, we consider the interaction Hamiltonian as in \eqref{interacting_Hamiltonian} and decompose it into eigenoperators if the the total Hamiltonian $H_I$.
These eigenoperator generate a complete basis of the space $\{\hat S_i(\omega)\}$ of the bounded operators acting on the Hilbert $\mathcal{H}$, they satisfy the condition
\begin{equation}
    \mathbb{H}\hat S_i(\omega) = -\omega \hat S_i(\omega) \qquad \mathbb{H}\hat S_i^\dagger(\omega) = \omega \hat S_i^\dagger(\omega).
\end{equation}
Here, $\omega$ indicates the energy difference after the operator $\hat A_i(\omega)$ has acted.
It satisfies the relations
\begin{equation}
    \begin{split}
        e^{i\hat H_St}\hat A(\omega)e^{-i\hat H_St} = e^{-i\omega t}\\
        e^{i\hat H_St}\hat A^\dagger(\omega)e^{-i\hat H_St} = e^{i\omega t}\\
    \end{split}
\end{equation}
We can decompose the operators $S_i$ as $\hat S_i = \sum_\omega \hat S_i(\omega)$.
To apply this decomposition in \eqref{C_markovian}, we need to go back to the Schrödinger picture for the Hamiltonian acting on the proper system. Using $\tilde S_i(\omega)=e^{i\hat Ht}\hat S_i(\omega)e^{-i\hat H t}$, we obtain the Hamiltonian in the interacting picture
\begin{equation}\label{eigen_Hamiltonian}
    \tilde H_i(t) = \sum_{i,\omega} e^{-i\hat Ht}\hat S_i(\omega) \otimes \tilde E_i (t)= \sum_{i,\omega} e^{i\hat Ht}\hat S_i^\dagger(\omega) \otimes \tilde E_i (t)
\end{equation}

We Insert it into \eqref{C_markovian} and expanding.
After expanding the commutators in, we substitute the decomposition for $\hat S_i(\omega)$. Using the cyclic property of the trace and the fact that $\Tr[\hat H_e,\hat\rho_E(0)]=0$, we arrive at the result
\begin{equation}\label{c_substitue}
    \frac{d\tilde\rho(t)}{dt} = \sum_{\omega,\omega',i,j}e^{i(\omega-\omega')t}\Gamma_{ij}\left[\hat S_j(\omega)\tilde\rho(t),\hat S_i^\dagger(\omega')\right]+ e^{-i(\omega-\omega')t}\Gamma_{ji}^\dagger\left[\hat S_j(\omega),\tilde\rho(t)\hat S_i^\dagger(\omega')\right],
\end{equation}
where $\Gamma_{kl}(\omega)$ contains the effect of the environment and it is defined as
\begin{equation}\label{environment_coefficients}
    \Gamma_{ij}(\omega) = \int_{0}^{\infty}ds e^{i\omega s}\Tr\left[\tilde E_i^\dagger(t)\tilde E_j(t-s)\hat\rho_E(0)\right]
\end{equation}
where the operator $\tilde E_j(t)=e^{i\hat H_E t}\hat E_j e^{-i\hat H_E t}$ is in the interaction picture. It does not depend on time since the environment is in a stationary state and the correlation function of the environment decay extremely fast.

Now, we make the final assumption: we consider the system in the rotating wave approximation. The terms proportional to $|\omega-\omega'| >> \alpha^2$ will oscillate much faster than the timescale of the system; thus, they do not contribute to its evolution. In the low-coupling regime, $\alpha\rightarrow 0$, we can consider that only the resonant terms, $\omega=\omega'$, contribute to the dynamics and remove all the others. Therefore, the equation \eqref{c_substitue} reduces to
\begin{eqnarray}\label{C_rotating_wave}
    \frac{d\tilde\rho(t)}{dt} = \sum_{\omega,i,j}\Gamma_{ij}\left[\hat S_j(\omega)\tilde\rho(t),\hat S_i^\dagger(\omega)\right]+\Gamma_{ji}^\dagger\left[\hat S_j(\omega),\tilde\rho(t)\hat S_i^\dagger(\omega)\right].
\end{eqnarray}

The operators $\Gamma_{ij}(\omega)$ are not necessarily Hermitian. Thus, we divide them into the Hermitian and not Hermitian parts, $\Gamma_{ij}(\omega) =\frac{1}{2}\gamma_{ij}(\omega)+i\pi_{ij}(\omega)$, respectively
\begin{equation}
    \begin{split}
        \gamma_{ij}(\omega) &=   \Gamma_{ij}(\omega) + \Gamma_{ij}^\dagger(\omega) = \int_{-\infty}^\infty ds e^{i\omega s}\Tr\left[\left\{\tilde E_i^\dagger(t),\tilde E_j(t-s)\right\}\hat\rho_E(0)\right]\\
        \pi_{ij}(\omega) &= \frac{-i}{2}\left(\Gamma_{ij}(\omega)-\Gamma_{ij}^\dagger(\omega)\right)=\int_{-\infty}^\infty ds e^{i\omega s}\Tr\left[\left[\tilde E_i^\dagger(t),\tilde E_j(t-s)\right]\hat\rho_E(0)\right]\\
    \end{split}
\end{equation}

Inserting them into the equation \eqref{C_rotating_wave} and returning to the Schrödinger picture, we obtain
\begin{equation}\label{general_Lindblad_equation}
    \frac{d}{dt}\hat\rho = -i\left[\hat H + \hat H_{LS},\hat\rho\right] + \sum_{i,j,\omega} \gamma_{ij}(\omega) \left(\hat S_i(\omega) \hat\rho \hat S^\dagger_j(\omega) - \frac{1}{2}\left\{ \hat S^\dagger_i(\omega)\hat S_j(\omega), \hat\rho\right\} \right),
\end{equation}
where $\hat H_{LS} = \sum_{\omega,i,j} \pi_{ij}(\omega)\hat S^\dagger_i(\omega)\hat S_j(\omega)$ is called Lamb shift Hamiltonian and it adjusts the energy levels due to the interaction with the environment. The equation \eqref{general_Lindblad_equation} is the general version of the Markovian master equation. The matrix $\gamma(\omega)$ must be positive define, although the trace preserving of the dynamics is not guaranteed.

If the matrix $\gamma(\omega)$ can be diagonalized, namely exist a diagonal matrix $D=\hat O \gamma(\omega) \hat O^\dagger$ with $\hat O$ a unitary operator, we can write the Lindblad-Gorini-Kossakowski-Sudarshan master equation as
\begin{equation}\label{Lindbladian}
    \frac{d}{dt}\hat\rho =\mathcal{L}\left[\hat\rho\right] = -i\left[\hat H+\hat H_{LS},\hat\rho\right] + \sum_k \gamma_k(\omega) \left(\hat J_k(\omega) \hat\rho \hat J^\dagger_k(\omega) - \frac{1}{2}\left\{ \hat J^\dagger_k(\omega)\hat J_k(\omega), \hat\rho\right\} \right).
\end{equation}
The operator $\hat J_k(\omega)= \sum_i O_{ki} \hat S_{i}(\omega)$ are called jump operators, the superoperator $\mathcal{L}$ is called Lindblad superoperator and $\gamma_i(\omega)$ are the damping rates. In the limit $\gamma_k(\omega) = 0$ the Von Neumann equation is recovered with the Hamiltonian $\hat H+\hat H_{LS}$.

\section{Time evolution}
Following the resolution proposed by Fujii \cite{fujii2012} for a quantum harmonic oscillator, we can solve the Lindblad master equation. 
First of all, we vectorize the density matrix: let introduce an Hilbert space with dimension $N^2$ such that a vector is $\sket{\rho} = (\rho_{00},\rho_{01},...,\rho_{NN-1},\rho_{NN})^T$ and the scalar product is $\sbraket{\phi}{\rho}= \Tr[\hat\phi^\dagger\hat\rho]$. This is called Fock-Liouville space \cite{Manzano}.

The follow operation can be vectorize as
\begin{equation}
    \hat A \hat\rho \hat B \rightarrow (\hat A\otimes\hat B)\sket{\rho} \qquad \hat A \hat\rho +\hat\rho\hat B \rightarrow \left(\hat A\otimes\mathbb{I} +\mathbb{I}\otimes\hat B\right)\sket{\rho},
\end{equation}
where $\mathbb{I}$ is the identity matrix, $\hat A$ and $\hat B$ are two generic operator. The glyph $\otimes$ represents the tensorial product that generates a $N^2\times N^2$ matrix defined as
\begin{equation}
    \hat A\otimes\hat B = \begin{pmatrix}
        A_{11} \hat B & \cdots & A_{1N}\hat B\\
        \vdots & \ddots &\vdots\\
        A_{n1}\hat B& \cdots & A_{NN}\hat B\\
    \end{pmatrix}.
\end{equation}
More detail are shown in the Appendix \ref{A_vectorial_density_matrix}.

In this space the Lindblad equation \eqref{Lindbladian} becomes
\begin{equation}\label{vectorize_master_equation}
    \frac{d}{dt}\sket{\rho(t)} = \tilde{\mathcal{L}}\sket{\rho(t)},
\end{equation}
where $\tilde{\mathcal{L}}$ is the operator 
\begin{equation}
    \tilde{\mathcal{L}}=-i\left[\hat H\otimes\mathbb{I}-\mathbb{I}\otimes\hat H\right]+\sum_{k}\gamma_{k}\left[\hat J_{k}\otimes\hat J^\dagger_{k} + \hat J_{k}^\dagger \hat J_{k}\otimes\mathbb{I} +\mathbb{I}\otimes\hat J_{k}^\dagger \hat J_{k}\right].
\end{equation}
The solution equation \eqref{vectorize_master_equation} can be written as
\begin{equation}
    \sket{\rho(t)} = \hat U(t,0)\sket{\rho(0)},
\end{equation}
where $U(t,0)$ is the evolution operator
\begin{equation}
    \begin{split}
        \hat U(t,0) = \exp&\left\{-it\left(\hat H\otimes\mathbb{I}-\mathbb{I}\otimes\hat H\right)\phantom{\left[\frac{1}{2}\right]}\right.\\
        &+\left. t\sum_{k}\gamma_{k}\left[\hat J_{k}\otimes\hat J^\dagger_{k}-\frac{1}{2}\hat J_{k}\hat J^\dagger_{k}\otimes\mathbb{I}-\frac{1}{2}\mathbb{I}\otimes\hat J_{k}\hat J^\dagger_{k}\right]\right\}
    \end{split}
\end{equation}
The evolution operator is not unitary.


\section{Properties of the Lindblad equation}
The Lindblad master equation satisfies some important properties.

It defines a set of dynamical maps $\phi_t\left(\hat\rho\right)= e^{\mathcal{L}t}\hat\rho(0)$ on the space of density matrices such that
\begin{equation}
    \hat\rho(t) = \phi_t\left(\hat\rho(0)\right).
\end{equation}
These maps have the semigroup property, that is,
\begin{equation}
    \phi_s\left(\phi_t\left(\hat\rho(0)\right)\right)=\phi_{t+s}\left(\hat\rho(0)\right)
\end{equation}

The Lindblad master equation is the most general form for the generator of a quantum dynamical semigroup. As a matter of fact, the Lindblad equation can also be derived starting from this assumption \cite{Breuer-Petruccione}.

The Lindblad master equation is invariant under the following transformations \cite{Breuer-Petruccione}:
\begin{itemize}
    \item Unitary transformation of the Lindblad operator:
        \begin{equation}
            \sqrt{\gamma_i}\hat J_i \rightarrow \sqrt{\gamma_i'}\hat J_i' = \sum_j u_{ij}\sqrt{\gamma_j}\hat J_j
        \end{equation}
        where $u_{ij}$ is an unitary matrix.
    \item Inhomogeneous transformation:
        \begin{equation}
            \begin{split}
                \hat J_i &\rightarrow \hat J'_i = \hat J_i + a_i\mathbb{I}\\
                \hat H_I &\rightarrow \hat H' = \hat H + \frac{1}{2i}\sum_j\gamma_j\left(a^*_j\hat J_j -a_j\hat J_j^\dagger\right) + b \mathbb{I}
            \end{split}
        \end{equation}
        where $a_i \in\mathbf{C}$ and $b \in \mathbf{R}$, $\mathbb{I}$ is the identity matrix.
\end{itemize}
The last transformation allows us to always choose a traceless jump operator.

Lastly, we can prove that the dynamics \eqref{Lindblad_master_equation} conserve $\Tr[\hat\rho]$. As a matter of fact, the derivative along time of it is
\begin{equation}\label{Lindblad_master_equation}
    \frac{d}{dt}\Tr[\hat\rho] = \Tr\left[-i\left(\hat H\hat\rho - \hat\rho\hat H \right) +  \hat J_k \hat\rho \hat J_k^\dagger -\frac{1}{2} \left(\hat H\hat\rho + \hat\rho\hat H\right)\right] = 0,
\end{equation}
proved using the cyclic property of the trace. 
However, it does not conserve the purity $\Tr\left[\hat\rho^2\right]$ that decreases \cite{Manzano}. 

\section{Entropy production}\label{C_entropy_production}

In thermodynamics, the irreversibility is encoded in the entropy function: a process is reversible if and only if the process does not produce entropy, $\Delta S = 0$, otherwise is irreversible. 
The Lindblad master equation \eqref{Lindbladian} should describe irreversible processes of an open quantum system, $\Delta S \geq 0$, since the thermal fluctuations destroy time reversal. 

Considering the derivative of the Von Neumann entropy
\begin{equation}
    \dot S\left(\hat\rho(t)\right) = -\Tr\left[\frac{d\hat\rho}{dt}\ln\hat\rho\right] + \Tr\left[\frac{d\hat\rho}{dt}\right]
\end{equation}
Knowing that the dynamics is trace preserving, namely $\Tr\left[\frac{d\hat\rho}{dt}\right] = 0$, it reduces to
\begin{equation}\label{entropy_production}
    \dot S\left(\hat\rho(t)\right) = -\Tr\left[\frac{d\hat\rho}{dt}\ln\hat\rho\right] 
\end{equation} 
To prove that the Lindblad dynamics \eqref{Lindbladian} is truly irreversible we can insert it into \eqref{entropy_production}. The Von Neumann dynamical part does not produce entropy, thus, we can only consider the dissipative one. We reach the equation
\begin{equation}
    \dot S(\hat\rho) = -\Tr\left[\sum_{k}\gamma_{k}\left[\hat J_{k} \hat\rho\hat J_{k}^\dagger -\frac{1}{2} \left\{ \hat J_{k}^\dagger \hat J_{k}, \hat\rho\right\}\right]\ln\hat\rho\right].
\end{equation}
We expand the commutator
\begin{equation}
    \dot S(\hat\rho) =-\sum_{k}\gamma_{k}\Tr\left[\hat J_{k} \hat\rho\hat J_{k}^\dagger\ln\hat\rho -\frac{1}{2}\hat J_{k}^\dagger \hat J_{k}\hat\rho \ln\hat\rho-\frac{1}{2} \hat\rho\hat J_{k}^\dagger \hat J_{k}\ln\hat\rho\right]\\
\end{equation}     
Since $\ln\hat\rho$ and $\hat\rho$ commute, the second and third terms can be summed.
\begin{equation}\label{Lindbladian_entropy_production}
    \dot S(\hat\rho) =  -\sum_{k}\gamma_{k}\left(\Tr\left[\hat J_{k} \hat\rho\hat J_{k}^\dagger\ln\hat\rho\right]-\Tr\left[\hat J_{k}^\dagger \hat J_{k}\hat\rho\ln\hat\rho\right]\right)
\end{equation}
To ensure that $\dot S(\hat\rho) > 0$, the system must satisfy the following condition
\begin{equation}\label{entropy_inequality}
    \sum_{k}\gamma_{k}\Tr\left[\hat J_{k}\hat\rho\hat J_{k}^\dagger\ln\hat\rho\right] \leq \sum_{k}\gamma_{k} \Tr\left[\hat J_{k}^\dagger\hat J_{k}\hat\rho\ln\hat\rho\right],
\end{equation}
that depends on the choice of the $\gamma_{k}$. 
Since the damping rates are positive, the inequality \eqref{entropy_inequality} holds if the following inequality is satisfied
\begin{equation}\label{reduced_entropy_inequality}
    \Tr\left[\hat J_{k}\hat\rho\hat J_{k}^\dagger\ln\hat\rho\right] \leq \Tr\left[\hat J_{k}^\dagger\hat J_{k}\hat\rho\ln\hat\rho\right].
\end{equation}
It can be proved to be satisfied for any choice of $\hat J_{k}$.
First we diagonalize the density matrix. Let $\ket{\lambda}$ be the eigenstate with eigenvalue $\lambda$, the density matrix can be written as
\begin{equation}
    \hat \rho = \sum_\lambda \rho_\lambda\ket{\lambda}\bra{\lambda}.
\end{equation}
We transform also the jump operator in this basis $\hat{\mathcal{J}}_{k}= \hat O \hat J_{k}\hat O^\dagger$.
The l.h.s. of the inequality \eqref{reduced_entropy_inequality} with some algebra can be reduces to 
\begin{equation}
    \begin{split}
        \Tr\left[\hat J_{k} \hat\rho\hat J_{k}^\dagger\ln\hat\rho\right]=& \sum_{\lambda\mu}\Tr\left[\hat{\mathcal{J}}_{k} \rho_\lambda\ket{\lambda}\bra{\lambda}\hat{\mathcal{J}}_{k}^\dagger\ln\rho_\mu \ket{\mu}\bra{\mu}\right] \\
        %&= \sum_{\lambda\mu}\rho\lambda\ln\rho\mu\Tr\left[\bra{\mu}\hat{\mathcal{J}}_{ij} \ket{\lambda}\bra{\lambda}\hat{\mathcal{J}}_{ij}^\dagger\ket{\mu}\right]\\
        &= \sum_{\lambda\mu}\rho_\lambda\ln\rho_\mu\Tr\left[|\bra{\mu}\hat{\mathcal{J}}_{k} \ket{\lambda}|^2\right]\\
        %&= \sum_{\lambda\mu}\rho_\lambda\ln\rho_\mu\Tr\left[x^{(ij)}_{\lambda,\mu}\right]\\
        &= N \sum_{\lambda\mu}\rho_\lambda\ln\rho_\mu x^{(k)}_{\lambda\mu}
    \end{split}
\end{equation}
where $x^{(k)}_{\lambda\mu} = |\bra{\mu}\hat{\mathcal{J}}_{k} \ket{\lambda}|^2$ is a non negative scalar. It is symmetric respect the change $\lambda \leftrightarrow \mu$. 

The other term becomes
\begin{equation}
    \begin{split}
        \Tr\left[\hat J_{k}^\dagger \hat J_{k}\hat\rho\ln\hat\rho\right]=& \sum_{\lambda\mu}\Tr\left[\hat{\mathcal{J}}_{k}^\dagger \hat{\mathcal{J}}_{k}\rho_\lambda\ket{\lambda}\bra{\lambda}\ln\rho_\mu\ket{\mu}\bra{\mu}\right]\\
        & = \sum_{\lambda\mu}\rho_\lambda\ln\rho_\lambda\Tr\left[\bra{\mu}\hat{\mathcal{J}}_{k}^\dagger\hat{\mathcal{J}}_{k}\ket{\lambda}\braket{\lambda}{\mu}\right]\\
        &= \sum_{\lambda}\rho_\lambda\ln\rho_\lambda\Tr\left[\bra{\lambda}\hat{\mathcal{J}}_{k}^\dagger\hat{\mathcal{J}}_{k}\ket{\lambda}\right]\\
    \end{split}
\end{equation}

The second braket in the trace is just a Kronecker delta. We use the completeness relation $I = \sum_\mu\ket{\mu}\bra{\mu}$ obtaining
\begin{equation}
    \begin{split}
        \Tr\left[\hat J_{k}^\dagger \hat J_{k}\hat\rho\ln\hat\rho\right]&= \sum_{\lambda\mu}\rho_\lambda\ln\rho_\lambda\Tr\left[\bra{\lambda}\hat{\mathcal{J}}_{k}^\dagger\ket{\mu}\bra{\mu}\hat{\mathcal{J}}_{k}\ket{\lambda}\right]\\
        &= N\sum_{\lambda\mu}\rho_\lambda\ln\rho_\lambda x^{(ij)}_{\lambda\mu}.\\
    \end{split}
\end{equation}


The inequality \eqref{reduced_entropy_inequality} reduces to 
\begin{equation}
    N \sum_{\lambda\mu}\rho_\lambda\ln\rho_\mu x^{(k)}_{\lambda\mu} \leq N\sum_{\lambda\mu}\rho_\lambda\ln\rho_\lambda x^{(k)}_{\lambda\mu}.
\end{equation}

We can rearrange the term in the two sum as
\begin{equation}
    \begin{split}
        \sum_{\lambda}\sum_{\mu<\lambda}\left(\rho_\lambda\ln\rho_\mu + \rho_\mu\ln\rho_\lambda \right) x^{(k)}_{\lambda\mu} +&\sum_\lambda \rho_\lambda\ln\rho_\lambda x^{(k)}_{\lambda\lambda} \leq\\ 
        \sum_{\lambda}\sum_{\mu<\lambda}&\left(\rho_\lambda\ln\rho_\lambda+ \rho_\mu\ln\rho_\mu\right)x^{(k)}_{\lambda\mu} + \sum_{\lambda}\rho_\lambda\ln\rho_\lambda x_{\lambda\lambda}{(k)} 
    \end{split}
\end{equation}

Therefore, it is sufficient that the following equation is satisfied
\begin{equation}
    \rho_\lambda\ln\rho_\mu + \rho_\mu\ln\rho_\lambda\leq \rho_\lambda\ln\rho_\lambda+ \rho_\mu\ln\rho_\mu.
\end{equation}

Moving all the term in the right hand side we reach
\begin{equation}
    \begin{split}
        \rho_\lambda\ln\rho_\mu + \rho_\mu\ln\rho_\lambda - \rho_\lambda\ln\rho_\lambda- \rho_\mu\ln\rho_\mu \leq 0\\
        %\rho_\lambda\ln\left(\frac{\rho_\mu}{\rho_\lambda}\right) - \rho_\mu\ln\left(\frac{\rho_\mu}{\rho_\lambda}\right) \leq 0\\
        \left(\rho_\lambda-\rho_\mu\right)\ln\left(\frac{\rho_\mu}{\rho_\lambda}\right) \leq0\\
    \end{split}
\end{equation}
That is always satisfied. The equality is satisfied for the stationary distribution.

The last result tell us that the dynamics increases the entropy and, thus, change a pure density matrix in a mixed one.
As a consequence, the stationary distribution must have maximum entropy.

\section{Stationary distribution}
The Lindblad equation allows for a stationary distribution that satisfies the condition
\begin{equation}
    \mathcal{L}\hat\rho = 0.
\end{equation}

In the previous section, we assumed that the environment is in a Gibbs state. Now, consider that the damping parameter satisfies the relation
\begin{equation}\label{C_gamma}
    \gamma_{ij}(-\omega) = e^{-\beta \omega}\gamma_{ij}(\omega)
\end{equation}
which is called KMS condition \cite{Breuer-Petruccione}.
If this condition is satisfied, it can be proven that the stationary distribution is equal to the Gibbs states \cite{Breuer-Petruccione}
\begin{equation}
    \hat\rho^* = \frac{e^{-\beta \hat H}}{\Tr\left[e^{-\beta \hat H}\right]}.
\end{equation}

If the spectrum of the Hamiltonian $H = \sum_n \epsilon_n\ket{n}\bra{n}$ is not degenerate, it gives rise to a closed equation for the population 
\begin{equation}
    P(n,t) = \bra{n}\hat\rho(t)\ket{n}
\end{equation}
Thus, the dynamics decouple the diagonal and off-diagonal terms. The former are governed by the Pauli master equation
\begin{equation}\label{to_Fermi_golden_rule}
    \frac{dP(n,t)}{dt} = \sum_m \left[ W(n|m)P(m,t) - W(m|n)P(n,t)\right]
\end{equation}
with time independent transition rate
\begin{equation}
    W(n|m) = \sum_{ij} \gamma_{ij}(\epsilon_n -\epsilon_m)\bra{n}\hat J_i(t)\ket{m} \bra{m}\hat J_j(t)\ket{n}. 
\end{equation}

Using the equation \eqref{C_gamma} and considering the l.h.s. of the equation \eqref{to_Fermi_golden_rule} , we obtain the Fermi Golden rule
\begin{equation}
    W(n|m)e^{-\beta \epsilon_n} = W(m|n) e^{-\beta \epsilon_m}
\end{equation}
which is nothing other that a detailed balance condition with stationary distribution
\begin{equation}
    \hat\rho = \frac{1}{Z}e^{-\beta \hat H}
\end{equation}
with $\beta$ the inverse temperature of the environment and $Z = \Tr\left[e^{-\beta \hat H}\right]$ is the partition function.



%\section{Caldeira-Leggett model}

The Caldeira-Leggett model was proposed in 1983 to reproduce the Brownian motion via a quantum process.
It study the dynamics of a particle in contact with a thermal bath made by a set of harmonic oscillator.
However, the model present the Markovian propery only in the high temperature weak coupling limit and a not Markovian one in otherwise.

The free Hamiltonian of the particle is
\begin{equation}
    H_S = \frac{\hat p^2}{2m} + V(\hat q),
\end{equation}
where $\hat p$ is the momentum operator and $V(\hat q)$ is the potential.

The particle is in contact with the bath composed by N harmonic oscillator. The Hamiltonian is
\begin{equation}
    H_E = \sum\Omega_n(\hat b_n^\dagger\hat b_n+\frac{1}{2})= \sum_n \left(\frac{p_n^2}{2m_n}+ \frac{1}{2}m_n\Omega_n^2q_n^2\right).
\end{equation}
where $\hat b_n^\dagger$ and $\hat b_n$ are respectively the creation and annihilation operators of the bath, and $p_n$ and $q_n$ the momentum and position operator of the oscillators of bath.

The interaction Hamiltonian is given by
\begin{equation}
    H_I = \hat q\sum_n k_n \hat q_n = - \hat B \hat q = \hat q \sum_n k_n\frac{1}{\sqrt{2m_n\Omega_n}}\left(\hat b+ \hat b^\dagger\right)
\end{equation}
where $\hat B$ is the bath operator.

We need to add a counterterm 
\begin{equation}
    H_C = \sum_n \frac{\hat q^2}{2m_n\Omega_n^2}. 
\end{equation}

Starting from equation \eqref{C_markovian} in the Schrödinger pictures
\begin{equation}
    \frac{d \rho(t)}{dt} = i\left[\hat H_S+ \hat H_C, \hat\rho(t)\right] -\int_0^\infty ds \Tr_E\left[\hat H_I(t),\left[\hat H_I(t-s),\hat\rho(t)\otimes\hat\rho_E(t)\right]\right],
\end{equation}

We can rewrite it using the correlation relations of the bath
\begin{equation}
    D(s) =i \left<\left[\hat B, \hat B^\dagger(t-s)\right]\right> \qquad D'(s) = \left<\left\{\hat B, \hat B^\dagger(t-s)\right\}\right>
\end{equation}
into the formula 
\begin{equation}\label{CL_master_eq}
    \frac{d \rho(t)}{dt} = -i\left[\hat H_S+ \hat H_C, \hat\rho(t)\right] -\int_0^\infty ds \frac{i}{2} D(s)\left[\hat q, \left\{\hat q(-s), \hat \rho(t)\right\}\right] - \frac{1}{2} D'(s)\left[\hat q, \left[\hat q(-s), \hat\rho(t)\right]\right]  
\end{equation}

Using the spectral density
\begin{equation}
    J(\Omega) = \sum_n \frac{k_n^2}{2m_n\Omega}\delta(\Omega-\Omega_n)
\end{equation}
we can rewrite the correlation as
\begin{equation}
    D(s) = 2\int_0^\infty d\Omega J(\Omega) \sin\Omega s \qquad D'(s) = 2\int_0^\infty d\Omega J(\Omega) \coth\left(\frac{\Omega}{2k_BT}\right)\cos\Omega s.
\end{equation}

We can simplify further the equation \eqref{CL_master_eq} looking at the $\hat q(-s)$, indeed, it can be written as 
\begin{equation}
    \hat q(-s) = e^{-i\hat H_S s} \hat q e^{i\hat H_S s} \approx \hat q- i\left[\hat H_S, \hat q\right]s = \hat q - \frac{\hat p}{m}s 
\end{equation}

Inserting it into \eqref{CL_master_eq} we obtain
\begin{equation}
    \begin{split}
        \frac{d \rho(t)}{dt} = & -i\left[\hat H_S+ \hat H_C, \hat\rho(t)\right] -\frac{i}{2}\int\limits_0^\infty ds  D(s)\left[\hat q, \left\{\hat q, \hat \rho(t)\right\}\right]\\
        &+ \frac{i}{2m}\int_0^\infty ds \, sD(s)\left[\hat q, \left\{\hat p, \hat \rho(t)\right\}\right] - \frac{1}{2}\int_0^\infty ds D'(s)\left[\hat q, \left[\hat q, \hat\rho(t)\right]\right]\\
        & +\frac{1}{2m}\int_0^\infty ds \; sD'(s)\left[\hat q, \left[\hat p, \hat\rho(t)\right]\right].\\
    \end{split}  
\end{equation}

Then, we solve the integral in the high temperature limit: the first term give us the exactly the counter term  $-i\left[\hat H_C, \hat \rho\right]$, the second becomes $-\frac{i\gamma}{2}\left[\hat q, \left\{\hat p,\hat \rho(t)\right\}\right]$, the third becomes $2m \gamma k_BT\left[\hat q,\left[\hat q,\hat\rho(t)\right]\right]$ and the last give us $\frac{2\gamma k_BT}{\Omega}\left[\hat q, \left[\hat p, \hat \rho\right]\right]$ which is negligible in the limit $\Omega \rightarrow \infty$ that we are dealing with.

We have reached the Caldeira-Leggett master equation
\begin{equation} \label{Caldeira-Leggett_master_equation}
    \frac{d\hat\rho}{dt} = -i\left[\hat H_S, \hat\rho(t)\right]-\frac{i\gamma}{2}\left[\hat q, \left\{\hat p,\hat \rho(t)\right\}\right] - 2m \gamma k_BT\left[\hat q,\left[\hat q,\hat\rho(t)\right]\right]. 
\end{equation}
The Caldeira-Leggett master equation reminds of the Langevin equation where the second term is the dissipative one and the third describes the fluctuations.
