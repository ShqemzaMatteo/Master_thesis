\chapter{Lindblad Master Equation}\label{C_Lindblad}
Before exploring the following chapter, it is useful to introduce the Lindblad master equation, also called Gorini-Kossakowski-Sudarshan-Lindblad equation \cite{Lindblad,G_K_S}. This equation was introduced to describe the behavior of an open quantum system, namely a quantum system in contact with the environment. This is important because the Schrödinger equation applies only to closed systems, which are idealized and not realistic: all the quantum experiments we can build are influenced to the external environment.

The model investigates the evolution of a quantum system coupled to a Markovian environment, the interaction has no memory of the past.  
The Schrödinger equation requires a unitary time operator that does not permit energy dissipation. In contrast, the time operator of Lindblad master equation allows the system to exchange energy with its surroundings. 
Despite this, the Lindblad dynamics remains trace preserving and completely positive.

\section{Derivation of the formula}
We show the derivation of the Lindblad equation following \cite{Manzano,Breuer-Petruccione}.
First, let $\mathcal{H}_T$ be the Hilbert space of the system and the environment combined, which can be divided into the Hilbert space $\mathcal{H}$ of the proper system and $\mathcal{H}_E$ of the environment. The combined system is a closed quantum system and evolves following the von Neumann equation 
\begin{equation}
    \partial_t\hat\rho_T(t) = -i[\hat H_T,\hat\rho_T(t)],
\end{equation}
where $\hat H_T$ is the Hamiltonian of the total universe.
Since we are only interesting in the system's dynamics without the environment, we can trace out the degrees of freedom associated with it, obtaining $\hat\rho(t) = \Tr_E[\hat\rho_T]$. The total Hamiltonian can be separated as 
\begin{equation}
    H_T = H \otimes \mathbb{I}_E + \mathbb{I}_S \otimes H_E + \alpha H_I,
\end{equation}
where $H$ is the Hamiltonian of the system, $H_E$ the Hamiltonian of the environment and $H_I$ is the interaction Hamiltonian, $\alpha$  measure the strength of the interaction. Tha Hamiltonians $H$ and $H_E$ commutes.
It is useful to work in the interaction picture, where the operators becomes
\begin{equation}
    \tilde O(T) = e^{i(\hat H+\hat H_E)t}\hat O e^{-i(\hat H+\hat H_E)t},
\end{equation}
and the von Neumann equation reduces to 
\begin{equation}\label{C_interacting_picture}
    \frac{d\tilde\rho_T(t)}{dt}= -i\alpha\left[\tilde H_I(t),\tilde\rho_T(t)\right].
\end{equation}  
The solution to \eqref{C_interacting_picture} is 
\begin{equation}\label{interacting_picture_exact_solution}
    \tilde\rho_T(t) = \tilde\rho_T(0) -i\alpha\int_{0}^{t}ds\left[\tilde H_I(s),\tilde\rho_T(s)\right].
\end{equation}

Even though the equation \eqref{interacting_picture_exact_solution} has an exact solution, it is complicated to compute. To simplify the calculation, a perturbative approach is useful. We apply the equation \eqref{interacting_picture_exact_solution} into the equation \eqref{C_interacting_picture} yielding
\begin{equation}
    \frac{d\tilde\rho_T(t)}{dt}= -i\alpha\left[\tilde H_I(t),\tilde\rho_T(0)\right] -\alpha^2 \int_{0}^{t}ds\left[\tilde H_I(t),\left[\tilde H_I(s),\tilde\rho_T(s)\right]\right]
\end{equation}
Applying this method again, we obtain
\begin{equation}
    \frac{d\tilde\rho_T(t)}{dt}= -i\alpha\left[\tilde H_I(T),\tilde\rho_T(0)\right] -\alpha^2 \int_{0}^{t}ds\left[\tilde H_I(t),\left[\tilde H_I(s),\tilde\rho_T(t)\right]\right] + O(\alpha^3)
\end{equation}
Now, we make an approximation: we consider the strength of the interaction $\alpha$ to be weak, allowing us to neglect the last term.
Then, we can trace out the environment, obtaining
\begin{equation}\label{trace_C_interacting_picture}
    \frac{d\tilde\rho}{dt} = -i\alpha\Tr_E\left[\tilde H_I(T),\tilde\rho_T(0)\right] -\alpha^2 \int_{0}^{t}ds \Tr_E\left[\tilde H_I(t),\left[\tilde H_I(s),\tilde\rho_T(t)\right]\right].
\end{equation}

However, the equation \eqref{trace_C_interacting_picture} still depends on the total density matrix. To proceed, we make two more assumptions. First, we consider the initial state of the universe to be a separable state $\hat\rho_T(0)=\hat\rho(0)\otimes \hat\rho_E(0)$. This holds if the system has just been put in contact with the environment or if the correlation between the system and the environment is short-lived. This is called Born approximation. 
Second, we consider the environment as a thermal reservoir, which is in a thermal state 
\begin{equation}
    \hat\rho_E(0) = \frac{e^{-\hat H_E/T}}{\Tr\left[e^{-\hat H_E/T}\right]},
\end{equation}
where $T$ is the temperature (the Boltzmann constant $k_B = 1$).
Moreover, without loss of generality, we can write the interaction Hamiltonian in the form
\begin{equation}\label{interacting_Hamiltonian}
    \hat H_I(t) = \sum_i \hat S_i\otimes \hat E_i,
\end{equation}
where $\hat S_i$ is an operator acting on $\mathcal{H}$ (it is not a spin operator) and $\hat E_i$ is an operator acting on $\mathcal{H}_E$. After making this assumption, the equation \eqref{trace_C_interacting_picture} becomes
\begin{equation}
    \begin{split}
        \frac{d\tilde\rho}{dt} =&-i\alpha\sum_i\left(\tilde S_i(t)\tilde \rho(0)\Tr_E\left[\tilde E_i(t)\tilde\rho_E(0)\right] - \tilde \rho(0)\tilde S_i(t)\Tr_E\left[\tilde\rho_E(0)\tilde E_i(t)\right]\right) \\ 
        &-\alpha^2 \int_{0}^{t}ds \Tr_E\left[\tilde H_I(t),\left[\tilde H_I(s),\tilde\rho(t)\otimes\tilde\rho_E(t)\right]\right].
    \end{split}
\end{equation}

The first term on the r.h.s. vanishes because $ \Tr_E\left[\tilde E_i(t)\tilde\rho_E(0)\right] = \left<E_i(t)\right>$ can be considered to be zero. 
It may seem strange, however, if it does not vanish, we can always redefine the environmental Hamiltonian as $\hat E'_i = \hat E_i - \left<E_i(t)\right>$. The extra term is a constant and does not modify the von Neumann equation.
The second term requires a stronger assumption: since $\alpha$ is small, the system and the environment should remain uncorrelated throughout the evolution, meaning that the timescale of the correlation should be much shorter than the timescale of the system. Thus, we can consider that the total density matrix is always separable, with the environment in the thermal state.
Nevertheless, the equation is still not markovian, since it still depends on a specific initial time $t= 0$. To add this property, we can extend the lower limit of the integration to infinity with no real change in the outcome; this is valid when the integrand disappears sufficiently fast \cite{Breuer-Petruccione}. Then, changing the integration variable to $t - s$, we arrive at 
\begin{eqnarray}\label{C_markovian}
    \frac{d\tilde \rho(t)}{dt} = -\alpha^2 \int_0^\infty ds \Tr_E\left[\tilde H_I(t),\left[\tilde H_I(t-s),\tilde\rho(t)\otimes\tilde\rho_E(t)\right]\right].
\end{eqnarray}
This is called Redfield equation \cite{Redfield}.
This is the Markov approximation, which is justified if the timescale over which the state of the system varies appreciably is large compared to the timescale over which the reservoir correlation functions decay. The approximations made before are called Born-Markov approximation \cite{Breuer-Petruccione}.

%The equation \eqref{C_markovian} can generate a negative density matrix. 
%To exclude this possibility, we consider a superoperator $\mathbb{H} A = \left[H,A\right]$, with $A$ a general operator.

Now, we perform the last approximation known as rapid wave approximation, which involves averaging over the rapid oscillating term. To do it, we consider the interaction Hamiltonian \eqref{interacting_Hamiltonian} and decompose it into eigenoperators if the the system Hamiltonian $H$.
These eigenoperators generate a complete basis of the space $\{\hat S_i(\omega)\}$ of the bounded operators acting on the Hilbert $\mathcal{H}$, they satisfy the conditions
\begin{equation}
    \left[H,\hat S_i(\omega)\right] = -\omega \hat S_i(\omega) \qquad \left[H,\hat S_i^\dagger(\omega)\right] = \omega \hat S_i^\dagger(\omega).
\end{equation}
Here, $\omega$ indicates the energy difference after the operator $\hat A_i(\omega)$ has acted.
The eigenoperators $\hat S_i(\omega)$ satisfy the relations
\begin{equation}
    \begin{split}
        e^{i\hat H_St}\hat A(\omega)e^{-i\hat H_St} = e^{-i\omega t}\\
        e^{i\hat H_St}\hat A^\dagger(\omega)e^{-i\hat H_St} = e^{i\omega t}\\
    \end{split}
\end{equation}
We can decompose the operators $S_i$ as $\hat S_i = \sum_\omega \hat S_i(\omega)$.
To apply this decomposition in \eqref{C_markovian}, we need to go back to the Schrödinger picture for the Hamiltonian acting on the proper system. Using $\tilde S_i(\omega)=e^{i\hat Ht}\hat S_i(\omega)e^{-i\hat H t}$, we obtain the Hamiltonian
\begin{equation}\label{eigen_Hamiltonian}
    \tilde H_i(t) = \sum_{i,\omega} e^{-i\hat Ht}\hat S_i(\omega) \otimes \tilde E_i (t)= \sum_{i,\omega} e^{i\hat Ht}\hat S_i^\dagger(\omega) \otimes \tilde E_i (t)
\end{equation}

We insert the equation \eqref{eigen_Hamiltonian} into \eqref{C_markovian}.
After expanding the commutators, we substitute the decomposition for $\hat S_i(\omega)$. Using the cyclic property of the trace and the fact that $\Tr[\hat H_e,\hat\rho_E(0)]=0$, we arrive at the result
\begin{equation}\label{c_substitue}
    \frac{d\tilde\rho(t)}{dt} = \sum_{\omega,\omega',i,j}e^{i(\omega-\omega')t}\Gamma_{ij}\left[\hat S_j(\omega)\tilde\rho(t),\hat S_i^\dagger(\omega')\right]+ e^{-i(\omega-\omega')t}\Gamma_{ji}^\dagger\left[\hat S_j(\omega),\tilde\rho(t)\hat S_i^\dagger(\omega')\right],
\end{equation}
where $\Gamma_{kl}(\omega)$ contains the effect of the environment and it is defined as
\begin{equation}\label{environment_coefficients}
    \Gamma_{ij}(\omega) = \int_{0}^{\infty}ds e^{i\omega s}\Tr\left[\tilde E_i^\dagger(t)\tilde E_j(t-s)\hat\rho_E(0)\right].
\end{equation}
Here, the operator $\tilde E_j(t)=e^{i\hat H_E t}\hat E_j e^{-i\hat H_E t}$ is in the interaction picture. It does not depend on time since the environment is in a stationary state and the correlation function of the environment decay extremely fast.

Now, we make the final assumption: we consider the system in the rotating wave approximation. The terms proportional to $|\omega-\omega'| \gg \alpha^2$ will oscillate much faster than the timescale of the system; thus, they do not contribute to its evolution. In the low-coupling regime, $\alpha\rightarrow 0$, we can consider that only the resonant terms, $\omega=\omega'$, contribute to the dynamics and remove all the others. Therefore, the equation \eqref{c_substitue} reduces to
\begin{eqnarray}\label{C_rotating_wave}
    \frac{d\tilde\rho(t)}{dt} = \sum_{\omega,i,j}\Gamma_{ij}\left[\hat S_j(\omega)\tilde\rho(t),\hat S_i^\dagger(\omega)\right]+\Gamma_{ji}^\dagger\left[\hat S_j(\omega),\tilde\rho(t)\hat S_i^\dagger(\omega)\right].
\end{eqnarray}

The operators $\Gamma_{ij}(\omega)$ are not necessarily Hermitian. Thus, we divide them into the Hermitian and not Hermitian parts, such that $\Gamma_{ij}(\omega) =\frac{1}{2}\gamma_{ij}(\omega)+i\pi_{ij}(\omega)$. They yields respectively
\begin{equation}
    \begin{split}
        \gamma_{ij}(\omega) &=   \Gamma_{ij}(\omega) + \Gamma_{ij}^\dagger(\omega) = \int_{-\infty}^\infty ds e^{i\omega s}\Tr\left[\left\{\tilde E_i^\dagger(t),\tilde E_j(t-s)\right\}\hat\rho_E(0)\right]\\
        \pi_{ij}(\omega) &= \frac{-i}{2}\left(\Gamma_{ij}(\omega)-\Gamma_{ij}^\dagger(\omega)\right)=\int_{-\infty}^\infty ds e^{i\omega s}\Tr\left[\left[\tilde E_i^\dagger(t),\tilde E_j(t-s)\right]\hat\rho_E(0)\right]\\
    \end{split}
\end{equation}

Inserting them into the equation \eqref{C_rotating_wave} and returning to the Schrödinger picture, we obtain
\begin{equation}\label{general_Lindblad_equation}
    \frac{d}{dt}\hat\rho = -i\left[\hat H + \hat H_{LS},\hat\rho\right] + \sum_{i,j,\omega} \gamma_{ij}(\omega) \left(\hat S_i(\omega) \hat\rho \hat S^\dagger_j(\omega) - \frac{1}{2}\left\{ \hat S^\dagger_i(\omega)\hat S_j(\omega), \hat\rho\right\} \right),
\end{equation}
where $\hat H_{LS} = \sum_{\omega,i,j} \pi_{ij}(\omega)\hat S^\dagger_i(\omega)\hat S_j(\omega)$ is called Lamb shift Hamiltonian. It adjusts the energy levels due to the interaction with the environment. The equation \eqref{general_Lindblad_equation} is the general version of the Markovian master equation. The matrix $\gamma(\omega)$ must be positive definite, although the trace preservation of the dynamics is not guaranteed.

If the matrix $\gamma(\omega)$ can be diagonalized, namely exist a diagonal matrix $D=\hat O \gamma(\omega) \hat O^\dagger$ with $\hat O$ being a unitary operator, we can write the Lindblad-Gorini-Kossakowski-Sudarshan master equation as
\begin{equation}\label{Lindbladian}
    \frac{d}{dt}\hat\rho =\mathcal{L}\left[\hat\rho\right] = -i\left[\hat H+\hat H_{LS},\hat\rho\right] + \sum_k \gamma_k(\omega) \left(\hat J_k(\omega) \hat\rho \hat J^\dagger_k(\omega) - \frac{1}{2}\left\{ \hat J^\dagger_k(\omega)\hat J_k(\omega), \hat\rho\right\} \right).
\end{equation}
The operators $\hat J_k(\omega)= \sum_i O_{ki} \hat S_{i}(\omega)$ are called jump operators, the superoperator $\mathcal{L}$ is called Lindblad superoperator and $\gamma_i(\omega)$ are the damping rates. In the limit $\gamma_k(\omega) = 0$ the von Neumann equation is recovered with the Hamiltonian $\hat H+\hat H_{LS}$.

\section{Time evolution}
In this section, we present a solution for the Lindblad master equation \cite{fujii2012}. 
First, we vectorize the density matrix: let introduce an Hilbert space with dimension $N^2$ such that a vector is $\sket{\rho} = (\rho_{00},\rho_{01},...,\rho_{NN-1},\rho_{NN})^T$ and the scalar product is $\sbraket{\phi}{\rho}= \Tr[\hat\phi^\dagger\hat\rho]$. This is known as the Fock-Liouville space \cite{Manzano}.

The follow operation can be vectorize as
\begin{equation}
    \hat A \hat\rho \hat B \rightarrow (\hat A\otimes\hat B)\sket{\rho} \qquad \hat A \hat\rho +\hat\rho\hat B \rightarrow \left(\hat A\otimes\mathbb{I} +\mathbb{I}\otimes\hat B\right)\sket{\rho},
\end{equation}
where $\mathbb{I}$ is the identity matrix, $\hat A$ and $\hat B$ are two generic operator. The symbol $\otimes$ denotes the tensorial product which generates a $N^2\times N^2$ matrix defined as
\begin{equation}
    \hat A\otimes\hat B = \begin{pmatrix}
        A_{11} \hat B & \cdots & A_{1N}\hat B\\
        \vdots & \ddots &\vdots\\
        A_{n1}\hat B& \cdots & A_{NN}\hat B\\
    \end{pmatrix}.
\end{equation}
Further details can be found in the Appendix \ref{A_vectorial_density_matrix}.

In this space the Lindblad equation \eqref{Lindbladian} becomes
\begin{equation}\label{vectorize_master_equation}
    \frac{d}{dt}\sket{\rho(t)} = \tilde{\mathcal{L}}\sket{\rho(t)},
\end{equation}
where $\tilde{\mathcal{L}}$ is the operator 
\begin{equation}
    \tilde{\mathcal{L}}=-i\left[\hat H\otimes\mathbb{I}-\mathbb{I}\otimes\hat H\right]+\sum_{k}\gamma_{k}\left[\hat J_{k}\otimes\hat J^\dagger_{k} + \hat J_{k}^\dagger \hat J_{k}\otimes\mathbb{I} +\mathbb{I}\otimes\hat J_{k}^\dagger \hat J_{k}\right].
\end{equation}
The solution to the equation \eqref{vectorize_master_equation} can be written as
\begin{equation}
    \sket{\rho(t)} = \hat U(t,0)\sket{\rho(0)},
\end{equation}
where $U(t,0)$ is the evolution operator
\begin{equation}
    \begin{split}
        \hat U(t,0) = \exp&\left\{-it\left(\hat H\otimes\mathbb{I}-\mathbb{I}\otimes\hat H\right)\phantom{\left[\frac{1}{2}\right]}\right.\\
        &+\left. t\sum_{k}\gamma_{k}\left[\hat J_{k}\otimes\hat J^\dagger_{k}-\frac{1}{2}\hat J_{k}\hat J^\dagger_{k}\otimes\mathbb{I}-\frac{1}{2}\mathbb{I}\otimes\hat J_{k}\hat J^\dagger_{k}\right]\right\}
    \end{split}
\end{equation}
The evolution operator is not unitary.

Depending on the choice of the jump operators and of the damping rates, the density matrix can converge to a stationary distribution.  
Some cases are discussed in \cite{Breuer-Petruccione} where the stationary distribution corresponds to the Boltzmann distribution. 
In other scenarios, some stationary currents between the states may persist leading to a not equilibrium stationary solution, or the system may not converge altogether.

\section{Properties of the Lindblad equation}
The Lindblad master equation satisfies some important properties.

It defines a set of dynamical maps $\phi_t\left(\hat\rho\right)= e^{\mathcal{L}t}\hat\rho(0)$ on the space of density matrices, such that
\begin{equation}
    \hat\rho(t) = \phi_t\left(\hat\rho(0)\right).
\end{equation}
These maps have the semigroup property, that is
\begin{equation}
    \phi_s\left(\phi_t\left(\hat\rho(0)\right)\right)=\phi_{t+s}\left(\hat\rho(0)\right)
\end{equation}

The Lindblad master equation is the most general form for the generator of a quantum dynamical semigroup. As a matter of fact, the Lindblad equation can also be derived from this assumption \cite{Breuer-Petruccione}.

The Lindblad master equation is invariant under the following transformations \cite{Breuer-Petruccione}:
\begin{itemize}
    \item Unitary transformation of the Lindblad operator:
        \begin{equation}
            \sqrt{\gamma_i}\hat J_i \rightarrow \sqrt{\gamma_i'}\hat J_i' = \sum_j u_{ij}\sqrt{\gamma_j}\hat J_j
        \end{equation}
        where $u_{ij}$ is an unitary matrix.
    \item Inhomogeneous transformation:
        \begin{equation}
            \begin{split}
                \hat J_i &\rightarrow \hat J'_i = \hat J_i + a_i\mathbb{I}\\
                \hat H_I &\rightarrow \hat H' = \hat H + \frac{1}{2i}\sum_j\gamma_j\left(a^*_j\hat J_j -a_j\hat J_j^\dagger\right) + b \mathbb{I}
            \end{split}
        \end{equation}
        where $a_i \in\mathbf{C}$ and $b \in \mathbf{R}$, $\mathbb{I}$ is the identity matrix.
\end{itemize}
The latter transformation allows us to always choose a traceless jump operator.

Lastly, we can prove that the dynamics \eqref{Lindblad_master_equation} conserve the trace of the density matrix. As a matter of fact, its time derivative is given by
\begin{equation}\label{Lindblad_master_equation}
    \frac{d}{dt}\Tr[\hat\rho] = \Tr\left[-i\left(\hat H\hat\rho - \hat\rho\hat H \right) +  \hat J_k \hat\rho \hat J_k^\dagger -\frac{1}{2} \left(\hat H\hat\rho + \hat\rho\hat H\right)\right] = 0,
\end{equation}
we have use the cyclic property of the trace. 
However, it is important to note that the Lindblad master equation does not conserve the purity $\Tr\left[\hat\rho^2\right]$ that decreases \cite{Manzano}. 

\chapter{Entropy production} \label{A_entropy_increasing}

In chapter \ref{C_Quantum Stochastic Walk} computing the derivative of the entropy for the quantum stochastic walk we reach the result \eqref{QSW_entropy_production}
\begin{equation}
        \dot S(\hat\rho) =  -\sum_{ij}A_{ij}\left(\Tr\left[\hat J_{ij} \hat\rho\hat J_{ij}^\dagger\ln\hat\rho\right]-\Tr\left[\hat J_{ij}^\dagger \hat J_{ij}\hat\rho\ln\hat\rho\right]\right)
\end{equation}
that it is positive ensuring the increasing of the entropy.

To prove the last statement, since $A_{ij}$is always positive, we need to prove the following inequality
\begin{equation}\label{AA_entropy_inequality}
    \Tr\left[\hat J_{ij}\hat\rho\hat J_{ij}^\dagger\ln\hat\rho\right] \leq \Tr\left[\hat J_{ij}^\dagger\hat J_{ij}\hat\rho\ln\hat\rho\right].
\end{equation}

First we diagonalize the density matrix. let be $\ket{\Lambda_i}$ the eigenstate with eigenvalue $\Lambda_i$, the density matrix can be written as
\begin{equation}
    \hat \rho = \sum_k \Lambda_k\ket{\Lambda_k}\bra{\Lambda_k}.
\end{equation}
We transform also the jump operator in this basis $\hat{\mathcal{J}}_{ij}= \hat O \hat J_{ij}\hat O^\dagger$.
The l.h.s. of the inequality \eqref{AA_entropy_inequality} becomes with some algebra can be reduces to 
\begin{equation}
    \begin{split}
        \Tr\left[\hat J_{ij} \hat\rho\hat J_{ij}^\dagger\ln\hat\rho\right]=& \sum_{kl}\Tr\left[\hat{\mathcal{J}}_{ij} \Lambda_k\ket{\Lambda_k}\bra{\Lambda_k}\hat{\mathcal{J}}_{ij}^\dagger\ln\Lambda_l \ket{\Lambda_l}\bra{\Lambda_l}\right] \\
        %&= \sum_{kl}\Lambda_k\ln\Lambda_l\Tr\left[\bra{\Lambda_l}\hat{\mathcal{J}}_{ij} \ket{\Lambda_k}\bra{\Lambda_k}\hat{\mathcal{J}}_{ij}^\dagger\ket{\Lambda_l}\right]\\
        &= \sum_{kl}\Lambda_k\ln\Lambda_l\Tr\left[|\bra{\Lambda_l}\hat{\mathcal{J}}_{ij} \ket{\Lambda_k}|^2\right]\\
        %&= \sum_{kl}\Lambda_k\ln\Lambda_l\Tr\left[x^{(ij)}_{k,l}\right]\\
        &= N \sum_{kl}\Lambda_k\ln\Lambda_lx^{(ij)}_{kl}
    \end{split}
\end{equation}
where $x^{(ij)}_{kl} = |\bra{\Lambda_l}\hat{\mathcal{J}}_{ij} \ket{\Lambda_k}|^2$ and it is symmetric respect the change $k \leftrightarrow l$. 

The r.h.s. becomes
\begin{equation}
    \begin{split}
        \Tr\left[\hat J_{ij}^\dagger \hat J_{ij}\hat\rho\ln\hat\rho\right]=& \sum_{kl}\Tr\left[\hat J_{ij}^\dagger \hat J_{ij}\Lambda_k\ket{\Lambda_k}\bra{\Lambda_k}\ln\Lambda_l\ket{\Lambda_l}\bra{\Lambda_l}\right]\\
        & = \sum_{kl}\Lambda_k\ln\Lambda_l\Tr\left[\bra{\Lambda_l}\hat J_{ij}^\dagger\hat J_{ij}\ket{\Lambda_k}\braket{\Lambda_k}{\Lambda_l}\right]\\
        &= \sum_{k}\Lambda_k\ln\Lambda_k\Tr\left[\bra{\Lambda_k}\hat J_{ij}^\dagger\hat J_{ij}\ket{\Lambda_k}\right]\\
        &= \sum_{kl}\Lambda_k\ln\Lambda_k\Tr\left[\bra{\Lambda_k}\hat J_{ij}^\dagger\ket{\Lambda_l}\bra{\Lambda_l}\hat J_{ij}\ket{\Lambda_k}\right]\\
        &= N\sum_{kl}\Lambda_k\ln\Lambda_k x^{(ij)}_{kl}.\\
    \end{split}
\end{equation}

The second braket in the trace is just a Kronecker delta. We use the completeness relation $I = \sum_l\ket{\Lambda_l}\bra{\Lambda_l}$ obtaining
\begin{equation}
    \begin{split}
        \Tr\left[\hat J_{ij}^\dagger \hat J_{ij}\hat\rho\ln\hat\rho\right]&= \sum_{kl}\Lambda_k\ln\Lambda_k\Tr\left[\bra{\Lambda_k}\hat J_{ij}^\dagger\ket{\Lambda_l}\bra{\Lambda_l}\hat J_{ij}\ket{\Lambda_k}\right]\\
        &= N\sum_{kl}\Lambda_k\ln\Lambda_k x^{(ij)}_{kl}.\\
    \end{split}
\end{equation}


The inequality \eqref{AA_entropy_inequality} reduces to 
\begin{equation}
    N \sum_{kl}\Lambda_k\ln\Lambda_lx^{(ij)}_{kl} \leq N\sum_{kl}\Lambda_k\ln\Lambda_k x^{(ij)}_{kl}.
\end{equation}

We can rearrange the term in the two sum as
\begin{equation}
    \begin{split}
        \sum_{k}\sum_{l<k}2\Lambda_k\ln\Lambda_lx^{(ij)}_{k,l} +& \sum_k\Lambda_k\ln\Lambda_k x^{(ij)}_{k,k} \leq\\ 
        \sum_{k}\sum_{l<k}&\left(\Lambda_k\ln\Lambda_k+ \Lambda_l\ln\Lambda_l\right)x^{(ij)}_{kl} + \sum_{k}\Lambda_k\ln\Lambda_k x_{kk} 
    \end{split}
\end{equation}

Therefore, to prove it it necessary that 
\begin{equation}
    2\Lambda_k\ln\Lambda_l \leq \Lambda_k\ln\Lambda_k+ \Lambda_l\ln\Lambda_l, 
\end{equation}
but it can be proved using the triangular inequality.

\begin{comment}
\section{Stationary distribution}
The Lindblad equation allows for a stationary distribution that satisfies the condition
\begin{equation}
    \mathcal{L}\hat\rho = 0.
\end{equation}

In the previous section, we assumed that the environment is in a Gibbs state. Now, consider that the damping parameter satisfies the relation
\begin{equation}\label{C_gamma}
    \gamma_{ij}(-\omega) = e^{-\beta \omega}\gamma_{ij}(\omega)
\end{equation}
which is called KMS condition \cite{Breuer-Petruccione}.
If this condition is satisfied, it can be proven that the stationary distribution is equal to the Gibbs states \cite{Breuer-Petruccione}
\begin{equation}
    \hat\rho^* = \frac{e^{-\beta \hat H}}{\Tr\left[e^{-\beta \hat H}\right]}.
\end{equation}

If the spectrum of the Hamiltonian $H = \sum_n \epsilon_n\ket{n}\bra{n}$ is not degenerate, it gives rise to a closed equation for the population 
\begin{equation}
    P(n,t) = \bra{n}\hat\rho(t)\ket{n}
\end{equation}
Thus, the dynamics decouple the diagonal and off-diagonal terms. The former are governed by the Pauli master equation
\begin{equation}\label{to_Fermi_golden_rule}
    \frac{dP(n,t)}{dt} = \sum_m \left[ W(n|m)P(m,t) - W(m|n)P(n,t)\right]
\end{equation}
with time independent transition rate
\begin{equation}
    W(n|m) = \sum_{ij} \gamma_{ij}(\epsilon_n -\epsilon_m)\bra{n}\hat J_i(t)\ket{m} \bra{m}\hat J_j(t)\ket{n}. 
\end{equation}

Using the equation \eqref{C_gamma} and considering the l.h.s. of the equation \eqref{to_Fermi_golden_rule} , we obtain the Fermi Golden rule
\begin{equation}
    W(n|m)e^{-\beta \epsilon_n} = W(m|n) e^{-\beta \epsilon_m}
\end{equation}
which is nothing other that a detailed balance condition with stationary distribution
\begin{equation}
    \hat\rho = \frac{1}{Z}e^{-\beta \hat H}
\end{equation}
with $\beta$ the inverse temperature of the environment and $Z = \Tr\left[e^{-\beta \hat H}\right]$ is the partition function.
\end{comment}


%\section{Caldeira-Leggett model}

The Caldeira-Leggett model was proposed in 1983 to reproduce the Brownian motion via a quantum process.
It study the dynamics of a particle in contact with a thermal bath made by a set of harmonic oscillator.
However, the model present the Markovian propery only in the high temperature weak coupling limit and a not Markovian one in otherwise.

The free Hamiltonian of the particle is
\begin{equation}
    H_S = \frac{\hat p^2}{2m} + V(\hat q),
\end{equation}
where $\hat p$ is the momentum operator and $V(\hat q)$ is the potential.

The particle is in contact with the bath composed by N harmonic oscillator. The Hamiltonian is
\begin{equation}
    H_E = \sum\Omega_n(\hat b_n^\dagger\hat b_n+\frac{1}{2})= \sum_n \left(\frac{p_n^2}{2m_n}+ \frac{1}{2}m_n\Omega_n^2q_n^2\right).
\end{equation}
where $\hat b_n^\dagger$ and $\hat b_n$ are respectively the creation and annihilation operators of the bath, and $p_n$ and $q_n$ the momentum and position operator of the oscillators of bath.

The interaction Hamiltonian is given by
\begin{equation}
    H_I = \hat q\sum_n k_n \hat q_n = - \hat B \hat q = \hat q \sum_n k_n\frac{1}{\sqrt{2m_n\Omega_n}}\left(\hat b+ \hat b^\dagger\right)
\end{equation}
where $\hat B$ is the bath operator.

We need to add a counterterm 
\begin{equation}
    H_C = \sum_n \frac{\hat q^2}{2m_n\Omega_n^2}. 
\end{equation}

Starting from equation \eqref{C_markovian} in the Schrödinger pictures
\begin{equation}
    \frac{d \rho(t)}{dt} = i\left[\hat H_S+ \hat H_C, \hat\rho(t)\right] -\int_0^\infty ds \Tr_E\left[\hat H_I(t),\left[\hat H_I(t-s),\hat\rho(t)\otimes\hat\rho_E(t)\right]\right],
\end{equation}

We can rewrite it using the correlation relations of the bath
\begin{equation}
    D(s) =i \left<\left[\hat B, \hat B^\dagger(t-s)\right]\right> \qquad D'(s) = \left<\left\{\hat B, \hat B^\dagger(t-s)\right\}\right>
\end{equation}
into the formula 
\begin{equation}\label{CL_master_eq}
    \frac{d \rho(t)}{dt} = -i\left[\hat H_S+ \hat H_C, \hat\rho(t)\right] -\int_0^\infty ds \frac{i}{2} D(s)\left[\hat q, \left\{\hat q(-s), \hat \rho(t)\right\}\right] - \frac{1}{2} D'(s)\left[\hat q, \left[\hat q(-s), \hat\rho(t)\right]\right]  
\end{equation}

Using the spectral density
\begin{equation}
    J(\Omega) = \sum_n \frac{k_n^2}{2m_n\Omega}\delta(\Omega-\Omega_n)
\end{equation}
we can rewrite the correlation as
\begin{equation}
    D(s) = 2\int_0^\infty d\Omega J(\Omega) \sin\Omega s \qquad D'(s) = 2\int_0^\infty d\Omega J(\Omega) \coth\left(\frac{\Omega}{2k_BT}\right)\cos\Omega s.
\end{equation}

We can simplify further the equation \eqref{CL_master_eq} looking at the $\hat q(-s)$, indeed, it can be written as 
\begin{equation}
    \hat q(-s) = e^{-i\hat H_S s} \hat q e^{i\hat H_S s} \approx \hat q- i\left[\hat H_S, \hat q\right]s = \hat q - \frac{\hat p}{m}s 
\end{equation}

Inserting it into \eqref{CL_master_eq} we obtain
\begin{equation}
    \begin{split}
        \frac{d \rho(t)}{dt} = & -i\left[\hat H_S+ \hat H_C, \hat\rho(t)\right] -\frac{i}{2}\int\limits_0^\infty ds  D(s)\left[\hat q, \left\{\hat q, \hat \rho(t)\right\}\right]\\
        &+ \frac{i}{2m}\int_0^\infty ds \, sD(s)\left[\hat q, \left\{\hat p, \hat \rho(t)\right\}\right] - \frac{1}{2}\int_0^\infty ds D'(s)\left[\hat q, \left[\hat q, \hat\rho(t)\right]\right]\\
        & +\frac{1}{2m}\int_0^\infty ds \; sD'(s)\left[\hat q, \left[\hat p, \hat\rho(t)\right]\right].\\
    \end{split}  
\end{equation}

Then, we solve the integral in the high temperature limit: the first term give us the exactly the counter term  $-i\left[\hat H_C, \hat \rho\right]$, the second becomes $-\frac{i\gamma}{2}\left[\hat q, \left\{\hat p,\hat \rho(t)\right\}\right]$, the third becomes $2m \gamma k_BT\left[\hat q,\left[\hat q,\hat\rho(t)\right]\right]$ and the last give us $\frac{2\gamma k_BT}{\Omega}\left[\hat q, \left[\hat p, \hat \rho\right]\right]$ which is negligible in the limit $\Omega \rightarrow \infty$ that we are dealing with.

We have reached the Caldeira-Leggett master equation
\begin{equation} \label{Caldeira-Leggett_master_equation}
    \frac{d\hat\rho}{dt} = -i\left[\hat H_S, \hat\rho(t)\right]-\frac{i\gamma}{2}\left[\hat q, \left\{\hat p,\hat \rho(t)\right\}\right] - 2m \gamma k_BT\left[\hat q,\left[\hat q,\hat\rho(t)\right]\right]. 
\end{equation}
The Caldeira-Leggett master equation reminds of the Langevin equation where the second term is the dissipative one and the third describes the fluctuations.
