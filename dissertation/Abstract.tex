\begin{abstract}
    The complex network framework has been successfully applied across various fields, from biochemistry to artificial intelligence. 
    This dissertation studies an information-theoretic approach to complex networks based on quantum information.
    
    We define a network entropy using the von Neumann entropy, where the density matrix is proportional to the exponential of the Laplacian matrix, scaled by a parameter $\beta$.
    This formulation provides a novel perspective on network dynamics, enabling the characterization of structural complexity.

    Furthermore, we establish a connection between this density matrix approach and the evolution of quantum walks in the presence of thermal noise. 
    Specifically, we show that the stationary distribution of a quantum walk on a network in contact with a thermal bath at temperature $T = 1/\beta$, leads to the same formulation. The interactions with the bath are models as Markovian and the system is analyzed through the Lindblad master equation. The temperature $T$ regulates the spatial correlation between the nodes: at lower temperatures, long-range correlations become more significant, influencing the system's relaxation dynamics. However, the quantum walk requires that the Laplacian is hermitian. Thus, the analogy holds only for network that satisfies the detailed balance condition.
    
    Finally, we introduce Kullback-Leibler and Jensen-Shannon divergences based on network entropy, which define a distance between networks according to their relaxation behavior. However, these measures rely solely on the network spectrum and thus cannot distinguish between different networks with the same spectral properties. 

    These findings provide a deeper understanding of network complexity and open new avenues for applying quantum information tools to the study of complex systems
\end{abstract}