\chapter{Quantum Network Master Equation}

In chapter \ref{C_Density_Matrix} we have introduce the concept of density matrix for a network, based on the communicability matrix. This quantity considers the correlation between the nodes created by the random walk dynamics. 

In this chapter we try to unify the two concept taken from the quantum realm: the quantum random walk and the density matrix.
In fact, we considering a quantum walk process exposed to thermal noise we obtain a stationaty distribution which coincides with the network's density matrix. 
The interaction between the quantum system and the thermal noise are considered Markovian, thus, it is studied with the Lindblad master equation.

\section{Quantum Stochastic Random Walk}\label{C_Quantum Stochastic Walk}

One of the early approaches to a Open Quantum Walk on network was proposed by Whitfield, Rodr\'iguez-Rosario and Aspuru-Guzik. \cite{QSW}. They defined a quantum walk on network in contact with a thermal bath with the dynamics described by a Lindblad master equation. In this framework, the jump operators are proportional to the adjacency  matrix $A_{ij}$ of the network. The thermal bath introduces noise into the dynamics, causing it which to deviate from the  Von Neumann equation \eqref{Von Neumann equation}. The system dissipation is reminiscent of the classic random walk.

Let us consider a quantum walk on a network $G(N,M)$, the system in contact with a thermal bath that randomly interact with modifying the dynamic of the quantum particle.
Let us introduce a Hilbert space $\mathcal{H}$ with an orthonormal basis $\{\ket{i}\}_{i<N}$, where each element $\ket{i}$ corresponds to the node $i$, satisfying $\braket{i}{j}=\delta_{ij}$. The system is described by a density matrix $\hat \rho$ whose evolution follows the Lindblad master equation \eqref{Lindblad_master_equation}.
The Laplacian operator $\hat L$, defined as in equation \eqref{Laplacian_operator}, serves as the Hamiltonian $\hat H$, while the jump operators $\{\hat J_k\}_{k<M}$ represent the thermal jumps between two node linked together. For convenience, we will denote the jump operator with two indeces referring to the starting node $j$ and the ending node $i$ of the jump: therefore, the jumps operator are $\hat J_{ij} = \ket{i}\bra{j}$. The damping rates are $\gamma_{ij} =A_{ij}/d_i$, like the transition rates in the classical random walk \eqref{transition_rates}.
The master equation can be expressed as follows:
\begin{equation}\label{stochastic_lindblad_master}
    \frac{d}{dt}\hat \rho = -\frac{i}{2}\left[\hat L,\hat\rho\right] + \sum_{ij}\gamma_{ij}\left[\hat J_{ij} \hat\rho\hat J_{ij}^\dagger -\frac{1}{2} \left\{ \hat J_{ij}^\dagger \hat J_{ij}, \hat\rho\right\}\right],
\end{equation}
where $[\cdot,\cdot]$ and $\{\cdot,\cdot\}$ denote respectively the commutator and the anticommutator.
The equation \eqref{stochastic_lindblad_master} is composed by two distinct terms. The first term, also called coherent dynamics,  is given by
$\mathcal{L}^{qm}\left[\hat\rho(t)\right] = -i\left[H,\hat\rho\right]$ is equal to the quantum walk dynamics. In contrast, the second term, denoted as $\mathcal{L}^{cl}\left[\hat\rho(t)\right] = \sum_i \gamma_i \left(\hat J_i \hat\rho \hat J^\dagger_i - \frac{1}{2}\left\{ \hat J^\dagger_i\hat J_i, \hat\rho\right\} \right)$, called decoherent dynamics, encodes the dissipation. 
When $\gamma_{ij} = 0$ we recover the Von Neumann equation for the quantum walk \eqref{Von Neumann equation}. 
$\mathcal{L}\left[\hat\rho(t)\right]$ act as a superoperator in the space of the density matrix.

In the Fock-Liouville space, the quantum system evolves according the equation
\begin{equation}
    \sket{\rho(t)}  = U(t,0) \sket{\rho(0)}
\end{equation} 
where the evolution operator is defined as \cite{Domino}
\begin{equation}
    \begin{split}
        \hat U(t,0) = \exp&\left\{-it\left(\hat H\otimes\mathbb{I}-\mathbb{I}\otimes\hat H\right)\right.\\
        &+\left. t\sum_{ij}\gamma_{ij}\left[ \hat J_{ij}\otimes\hat J^\dagger_{ij}-\frac{1}{2}\hat J_{ij}\hat J^\dagger_{ij}\otimes\mathbb{I}-\frac{1}{2}\mathbb{I}\otimes\hat J_{ij}\hat J^\dagger_{ij}\right]\right\}.
    \end{split}
\end{equation}

The master equation \eqref{stochastic_lindblad_master} contains both the quantum and classical aspects of a random walk over a network. As a matter of fact, the classical random walk behavior emerges when we consider the evolution of the diagonal elements of the density matrix under the dissipative part alone. Let $\rho = \ket{k}\bra{k}$ represent the density matrix of a system localized at node $k$. Its evolution is given by
\begin{equation}
    \begin{split}
        \mathcal{L}^{cl}\ket{k}\bra{k} &= \sum_{ij}\gamma_{ij}\left[\hat J_{ij} \ket{k}\bra{k}\hat J_{ij}^\dagger -\frac{1}{2} \left\{ \hat J_{ij}^\dagger \hat J_{ij}, \ket{k}\bra{k}\right\}\right]\\
        %&= \sum_{ij}\left[\sqrt{A_{ij}}\ket{i}\braket{j}{a}\braket{a}{j}\bra{i}\sqrt{A_{ij}} -\frac{1}{2}\ket{i}\sqrt{A_{ia}}\braket{j}{j}\sqrt{A_{ia}}\braket{i}{a}\bra{a}\right.\\ 
        %&\quad \left. - \frac{1}{2} \ket{a}\braket{a}{i}\sqrt{A_{ia}}\braket{j}{j}\sqrt{A_{ia}}\bra{i}\right]\\
        &= \sum_{i}\left[\gamma_{ik} \ket{i}\bra{i} -\gamma_{ik}\ket{k}\bra{k}\right]\\
        &=\sum_{i} \left(\gamma_{ki} - \gamma_{ki}\delta_{ki}\right) \ket{i}\bra{i} = -\sum_i L_{ki} \ket{i}\bra{i}.
    \end{split} 
\end{equation} 
where $d_i$ is the degree of the node $i$.
This expression recovers the dynamics of the classical random walk over the network.
Next, considering the off-diagonal terms, their evolution is described by
\begin{equation}
    \begin{split}
        \mathcal{L}^{cl}\ket{k}\bra{l} &= \sum_{ij}\gamma_{ij}\left[\hat J_{ij} \ket{k}\bra{l}\hat J_{ij}^\dagger -\frac{1}{2} \left\{ \hat J_{ij}^\dagger \hat J_{ij}, \ket{k}\bra{l}\right\}\right]\\
        %&= \sum_{ij}\left[\sqrt{A_{ij}}\ket{i}\braket{j}{k}\braket{l}{j}\bra{i}\sqrt{A_{ij}} -\frac{1}{2}\ket{i}\sqrt{A_{ik}}\braket{j}{j}\sqrt{A_{ik}}\braket{i}{k}\bra{l}\right.\\ 
        %&\quad\left. - \frac{1}{2} \ket{k}\braket{l}{i}\sqrt{A_{ik}}\braket{j}{j}\sqrt{A_{ik}}\bra{i}\right]\\
        &= \sum_{j}\left[-\frac{1}{2} \gamma_{jk}\ket{k}\bra{l} - \frac{1}{2} \gamma_{jl}\ket{k}\bra{l}\right]\\
        &= -\ket{k}\bra{l}.
    \end{split} 
\end{equation}

The operator $\mathcal{L}^{cl}$ does not mix the diagonal terms with the off-diagonal ones, allowing us to separate the superoperator into two blocks: one for the diagonal element and the another for the off-diagonal ones.
Thus, the superoperator $\mathcal{L}^{cl}$ has a diagonal form with spectrum given by $\sigma^{cl} = -(\lambda_1,...,\lambda_N,1,...,1))$, where $\lambda_i$ are the eigenvalue of the Laplacian matrix \cite{Bruderer_Plenio}.
If the network satisfies the detailed balance condition \eqref{detail_condition}, the Laplacian matrix has a zero eigenvalue. Therefore, the superoperator $\mathcal{L}^{cl}$ will also have a zero eigenvalue indicating the presence of a stationary distribution.

\begin{comment}
    \begin{equation}
    \begin{split}
    \mathcal{L}^{qm}\ket{k}\bra{l} =& -i\hat L\ket{k}\bra{l} + i \ket{k}\bra{l}\hat L\\
    &=\frac{i}{2}\sum_{ij} - L_{ij}\ket{i}\braket{j}{k}\bra{l} + \ket{k}\braket{l}{i}\bra{j}\\
    &=\frac{1}{2}\sum_i -L_{ik} \ket{i}\bra{l} + \sum_i L_{li}\ket{k}\bra{i}\\
\end{split}
\end{equation}

Instead the diagonal terms
\begin{equation}
\begin{split}
\mathcal{L}^{qm}\ket{l}\bra{l} =& -i\hat L\ket{l}\bra{l} + i \ket{l}\bra{l}\hat L\\
&=\frac{i}{2}\sum_{ij} - L_{ij}\ket{i}\braket{j}{l}\bra{l} + \ket{l}\braket{l}{i}\bra{j}\\
&=\frac{1}{2}\sum_i -L_{il} \ket{i}\bra{l} + \sum_i L_{li}\ket{l}\bra{i} = 0\\
\end{split}
\end{equation}
\end{comment}

\subsection{Stationary distribution}

Following the proof of the stationary distribution in Chapter \ref{C_Lindblad}, we can find the stationary density matrix $\hat\rho^*$ for the quantum stochastic walk.
We will consider only the dissipative dynamics, which decouples the diagonal and off-diagonal terms. The evolution of the diagonal elements is described by:
\begin{equation}
    \frac{d}{dt}\rho_{ii} = \sum_j\left[\gamma_{ij}\rho_{jj}(t) - \gamma_{ji}\rho_{ii}(t)\right],
\end{equation}

The stationary distribution must satisfy the detail balance, namely
\begin{equation}
    \gamma_{ij}\rho_{jj}(t) = \gamma_{ji}\rho_{ii}.
\end{equation}
Because the damping rates fort this system are symmetric, the diagonal entries must be equal. 
Instead, considering the vector $\sket{\rho}$, the block corresponding to the off-diagonal part of $\mathcal{L}$ is already eigenstate with eigenvalue $1$. Thus, the off-diagonal terms must be equal to zero.
The stationary density matrix can then be expressed as
\begin{equation}\label{QSW_stationary_distribution}
    \hat\rho^* = \frac{1}{N}\begin{pmatrix}
        1&&0\\
        &\ddots&\\
        0&&1\\
    \end{pmatrix}.
\end{equation}
As discussed in chapter \ref{C_entropy_production}, the stationary density matrix has maximal Von Neumann entropy 
\begin{equation}
    S\left(\hat\rho^*\right) = \ln N.
\end{equation}

The density matrix \eqref{QSW_stationary_distribution} commutes with the Laplacian matrix, confirming that it is indeed the stationary density matrix for the dynamics described by \eqref{stochastic_lindblad_master}.

\begin{comment}
    Instead, adding also the coherent part, we need to go in the basis $\{\ket{\lambda}\}$ eigenvector of $\hat L$. 
    The decoherent part becomes
    \begin{equation}
    \begin{split}
    \mathcal{L}^{cl}\ket{\lambda}\bra{\lambda} &= \sum_{ij}\gamma_{ij}\left[\hat J_{ij} \ket{\lambda}\bra{\lambda}\hat J_{ij}^\dagger -\frac{1}{2} \left\{ \hat J_{ij}^\dagger \hat J_{ij}, \ket{\lambda}\bra{\lambda}\right\}\right]\\
        &= \sum_{ij}\gamma_{ij}\left[\ket{i}\braket{j}{\lambda}\braket{\lambda}{j}\bra{i} -\frac{1}{2}\ket{j}\braket{i}{i}\braket{j}{\lambda}\bra{\lambda} - \frac{1}{2}\ket{\lambda}\braket{\lambda}{j}\braket{i}{i}\bra{j} \right]\\
        & = \sum_{ij}\gamma_{ij}\left[|\braket{j}{\lambda}|^2\ket{i}\bra{i} -\frac{1}{2}\ket{j}\braket{j}{\lambda}\bra{\lambda} - \frac{1}{2}\ket{\lambda}\braket{\lambda}{j}\bra{j} \right]\\
        &= \sum_{ij}\left[\pi_{ij}\rho_j\ket{i}\bra{i} -\frac{\pi_{ij}}{2}\braket{j}{\lambda}\ket{j}\bra{\lambda} - \frac{\pi_{ij}}{2}\braket{\lambda}{j}\ket{\lambda}\bra{j} \right]
    \end{split}
\end{equation}
\end{comment}


\section{Quantum Network Master Equation}\label{C_quantum_network_master_equation}

The previous description for the noise in the quantum walk lacks a parameter that, in thermodynamics, correspond to the temperature. 
Thus, we propose another framework: instead of considering the jump operator as a transition between the nodes, we consider the thermal bath interaction in the energy states.
Let us retake the standard Lindblad equation \eqref{Lindbladian} with Hamiltonian $\hat H = \hat L$.
We introduce the basis $\{\ket{\lambda}\}_{\lambda}$ such that $\hat H = \sum_\lambda \lambda\ket{\lambda}\bra{\lambda} = \hat L$ is diagonal (the network must satisfy the detailed balance condition \eqref{detail_condition}).
We define the jump operator $\hat J_{\lambda\mu} = \ket{\lambda}\bra{\mu}$ as the jumps from the energy state $\ket{\mu}$ to the energy state $\ket{\lambda}$ obtaining the master equation
\begin{equation}\label{Lindblad_energy_jump}
    \frac{d}{dt}\hat\rho = -i\left[\hat L,\hat\rho\right] + \sum_{\lambda\mu} \gamma_{\lambda\mu} \left(\hat J_{\lambda\mu} \hat\rho \hat J^\dagger_{\lambda\mu} - \frac{1}{2}\left\{ \hat J^\dagger_{\lambda\mu}\hat J_{\lambda\mu}, \hat\rho\right\} \right),
\end{equation}
where the coefficients $\gamma_{\lambda\mu}$ indicate the probability of taking their respective jumps.

We assume that the dynamics will tend to a stationary distribution in the form of
\begin{equation}\label{pretended_stationary_distribution}
    \hat \rho^* = \frac{e^{-\beta\hat L}}{Z},
\end{equation}
with $Z = \Tr\left[e^{-\beta\hat L}\right]$ being the partition function.
The master equation for the stationary distribution \eqref{pretended_stationary_distribution} reduces to
\begin{equation}\label{cancel_master_equation}
    0 = -i\left[\hat H, \frac{e^{-\beta\hat L}}{Z}\right] + \sum_{\lambda\mu} \gamma_{\lambda\mu} \left(\hat J_{\lambda\mu}  \frac{e^{-\beta\hat L}}{Z} \hat J^\dagger_{\lambda\mu} - \frac{1}{2}\left\{ \hat J^\dagger_{\lambda\mu}\hat J_{\lambda\mu},  \frac{e^{-\beta\hat H}}{Z}\right\} \right).
\end{equation}
The first term on the r.h.s. vanishes since the commutator is zero.

Now, we analyze the dissipation terms independently.
The first one can be written as 
\begin{equation}\label{BCS_first_term}
        \sum_{\lambda\mu} \gamma_{\lambda\mu} \ket{\lambda}\bra{\mu}\frac{e^{-\beta\hat L}}{Z} \ket{\mu}\bra{\lambda} = 
        \sum_{\lambda\mu} \gamma_{\lambda\mu} \frac{e^{-\beta \epsilon_\mu}}{Z} \ket{\lambda}\bra{\lambda}.
\end{equation}
While the second becomes
\begin{equation}\label{BCS_second_term}
    \sum_{\lambda\mu} \gamma_{\lambda\mu}\left[\frac{1}{2}\ket{\mu}\braket{\lambda}{\lambda}\bra{\mu}\frac{e^{-\beta\hat L}}{Z} +\frac{1}{2}\frac{e^{-\beta\hat L}}{Z}\ket{\mu}\braket{\lambda}{\lambda}\bra{\mu}\right]= \sum_{\lambda\mu} \gamma_{\lambda\mu}\left[\frac{e^{-\beta \epsilon_\mu}}{Z}\ket{\mu}\bra{\mu}\right].
\end{equation}
Therefore, inserting the equations \eqref{BCS_first_term} and \eqref{BCS_second_term} into the master equation \eqref{cancel_master_equation}, we obtain the condition
\begin{equation}\label{Kirchhoff_law_energy}
    \sum_{\lambda\mu}\left[\gamma_{\lambda\mu}\frac{e^{-\beta \mu}}{Z} - \gamma_{\mu\lambda}\frac{e^{-\beta \lambda}}{Z}\right]\ket{\lambda}\bra{\lambda} = 0.
\end{equation}
This is the Kirchhoff's current law which says that the sum of all the currents must vanish. The system should satisfy this condition in order to have the Boltzmann distribution.
However, for a fixed $\beta$, there are several possible choices for the coefficients $\gamma_{\lambda\mu}$ such that equation \eqref{Kirchhoff_law_energy} holds.
Each different choice generates a different path to reach the stationary distribution \eqref{pretended_stationary_distribution}.
We are looking for a solution that is explicitly depends on the parameter $\beta$. The simplest choice is that the damping rates satisfies teh detailed balance condition 
%We assume that the system must satisfy the maximal entropy production principle: the system should relax to the stationary distribution following the trajectory that has maximal entropy production. The principle is satisfied when the Kirchhoff law \eqref{Kirchhoff_law_energy} reduces to the detail balance condition
\begin{equation}
    \gamma_{\lambda\mu}\frac{e^{-\beta \mu}}{Z} - \gamma_{\mu\lambda}\frac{e^{-\beta \lambda}}{Z} = 0,
\end{equation}
which has the solution
\begin{equation}\label{gamma_detailed_balance}
    \begin{split}
        \gamma_{\lambda\mu} = c \;e^{-\frac{\beta}{2}\left(\lambda - \mu\right)}\\
        \gamma_{\mu\lambda} = c \; e^{\frac{\beta}{2}\left(\lambda - \mu\right)}.
    \end{split}
\end{equation}
The solution is not unique;  there exist a set of possible solutions which differ by a constant $c$.
%with $Z_\gamma = \sum_{\lambda\gamma}e^{-\frac{\beta}{2}\left(\epsilon_\lambda - \epsilon_\mu\right)}$ is the renormalization factor.

The von Neumann entropy measure the mi

Taking the limit $\beta \rightarrow \infty$, that is $T  \rightarrow 0$, the transition rates tend to
\begin{equation}
    \gamma_{\lambda\mu} \rightarrow \left\{\begin{aligned}
        0 \qquad \lambda > \mu\\
        1 \qquad \lambda = \mu\\
        \infty \qquad  \lambda < \mu \\
    \end{aligned}\right. . 
\end{equation}
The transitions from lower to higher energy state are suppressed, while the opposite one are extremely favorite. Thus, the system is led to the zero energy state that is the stationary state
\begin{equation}\label{beta_inf_stationary_distribution}
    \hat\rho^* = \ket{\lambda = 0}\bra{\lambda = 0}.
\end{equation} 
It is a pure state therefore the von Neumann entropy vanishes.

In the opposite limit $\beta \rightarrow 0$, that is $T \rightarrow \infty$, the transition rates become
\begin{equation}
    \gamma_{\lambda\mu} \rightarrow 1.
\end{equation}
We have the opposite effect, the particle can jump across the different energy state with uniform probability. Thus, the stationary distribution is the maximal entropy state, i.e. the uniform distribution
\begin{equation}
    \hat\rho^* = \frac{1}{N}\begin{pmatrix}
        1&&0\\
        &\ddots&\\
        0&&1\\
    \end{pmatrix}.
\end{equation}
It is a maximal entropy state with $S = \ln N$.

\subsection{Return to node's basis}
We go back to the position basis $\{\ket{i}\}_{i<N}$, where $\ket{i}$ indicates the particle in the node $i$.  
The jump operators can be expressed as
\begin{align}
    \hat J_{\lambda\mu} = \sum_{ij} \braket{i}{\lambda}\braket{\mu}{j}\hat J_{ij}\\
    \hat J_{\lambda\mu}^\dagger = \sum_{ij} \braket{j}{\mu}\braket{\lambda}{i}\hat J_{ij}^\dagger
\end{align}
where $\hat J_{ij} = \ket{i}\bra{j}$.
Thus, substituting these expressions into equation into the equation \eqref{Lindblad_energy_jump} we obtain
\begin{equation}\label{quantum_network_position}
    \frac{d}{dt}\hat\rho = -i\left[\hat H,\hat\rho\right] +\sum_{ijkl} \gamma_{ij;kl} \left(\hat J_{ij}\hat\rho \hat J_{kl}^\dagger - \frac{1}{2}\left\{ \hat J_{kl}^\dagger\hat J_{ij}, \hat\rho\right\} \right),
\end{equation}
where the damping coefficient are define as
\begin{equation}\label{gamma_position}
    \gamma_{ij;kl}= \sum_{\lambda\mu}\gamma_{\lambda\mu}\braket{i}{\lambda}\braket{\lambda}{k}\braket{l}{\mu}\braket{\mu}{j}.
\end{equation}
Unlike the energy basis, the damping rates in the position basis are no longer diagonal, making the master equation \eqref{quantum_network_position} appear more complex as in the the equation\eqref{general_Lindblad_equation}. Using the detailed balance condition \eqref{gamma_detailed_balance} in \eqref{gamma_position}, we rewrite the damping rates as:.
\begin{equation}
    \gamma_{ij;kl}= \sum_{\lambda\mu}e^{-\frac{\beta}{2}\left(\epsilon_\lambda - \epsilon_\mu\right)}\braket{i}{\lambda}\braket{\lambda}{k}\braket{l}{\mu}\braket{\mu}{j}.
\end{equation}
\begin{comment}
    The master equation for the stationary distribution \eqref{pretended_stationary_distribution} in position space is 
    \begin{equation}
    0 = \sum_{ijkl} \gamma_{ij;kl} \left(\hat J_{ij}  \frac{e^{-\beta\hat H}}{Z} \hat J^\dagger_{kl} - \frac{1}{2}\left\{ \hat J^\dagger_{ij}\hat J_{kl},  \frac{e^{-\beta\hat H}}{Z}\right\} \right).
\end{equation}
The first term reduces to
\begin{equation}
    \sum_{ijkl}\gamma_{ij;kl} \ket{i}\bra{j} \frac{e^{-\beta\hat H}}{Z} \ket{l}\bra{k} = \sum_{ijkl}\gamma_{ij;kl} \frac{e^{-\beta H_{jl}}}{Z}\ket{i}\bra{k}.
\end{equation}
The anticommutator reduces to
\begin{equation}
    \begin{split}
        \sum_{ijkl}  \frac{\gamma_{ij;kl}}{2}\left(\ket{j}\braket{i}{k}\bra{l} \frac{e^{-\beta\hat H}}{Z}+ \frac{e^{-\beta\hat H}}{Z}\ket{j}\braket{i}{k}\bra{l}\right) \\ 
        = \sum_{ijlm} \left(\frac{\gamma_{ij;il}}{2}\frac{e^{-\beta H_{lm}}}{Z} + \frac{\gamma_{il;im}}{2}\frac{e^{-\beta H_{jl}}}{Z}\right)\ket{j}\bra{m}
    \end{split}
\end{equation}
We have applied some algebra and exchange dummy indices.
Thus, combining all together we obtain
\begin{equation}
    \begin{split}
        \sum_{ijkl}\left(\gamma_{ij;kl} \frac{e^{-\beta H_{jl}}}{Z} - \frac{\gamma_{ji;jl}}{2} \frac{e^{-\beta H_{lk}}}{Z} - \frac{\gamma_{jl;jk}}{2}\frac{e^{-\beta H_{il}}}{Z}\right)\ket{i}\bra{k} = 0\\
        \sum_{ijkl}\left(\gamma_{(ik)\leftarrow(jl)} \frac{e^{-\beta H_{jl}}}{Z} - \frac{\gamma_{(jj)\leftarrow(il)}}{2} \frac{e^{-\beta H_{lk}}}{Z} - \frac{\gamma_{(jj)\leftarrow(lk)}}{2}\frac{e^{-\beta H_{il}}}{Z}\right)\ket{i}\bra{k} = 0\\
        %\sum_{ijl}\left(\gamma_{ij;il} \frac{e^{-\beta H_{jl}}}{Z} - \frac{\gamma_{ji;ji}}{2} \frac{e^{-\beta H_{li}}}{Z} - \frac{\gamma_{jl;ji}}{2}\frac{e^{-\beta H_{il}}}{Z}\right)\ket{i}\bra{i} = 0
    \end{split}
\end{equation}
It can be interpreted as a Kirchhoff law in the link. Let suppose that the transition of the particle through the link is not instantaneous, then at each movement the particle chooses one link to travel. The probability to be in a specific link $(i,j)$ is given by $\frac{e^{-\beta H_{ij}}}{Z}$. In order to have a stationary distribution the dumping rate must satisfy the Kirchhoff law over the links distribution.
\end{comment}

In the high temperature limit, $\beta \rightarrow 0$, the damping rates are
\begin{equation}
    \gamma_{ij;kl}= \sum_{\lambda\mu} 1 \braket{i}{\lambda}\braket{\lambda}{k}\braket{l}{\mu}\braket{\mu}{j}.
\end{equation}
Using the completeness relation, we obtain
\begin{equation}
    \gamma_{ij;kl} = 1 \delta_{ik}\delta_{jl},
\end{equation}
where $\delta_{ik}$ is the Kronecker delta. Thus, the particle travel always through the same link.
In this case the position quantum network master equation \eqref{quantum_network_position} acquires a “symmetric" form
\begin{equation}
    \frac{d}{dt}\hat\rho = -i\left[\hat H,\hat\rho\right] +\sum_{ij}\left(\hat J_{ij}\hat\rho \hat J_{ij}^\dagger - \frac{1}{2}\left\{ \hat J_{ij}^\dagger\hat J_{ij}, \hat\rho\right\} \right).
\end{equation}
Since the relation \eqref{gamma_detailed_balance} can be modify by a constant factor, we recover the master equation for the Quantum Stochastic Walk \eqref{stochastic_lindblad_master}.
The stationary distribution is 
\begin{equation}
    \hat\rho^* = \frac{1}{N}\begin{pmatrix}
        1&&0\\
        &\ddots&\\
        0&&1\\
    \end{pmatrix}.
\end{equation}

\begin{comment}
In contrast, in the low temperature limit, $\beta \rightarrow \infty$, the damping coefficient are
\begin{equation}
    \begin{split}
        \gamma_{ij;kl}&= \sum_{\lambda\mu}e^{-\frac{\beta}{2}\left(\epsilon_\lambda - \epsilon_\mu\right)}\braket{i}{\lambda}\braket{\lambda}{k}\braket{l}{\mu}\braket{\mu}{j}\Theta(\lambda-\mu)\\
        & + \sum_{\lambda\mu} 1\braket{i}{\lambda}\braket{\lambda}{k}\braket{l}{\mu}\braket{\mu}{j}\delta_{\lambda\mu}\\
        & + \sum_{\lambda\mu} e^{-\frac{\beta}{2}\left(\epsilon_\lambda - \epsilon_\mu\right)}\braket{i}{\lambda}\braket{\lambda}{k}\braket{l}{\mu}\braket{\mu}{j}\Theta(\mu-\lambda)
    \end{split}
\end{equation}
The first term cancel out,  and the second is a sum of Kronecker delta. Thus, it reduces to
\begin{equation}\label{gamma_T=0}
    \begin{split}
        \gamma_{ij;kl}&= \sum_{\lambda}\braket{i}{\lambda}\braket{\lambda}{k}\braket{l}{\lambda}\braket{\lambda}{j}+ \sum_{\mu>\lambda}\infty\braket{i}{\lambda}\braket{\lambda}{k}\braket{l}{\mu}\braket{\mu}{j} \rightarrow \infty
    \end{split}
\end{equation}
\end{comment}
In contrast, in the low temperature limit, $\beta \rightarrow \infty$, the stationary distribution \eqref{beta_inf_stationary_distribution} for the node reduces to
\begin{equation}
    \hat\rho^* = \sum_{ij} \braket{i}{\lambda = 0}\braket{\lambda = 0}{j} \ket{i}\bra{j} = \sqrt{\rho^*_i\rho^*_j} \ket{i}\bra{j}
\end{equation}
The von Neumann entropy vanishes, indicating a pure state. 

%Collegare a overdamped case 


\begin{comment}
    \newpage
    The same distribution can be obtain in position space. As a matter of fact, the master equation \eqref{stochastic_lindblad_master} for the distribution \eqref{pretended_stationary_distribution}. Let assume that it is the stationary distribution, thus
    \begin{equation}
    0 = \sum_{\lambda\mu} \gamma_{ij} \left(\hat J_{ij}  \frac{e^{-\beta\hat H}}{Z} \hat J^\dagger_{ij} - \frac{1}{2}\left\{ \hat J^\dagger_{ij}\hat J_{ij},  \frac{e^{-\beta\hat H}}{Z}\right\} \right).
\end{equation}

the first term can be written as
\begin{equation}
\sum_{ij} \gamma_{ij} \ket{i}\bra{j}  \frac{e^{-\beta\hat H}}{Z} \ket{j}\bra{i} = 
\sum_{ji} \gamma_{ij} \frac{e^{-\beta H_{jj}}}{Z} \ket{i}\bra{i}
\end{equation}

Whereas, the second term becomes
\begin{equation}
\sum_{ij}\frac{\gamma_{ij}}{2}\ket{j}\bra{j} \frac{e^{-\beta\hat H}}{Z} + \frac{\gamma_{ij}}{2}\frac{e^{-\beta\hat H}}{Z}\ket{j}\bra{j}=
\sum_{ijk} \frac{1}{2}\left(\gamma_{ij}\frac{e^{-\beta H_{jk}}}{Z} + \gamma_{ik}\frac{e^{-\beta H_{jk}}}{Z}\right)\ket{j}\bra{k}
\end{equation}

We can combine together all the terms obtaining
\begin{equation}
\begin{split}
\sum_{ji} \left(\gamma_{ji}\frac{e^{-\beta H_{ii}}}{Z} - \gamma_{ij}\frac{e^{-\beta H_{jj}}}{Z}\right)\ket{j}\bra{j} = 0\\
\sum_{jk} \frac{1}{2}\left(\sum_i\gamma_{ij} + \sum_i\gamma_{ik}   \right)\frac{e^{-\beta H_{jk}}}{Z}\ket{j}\bra{k} = 0\\
\end{split}
\end{equation}
The first equation is again the Kirchhoff's current law, while the second is a condition over the parameter $\gamma$.

If the previous condition are satisfy the stationary distribution of the system is the canonical one. In our case it becomes
\begin{equation}
\hat\rho^* = \frac{e^{-\beta\hat L}}{Z} \qquad Z = \Tr[e^{-\beta\hat L}]  
\end{equation}
that is the density matrix \eqref{density_matrix} introduced by De Domenico to identify networks.
\end{comment}


\subsection{Analogy with the Network's Entropy}

The equation \eqref{Lindblad_energy_jump} describes the evolution of a quantum walk in the presence of classical noise. The temperature $T$ determines the noise strength.
Increasing the parameter $\beta$ suppresses the energy state with high eigenvalue, until just the zero eigenstate remains. Thus, the parameter $\beta$ allows us to analyze the information's spread along paths of chosen eigenstate. The von Neumann entropy measures the uncertainty over the state of the particle. 

This entropy behaves as the network's entropy \eqref{entropy}. 
The network's entropy measures the complexity of the spread of information across the network, which depends on the parameter $\beta$. 
In fact, for low values of $\beta$, all the possible channel are available, resulting in a complex spread of information and, thus, high entropy. As $\beta$ increases, the channels with high eigenvalue are suppressed until only the zero eigenstate remains, thus, zero entropy. 

In addition, we can apply a Wick rotation to the quantum dynamics \eqref{Lindblad_energy_jump} returning to a special random walk of classical particle. This rotation connect the stationary distribution at inverse temperature $\beta$ \eqref{pretended_stationary_distribution} with the propagator at time $t$ \eqref{random_walk_solution}.
\begin{equation}
e^{-\beta\hat L} \rightarrow e^{-t\hat L}
\end{equation}
Thus, cooling down the quantum system is analogous to the temporal evolution of the classical one. In fact, the two limits $\beta \rightarrow \infty$ and $t \rightarrow \infty$ converge to the same distribution: the system will be entirely in the zero eigenstate of the Laplacian.
Moreover, the density distribution \eqref{pretended_stationary_distribution} is always in the maximal entropy state. As a consequence, also the distribution for the classical random walk should cross state with maximal entropy.

The complexity of the possible paths is encoded in the von Neumann entropy as explained in chapter \ref{C_Density_Matrix}.
The entropy allows us to classify different networks based on the dynamical properties of the network itself. 
We can achieve it introducing the Kullback-Leibler divergence \eqref{KL_divergence} and the Jensen-Shannon divergence \eqref{JS_metric}.
However, because these quantities employ the trace of a Laplacian's function, the entropy studies only the spectral properties of the network. Therefore, networks with same spectrum but different structures and eigenstates may be indistinguishable using these methods.


\section{Generalization to other dynamic}

Until now, we have examined only the random walk on network, but this framework can be generalize to other more complex dynamics on network \cite{De_Domenico_2023}.
The dynamics should be linear such that the evolution of the observable per node $i$ are 
\begin{equation}\label{general_dynamics}
    \frac{d}{dt} x_i = \sum_j H_{ij} x_j,
\end{equation}
where $H_{ij}$ controls the evolution of the system.
For the continuos time random walk the control matrix coincides with the Laplacian.
In order to apply the Wick rotation and we obtain the quantum version of the system.
Let $\{\ket{i}\}_{i<N}$, $N$ is the number of node, be a orthonormal basis for the Hilbert space $\mathcal{H}$, the state $\ket{\psi}$ is defined as
\begin{equation}
    \ket{\psi}= \sum_i\sqrt{x_i}\ket{i}
\end{equation}
such that $x_i = |\braket{i}{\psi}|^2$.
The evolution follow the Schrödinger equation
\begin{equation}
    \frac{d}{dt}\ket{\psi(t)} = -i\hat H\ket{\psi(t)}
\end{equation}
where 
$\hat H = \sum_{ij} H_{ij} \ket{i}\bra{j}$ is the control operator or the chosen dynamics. To satisfy the Schrödinger equation the control operator must be symmetric.
Now we can add thermal noise arriving at quantum master equation \eqref{Lindblad_energy_jump} which has stationary distribution 
\begin{equation}
    \rho^*= \frac{1}{Z}e^{-\beta\hat H}
\end{equation}
with $Z= \Tr\left[e^{-\beta\hat H}\right]$ is the partition function.

Therefore, the network entropy for network under the dynamics \eqref{general_dynamics} is 
\begin{equation}
    S = -\Tr\left[\frac{1}{Z}e^{-\beta\hat H}\ln\left(\frac{1}{Z}e^{-\beta\hat H}\right) \right].
\end{equation}
Based on the considered dynamics the network will have a different entropy's value.



\section{Symmetry breaking}

Until now, we have considered the network holding the detail balance condition and, therefore, be mapped in a symmetric matrix; 
but the majority of the networks do not satisfy this condition. 

To deal with them, we modify slightly the Lindblad master equation \eqref{stochastic_lindblad_master}. 
As a matter of fact, in the chapter \ref{C_Lindblad} we have analyzed also the case where the interaction with the environment is not symmetric \eqref{environment_coefficients}. Thus, taking the dissipative part of the equation \eqref{C_rotating_wave} in the Schrödinger picture with the coefficients $\Gamma_{ij} = L_{ij}$ and the jump operators $J_{ij} = \ket{i}\bra{j}$ we obtain 
\begin{equation}
    \frac{d\hat\rho(t)}{dt} = \sum_{ij}\Gamma_{ij}\left[\hat J_j\hat\rho(t),\hat J_i^\dagger\right]+\Gamma_{ji}^\dagger\left[\hat J_j,\hat\rho(t)\hat J_i^\dagger\right].
\end{equation}
Isolating the symmetric and antisymmetric part of the Laplacian, respectively $\gamma_{ij} = \left(L_{ij} + L_{ji}\right)$ and $\pi_{ij} =  \frac{-i}{2}\left(L_{ij}-L_{ji}\right)$ such that $\Gamma_{ij}(\omega) =\frac{1}{2}\gamma_{ij}(\omega)+i\pi_{ij}(\omega)$, we arrive to the equation
\begin{equation}
    \frac{d\hat\rho(t)}{dt} = \sum_{ij}\gamma_{ij}\hat J_j\hat\rho(t)\hat J_i^\dagger -\frac{\gamma_{ij}}{2}\left\{\hat J_i^\dagger\hat J_j,\hat\rho(t)\right\} + i\pi_{ij}\left[\hat J_i^\dagger\hat J_j,\hat\rho(t)\right],
\end{equation}
where $[\cdot,\cdot]$ and $\{\cdot,\cdot\}$ are respectively the commutator and anticommutator.

Let define a new Hamiltonian $\hat H_{A} = \sum_{ij}\pi_{ij}\hat J_i^\dagger\hat J_j$ that encodes the dynamics of the not symmetric part. 
It give origin to a coherent dynamics that follow the Von Neumann equation. As a matter of fact the total dynamics can be written as
\begin{equation}\label{antisymmetric_master_equation}
    \frac{d\hat\rho(t)}{dt} = i\left[\hat H_{A},\hat\rho(t)\right] + \sum_{ij}\gamma_{ij}\hat J_j\hat\rho(t)\hat J_i^\dagger -\frac{\gamma_{ij}}{2}\left\{\hat J_i^\dagger\hat J_j,\hat\rho(t)\right\}.
\end{equation}

The dynamics \eqref{antisymmetric_master_equation} does not converge no more to a stationary state due to the Von Neumann part.
We can generalize as in the \eqref{stochastic_lindblad_master} 

\begin{equation}
    \frac{d}{dt}\hat \rho = -i\left[\hat H + \omega\hat H_{A},\hat\rho\right] + \sum_{ij}\gamma_{ij}\left[\hat J_{ij} \hat\rho\hat J_{ij}^\dagger -\frac{1}{2} \left\{ \hat J_{ij}^\dagger \hat J_{ij}, \hat\rho\right\}\right].
\end{equation}
where $\hat H$ is the hermitian part of the Laplacian operator.


\bigskip

The quantum walk on network with a not hermitian Hamiltonian has a evolution operator that is not unitary. It can be split between the hermitian $\hat H_S$ and anti-hermitian $\hat H_A$ components as
\begin{equation}\label{not_hermitian_time_operator}
    U(t,0) = e^{-it\hat H_S} e^{-t\hat H_A}
\end{equation}
with
\begin{equation}
    \hat H_S =  \frac{1}{2}\left(\hat H + \hat H^\dagger\right) \qquad \qquad
    \hat H_A = \frac{-i}{2}\left(\hat H - \hat H^\dagger\right)
\end{equation}

If the Hamiltonian $\hat H$ is positive defined, the second exponential of the equation \eqref{not_hermitian_time_operator} dissipate energy. 
Thus, the system can be studied again with a Lindblad master equation \eqref{Lindbladian} in the form
\begin{equation}
    \frac{d}{dt}\hat\rho = -i\left[\hat H_S,\hat\rho\right] + \sum_k \gamma_k \left(\hat J_k \hat\rho \hat J^\dagger_k - \frac{1}{2}\left\{ \hat J^\dagger_k\hat J_k, \hat\rho\right\} \right).
\end{equation}
The jumps operators must reproduce the same dissipation as the anti-hermitian operator $\hat H_A$.

The damping coefficients do not have to hold the Kirchhoff law. Thus, the system may not converge to a stationary state but may persist some stationary density currents.