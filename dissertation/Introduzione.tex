\chapter*{Introduction}
\markboth{\textbf{INTRODUCTION}}{\textbf{INTRODUCTION}}
\addcontentsline{toc}{chapter}{Introduction}

The traffic jam in the morning, the airport traffic, but also the functioning of the brain can be study as a network problem. 
This kind of problem can be naturally study through network: the vertices indicates the elements and link the interaction between them, for the airport traffic the airport are the vertices and the flights the link, for the brain, instead, the neurons and the synapses. 

The first ideas of graph theory was introduce by Euler in 1736 with his famous problem about the Seven Bridges of Königsberg \cite{Euler}. The dilemma was about the find a route, if it exist, that cross all the seven bridges that connect the two island on the Pregel river with the rest of the city once and only one.  Euler solve the problem eliminating all the irrelevant information and focus just on the sequence of the bridges. In other words, he reformulates the problem considering the island and the rivers banks as nodes and the bridges as link.
Graph theory was begun. ne of the success of this theory is the proof of the five color problem. It declare that given a plane divided into regions, such as a political map, those regions can be always colored using no more that five different colors, such that two neighboring regions do not have the same color \cite{Heawood_color_theorem,Ringel_Color_Theorem}.

With the spread of the graph theory through different fields and the complexity of the networks, the necessity to find a way to reproduce reliable  artificial networks with basic algorithm was crucial. The first answer was given by Erd\H{o}s and Rényi \cite{erdos-renyi1960} with their random graphs. This model is widely studied and understood and use a base to for ages. 
However, due to the increasing our data and computational power, the Erd\H{o}s-Rényi model starts failing to capture the behavior of the real network like Internet. In fact, real networks present strong hubs and short distances, features that this model has not. 
To answer the new question, in the last 40 years many model has been proposed and studied, each one with their unique properties \cite{Barabasi_Albert_1999,Watts-Strogatz_1998}. The network theory has born. 

The natural problem we are facing with the network are dynamical, so we need to add dynamical model to the network. The simplest dynamics we can consider is the random walk, a single particle wandering across the network. 
Regardless its simplicity, the random walk on network has reveled very powerful and it is used to build several algorithm \cite{Classic_random_walk,Pagerank_2015,Pagerank_1998}.


With the new progress in quantum computing, many point of contact between quantum information and network theory arose.
One of the most important model is the quantum walk. It is the quantum equivalent of the classic random walk on network but, because the quantum effects, its behavior is different \cite{Kempe}. There are two different way to deal with time. We can consider time discrete and the motion is ruled by a quantum coin tossed at ech timestep \cite{Coin_quantum_walk}. Otherwise, we consider the time continuos and the evolution is ruled by the Schrödinger equation with the Laplacian of the network as Hamiltonian \cite{Farhi_98}. In this dissertation we focus on the second type.


However, there is a lack of a unified theoretical framework in the network theory, particularly concerning information theory and entropy. 
In the literature, there have been various attempts to formulate entropy for networks. A notable contribution was made by Bianconi \cite{Bianconi_entropy_1,Bianconi_entropy_2}, who considered an ensemble of all possible graphs with specific properties, but this approach neglected the dynamical aspects of the system.
Another attempt was made by De Domenico \cite{De_Domenico_2016}. Starting from the Estrada communicability matrix \cite{Estrada_2008}, he defines the network entropy at time $t$ as $\Tr[\rho \ln \rho]$ with $\rho = e^{-tL}$ as density matrix and $L$ as the Laplacian matrix. This type of entropy not only captures the topological features of the network but also its dynamical behavior. In fact, the entropy capture the property of the relaxation on the network. Starting from there, we can expand the information related quantities introducing The Kullback Lieber divergence and the Jensen Shannon divergence. This two quantities can be employ to distinguish between networks \cite{multilayer}.


The density matrix introduce by De Domenico is similar to the density matrix of a quantum canonical ensemble with the Laplacian as Hamiltonian. This fact suggests a link between the network entropy and the continuos time random walk.
This dissertation used technique borrowed from the study of open quantum system to light up this connection
  

The aim of this dissertation is to understand physics behind the new formulation for the information theory for complex network and its connection with the relaxation of a dynamics over a network. This results can be use in several fields: from the study of the interaction between the amino acids in the proteins, to the management of the urban traffic, passing through the social interaction in Internet.
The theoretical calculations come with numerical simulation made in python.

The chapter \ref{Network_Theory} is an introduction to the Network theory. There we explain the foundation of network, the classic random walk and the quantum version.

The Chapter \ref{C_Density_Matrix} focuses on the network entropy. Starting from the Estrada communicability matrix, it defines the density matrix for network and the network entropy and their applications.

Before entering the last argument, in the chapter \ref{C_Lindblad} there is a introduction to the Markovian open quantum system and the Gorini-Kossakowski-Sudarshan-Lindblad master equation. 

The last chapter explains the connection between the quantum walk with thermal noise and the network entropy.
