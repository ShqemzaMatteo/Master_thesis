\chapter*{Introduction}
\markboth{\textbf{INTRODUCTION}}{\textbf{INTRODUCTION}}
\addcontentsline{toc}{chapter}{Introduction}

Traffic congestion in the morning, airport scheduling, and even the functioning of the brain are all systems that can be naturally analyzed through complex networks. In such networks, vertices represent elements, and links define their interactions: airports and neurons serve as vertices, while flights and synapses act as links.

The foundations of graph theory trace back to 1736, when Euler introduced his famous problem about the Seven Bridges of Königsberg \cite{Euler}. The challenge was about finding a route, if it exist, that cross all the seven bridges connecting the two island on the Pregel river with the rest of the city once and only one. Euler solved the problem eliminating all the irrelevant information and focusing just on the sequence of the bridges. In other words, he reformulates the problem considering the island and the rivers banks as nodes and the bridges as links.
Graph theory was begun. One of the success of this theory was the proof of the five color problem. It declared that given a plane divided into regions, such as a political map, those regions could be always colored using no more that five different colors, such that two neighboring regions did not have the same color \cite{Heawood_color_theorem,Ringel_Color_Theorem}.

With the spread of the graph theory through different fields and the complexity of the networks, the necessity to find a way to reproduce reliable  artificial networks with basic algorithm was crucial. The first answer was given by Erd\H{o}s and Rényi \cite{erdos-renyi1960,Erdos-renyi1959} with their random graphs. This model is widely studied and understood; it was used a base to for ages. 
However, due to the increasing of the data and computational power, the Erd\H{o}s-Rényi model started failing to capture the behavior of the real network like Internet. In fact, real networks present strong hubs and short distances, features that this model did not have. 
To answer the new question, in the last 40 years many model has been proposed and studied, each one with their unique properties \cite{Barabasi_Albert_1999,Watts-Strogatz_1998}. The network theory was born. 

Real-world network problems are intrinsic dynamic, necessitating the integration of dynamical models. The simplest dynamics we can consider is the random walk, a single particle wandering across the network. 
Despite its simplicity, the random walk on network has proven to be a powerful tool, forming the basis of various algorithms \cite{Classic_random_walk,Pagerank_2015,Pagerank_1998}.

With the progress in quantum computing, many point of contact between quantum information and network theory arose.
One of the most important model is the quantum walk. It is the quantum equivalent of the classic random walk on network but, because of the quantum effects, its behavior is different \cite{Kempe}. There are two different way to deal with time. We can consider time discrete and the motion is ruled by a quantum coin tossed at each timestep \cite{Coin_quantum_walk}. Otherwise, we consider the time continuous and the evolution is ruled by the Schrödinger equation with the Laplacian of the network as Hamiltonian \cite{Farhi_98}. In this dissertation we focus on the latter.

However, there is a lack of a unified theoretical framework in the network theory, particularly concerning information theory and entropy. 
In the literature, there have been various attempts to formulate entropy for networks. A notable contribution was made by Bianconi \cite{Bianconi_entropy_1,Bianconi_entropy_2}, who considered an ensemble of all possible graphs with specific properties, but this approach neglected the dynamical aspects of the system.
Another attempt was made by De Domenico \cite{De_Domenico_2016}. Starting from the Estrada communicability matrix \cite{Estrada_2008}, he defined the network entropy at time $t$ as $\Tr[\rho \ln \rho]$ with $\rho = e^{-tL}$ as density matrix and $L$ as the Laplacian matrix. This type of entropy not only captured the topological features of the network but also its dynamical behavior. In fact, the entropy held the property of the relaxation of a random walk on the network. Starting from there, we expanded the information related quantities introducing The Kullback Leibler divergence and the Jensen Shannon divergence. This two quantities could be employed to distinguish between networks \cite{multilayer}.

The density matrix introduced by De Domenico is reminiscent of the density matrix of a quantum canonical ensemble with the Laplacian as Hamiltonian. This observation suggests a deep connection between network entropy and continuous time quantum walks.
The aim of this dissertation is to explore this connection using techniques borrowed from open quantum system, specifically the Lindblad master equation. These results can be use in several fields: from the study of the interaction between the amino acids in the proteins, to the management of the urban traffic, passing through the social interaction on the Internet.
The theoretical calculations come with numerical simulations made in python.

The dissertation is structured as follows.
The Chapter \ref{Network_Theory} is an introduction to the Network theory. There we explain the foundation of network, the classic random walk and the quantum version.

The Chapter \ref{C_Density_Matrix} focuses on the network entropy. Starting from the Estrada communicability matrix, it defines the density matrix for network and the network entropy and their applications.

Before entering the last argument, in the Chapter \ref{C_Lindblad} there is an introduction to the Markovian open quantum system and the Gorini-Kossakowski-Sudarshan-Lindblad master equation. 

The last Chapter explains the connection between the quantum walk with thermal noise and the network entropy.