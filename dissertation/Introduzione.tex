\chapter*{Introduction}
\markboth{\textbf{INTRODUCTION}}{\textbf{INTRODUCTION}}
\addcontentsline{toc}{chapter}{Introduction}

The traffic jam in the morning, the airport traffic, but also the functioning of the brain can be study as a network problem. 
A lot of studies are made to analyze the equilibrium point of a network but only few achieve to understand what happen in the not equilibrium point. This work aim to add a piece to the knowledge in this dark topic.




Due to the vast range of applications, there is a lack of a unified theoretical framework in the network theory, particularly concerning information theory and entropy. 
In the literature, there have been various attempts to formulate entropy for networks. A notable contribution was made by Bianconi \cite{Bianconi_entropy_1}\cite{Bianconi_entropy_2}, who considered an ensemble of all possible graphs with specific properties, but this approach neglected the dynamical aspects of the system.
Another attempt was made by De Domenico \cite{De_Domenico_2016}; he defines the entropy at time $t$ as $\Tr[\rho \ln \rho]$ with $\rho = e^{-tL}$ as density matrix and $L$ as the Laplacian matrix. This type of entropy not only captures the topological features of the network but also its dynamical behavior.


